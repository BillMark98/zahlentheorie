\documentclass{article}
\usepackage{amssymb,latexsym,amsmath,amsthm,mathtools,mathrsfs,bm}
\usepackage{tikz,tikz-3dplot} 
\usepackage{fouriernc}
% for vertical and horizontal lines in matrix
\usepackage{mleftright}

% % for not equivalent
% \usepackage{unicode-math}
\renewcommand{\baselinestretch}{1.5} 

\newtheorem{theorem}{Theorem}[section]
\newtheorem{lemma}[theorem]{Lemma}
\newtheorem{definition}[theorem]{Definition}
\newtheorem{Algorithmus}{Algorithmus}
\newtheorem{example}[theorem]{Example}
\newtheorem{collary}[theorem]{Corollary}
\newtheorem{prop}[theorem]{Proposition}
\newtheorem{remark}[theorem]{Remark}
\numberwithin{equation}{theorem}
\numberwithin{figure}{theorem}

\setcounter{MaxMatrixCols}{20}

\newcommand{\GExtend}{\ensuremath{\widetilde{G}}}
\newcommand{\Ftwo}{\ensuremath{\mathbb{F}_2}}
\newcommand{\tCompleteDesign}{\ensuremath{\bm{t\mbox{-}(v,k,\lambda)\; design}}}
\newcommand{\tParamDesign}[4]{\ensuremath{\bm{#1\mbox{-}(#2,#3,#4)\; design}}}
\newcommand{\steinerSystem}[3]{\ensuremath{{S(#1,#2,#3)}}}
\newcommand{\myBlock}{\textbf{block}}
\newcommand{\pluralMyBlock}{\textbf{blocks}}
\newcommand{\xDesign}[1]{\ensuremath{\bm{#1\mbox{-}design}}}
\newcommand{\dDes}{\ensuremath{\mathscr{D}}}
% codes symbol
\newcommand{\cCodes}{\ensuremath{\widetilde{\mathscr{G}}}}
\newcommand{\simpleCodes}{\ensuremath{\mathrm{C}}}
\newcommand{\cCodesVertical}{\cCodes^{\bot}}
\newcommand{\buildVertical}[1]{\ensuremath{#1^{\bot}}}
\newcommand{\buildLattice}[1]{\ensuremath{\Gamma_{#1}}}
\newcommand{\weightEnumerator}[3]{\ensuremath{W_{#1}(#2,#3)}}


% theta function and modular form
\newcommand{\thetaFunction}[1]{\ensuremath{\vartheta_{#1}}}
\newcommand{\modularK}[1]{\ensuremath{M_{#1}}}
\newcommand{\eisenSeries}[1]{\ensuremath{E_{#1}}}

\newcommand{\tildcCodes}{\ensuremath{\widetilde{\mathscr{C}}}}
\newcommand{\tSet}[1]{\ensuremath{\bm{#1\mbox{-}{set}}}}
\newcommand{\tSubset}[1]{\ensuremath{\bm{#1\mbox{-}{subset}}}}
\newcommand{\code}[3]{\ensuremath{\bm{(#1,#2,#3)\mbox{-}code}}}
\newcommand{\linearCode}[3]{\ensuremath{\bm{(#1,#2,#3)\mbox{-}code}}}
\newcommand{\ftwoN}[1]{\ensuremath{\mathbb{F}_2}^{#1}}
\newcommand{\myMatrixRing}[2]{\ensuremath{#1^{#2\times#2}}}
\newcommand{\wloe}{w.l.o.e}

\newcommand{\polyAB}[2]{\ensuremath{#1^4#2^4\,(#1^4 - #2^4)^4}}
% weight function
\newcommand{\wt}[1]{\ensuremath{\text{wt}(#1)}}
% distance function
\newcommand{\dist}[2]{\ensuremath{\text{d}(#1,#2)}}

% some number ring, field
\newcommand{\Integer}{\ensuremath{\mathbb{Z}}}
\newcommand{\Real}{\ensuremath{\mathbb{R}}}
\newcommand{\Complex}{\ensuremath{\mathbb{C}}}
\newcommand{\imaginary}{\ensuremath{i}}
\newcommand{\HalfPlane}{\ensuremath{\mathbb{H}}}

% some variables
\newcommand{\twoFourDegreeAB}{\ensuremath{A^4\,B^4\;(A^4 \,-\, B^4)^4}}
\title{Seminar}
\date{\today}
\author{Panwei Hu}

\begin{document}
\pagenumbering{gobble}
  \maketitle
  \newpage

\tableofcontents
	\newpage
  \newpage
  \pagenumbering{arabic}
\newpage
\section{Introduction}

This report is based on the book \cite{ebeling2013lattices} from chapter 2.7 to 2.9
We have dealt mainly with lattices in the previous two lectures. Now we will discuss the relationship between lattices and code. So the first step is to construct a bridge between these two. We will discuss the binary case only.\\
Before discussing further the codes, let us recall some results from the theta function and modular forms, which we simply state without proof.\\
Let $\Gamma \subset \Real^n$ be a lattice. We define $q = e^{2\pi\imaginary\tau}$, for $\tau \in \HalfPlane$.\\
The theta function $\thetaFunction{\Gamma}$ of the lattice $\Gamma$ is defined as 
\[
	\thetaFunction{\Gamma}(\tau) \, := \,\sum_{x \in \Gamma} q^{\frac{1}{2}(x\cdot x)} 
\] 
We recall the main theorem introduced on the lecture of Mr.Caspers. 
\begin{theorem}\label{evenUniLattice}
Let $\Gamma$ be an even unimodular lattice in $\Real^n$. Then
\begin{enumerate}
	\item n $\equiv$ 0 $\pmod 8$
	\item {\thetaFunction{\Gamma}} is a modular form of weight $\frac{n}{2}$
\end{enumerate}
\end{theorem}

\begin{prop}\label{latticeAndDualThetaProp}
	$\thetaFunction{\Gamma}(-\frac{1}{\tau})$ = $(\frac{\tau}{i})^{\frac{n}{2}} \frac{1}{\mbox{vol}(\,\Real^n/\Gamma)}\,\thetaFunction{\Gamma^{*}}(\tau)$
\end{prop}
\begin{definition}
Let $k \in \Integer$, $k$ even, $k > 2.$ The series
\[
	G_k(\tau) \; = \sum_{\substack{(m,n) \in \Integer^2\\				 (m,n) \neq (0,0)}} \frac{1}{(m\tau + n)^k}
\]
is called the \emph{Eisenstein series of index k}.
The \emph{normalized Eisenstein series} is defined as
\[
	\eisenSeries{k}(\tau) := \frac{1}{2\zeta(k)} G_k(\tau)
\]
where $\zeta(k)$ is the \emph{Rieann $\zeta$-function}
\end{definition}
We define
\[
	\Delta := \frac{1}{1728}\,(\eisenSeries{4}^3 \, - \, \eisenSeries{6}^2) 
\]
The $\Delta$ function has the following form:
\begin{prop}\label{deltaExpansionProp}
\[
	\Delta = q \prod_{r = 1}^{\infty}(1 - q^r)^{24}
\]
\end{prop}
\begin{definition}
	The modular form of weight k is a $\Complex$-vector space, and is denoted as {\modularK{k}}. $\modularK{k}^0$ denotes the $\Complex$-vector space of cusp forms of weight k.
\end{definition}
\begin{theorem}\label{cuspFormTheo}\hfill
\begin{enumerate}
	\item $\modularK{k}$ = 0 for $k$ odd, for $k < 0$ and for k = 2
	\item $\modularK{0} = \Complex$, $\modularK{0}^0 = 0$ and for $k$ = 4, 6, 8, 10, $\modularK{k}^0 $ = 0, $\modularK{k}$ = $\Complex \cdot \eisenSeries{k}$\
	\item Multiplication by $\Delta = \frac{1}{1728}(\eisenSeries{4}^3 -\eisenSeries{6}^2)$ defines an isomorphism of $\modularK{k-12}$ onto $\modularK{k}^0$
\end{enumerate}
\end{theorem}
\begin{collary}\label{E4E6collary}
The algebra $M$ of mdular forms is isomorphic to the polynomial algebra $\Complex[{\eisenSeries{4}}, {\eisenSeries{6}}]$.
\end{collary}
\begin{prop}\label{E8isomorphProp}
Let $\Gamma$ be an even unimodular lattice in $\Real^8$. Then
$\Gamma$ is isomorphic to $E_8$. And the theta function 
\[
	\thetaFunction{\Gamma} = E_4
\]
\end{prop}
We consider first the standard $\Integer^{n}$ lattices and define the canonical projection onto $\Ftwo^{n}$
\[
	\rho: \Integer^n \rightarrow \Ftwo^{n}, x \mapsto \bar{x}
\]
For a binary {\linearCode{n}{k}{d}} {\simpleCodes}. Since $|\frac{\Ftwo^n}{\simpleCodes} = 2^{n-k}|$, we have $|\frac{\Integer^{n}}{\rho^{-1}(\simpleCodes)} = 2^{n-k}|$. In particular $\rho^{-1}(\simpleCodes)$ is a lattice in $\Real^n$
By defining $\Gamma_{\simpleCodes} := \frac{1}{\sqrt{2}}\rho^{-1}(\simpleCodes)$. We have the following proposition which we state without proof.
\begin{prop}\label{codeLatticeProp} Let {\simpleCodes} be a linear code.
	\begin{enumerate}
		\item {\simpleCodes} $\subset$ {\buildVertical{\simpleCodes}}, if and only if {\buildLattice{\simpleCodes}} is an integral Lattice
		\item {\simpleCodes} is doubly even if and only if {\buildLattice{\simpleCodes}} is an even Lattice.
		\item {\simpleCodes} is self-dual if and only if {\buildLattice{\simpleCodes}} is a unimodular Lattice.
	\end{enumerate}
\end{prop}
\begin{lemma}\label{latticeDualcodeVertLemma}
Let {\simpleCodes} $\subset \Ftwo^n$ be a binary linear code.Then
\[
	\buildLattice{\simpleCodes}^* = \buildLattice{\buildVertical{\simpleCodes}}
\]
\end{lemma}
We recall some definitions from the coding theory
\begin{definition} Let {\simpleCodes} $\subset$ $\Ftwo^{n}$ a \linearCode{n}{k}{d}. 
	\begin{enumerate}
		\item Let $c \in \simpleCodes$, the hamming weight is defined as \wt{c} := |\{\,i| c(i) = 1\,\}|.
		\item The \emph{Hamming weight enumerator} of {\simpleCodes} is a polynomial of $\Integer[X,Y]$ defined as:
		\begin{align*}
			\weightEnumerator{\simpleCodes}{X}{Y} &:= \sum_{c \in \simpleCodes}X^{n - \wt{c}}Y^{\wt{c}}\\
			&= \sum_{i = 0}^{i = n}A_i X^{n - i}Y^i
		\end{align*}
		where $A_i$ denotes the number of codewords of length $i$ in {\simpleCodes}
	\end{enumerate}
\end{definition}
\begin{remark}
We recognize $\langle , \rangle$ as the standard inner product of vectors with values in real number. For simplicity, we will denote $\langle u,v \rangle$ as $u \cdot v$, if it is clear in context that we are having a scalar product of two vectors. For the latter case, if we have two identical vectors $u$, we will abuse the notation $u^2$.
\end{remark}
\begin{example}\label{hammingExample}
	For the Hamming code H $\subset$ $\Ftwo^7$ with the generator matrix:
	\[
	\left[
		\begin{matrix}
			1 &0 &0 &0 &0 &1 &1 \\
			0 &1 &0 &0 &1 &0 &1 \\
			0 &0 &1 &0 &1 &1 &0 \\
			0 &0 &0 &1 &1 &1 &1
		\end{matrix}
	\right]
	\]
	has the weight enumerator:
	\[
		\weightEnumerator{H}{X}{Y} = X^7 + 7X^4Y^3 + 7X^3Y^4 + 1
	\]
	In this case, it's also convenient to find the generator matrix of its dual code.
	\[
		\left[
			\begin{matrix}
				0 & 1 & 1 & 1 & 1 & 0 & 0 \\
				1 & 0 & 1 & 1 & 0 & 1 & 0 \\
				1 & 1 & 0 & 1 & 0 & 0 & 1
			\end{matrix}
		\right]
	\]
	And we can find that the weight enumerator is 
	\[
		\weightEnumerator{\buildVertical{H}}{X}{Y} = X^7 + 7X^3Y^4 + Y^7
 	\]
 	By adding an extra parity bit in the generator matrix of $H$, we obtained the extended Hamming code $\widetilde{H}$.\\
 	The weight enumerator is
 	\[
 		\weightEnumerator{\widetilde{H}} = X^8 + 14X^4Y^4 + Y^8
 	\]
\end{example}
Based on the lecture of theta series and modular forms, we have the following proposition.
\begin{prop}
Let {\simpleCodes} $\subset \Ftwo^n$ be a self-dual doubly even code. Then n $\equiv$ 0 $\pmod 8$ 
\end{prop}
\begin{proof}
Using the main theorem in the lecture of theta series and proposition \ref{codeLatticeProp}
\end{proof}
We now want to prove a relationship between $A_8$ and $A_4$ for the special case $n = 24$ and $\simpleCodes$ self-dual and doubly even.
\begin{prop}\label{A8A4}
Let $n$ = 24. $\simpleCodes \subset \Ftwo^{24}$ a self-dual doubly even code, and let {\weightEnumerator{\simpleCodes}{X}{Y}} be the Hamming weight enumerator. Then we have $A_i$ = 0, if $i$ $\not\equiv 0 \pmod 4$ and $A_8 = 759 - 4 A_4$
\end{prop}
\begin{proof}
Since {\simpleCodes} is self-dual doubly even, it follows that the theta function of $\buildLattice{\simpleCodes}$ is a modular form of weight $\frac{n}{2}$.\\
\[
	\thetaFunction{\buildLattice{\simpleCodes}} = \sum_{r = 0}^{\infty} a_r \, q^r
\]
where $a_r$ denotes the number of elements $x = \frac{1}{\sqrt{2}}(c + 2y) \in \buildLattice{\simpleCodes},\, c \in \simpleCodes, \, y\in \Integer^{24}$, with $x^2 = 2r$. \\
The idea of the proof is to first find out a relationship between $a_1, a_2$ and $A_4, A_8$. By using the following lemma \ref{a2a0a1} and proposition \ref{a2a1}, we could prove the equation.\\
In this proof, we do the first step of the proof, i.e. find the relationship between $a_1, a_2$ and $A_4, A_8$.\\
For further analysis, it's helpful to note that:
\begin{equation}\label{xsumCYEqu}
	x^2 = \frac{1}{2}(c \,+\,2y)^2 = \sum_{i = 1}^{24}(c_i \, + \, 2y_i)^2
\end{equation}
For $a_1$, we have
\begin{equation}\label{a1Eqn}
	a_1 = 24 \cdot 2 \,+ \,A_4
\end{equation}
It can be understood as following:\\
Note first that from (\ref{xsumCYEqu})
\begin{equation}\label{sumCode_a1Eqn}
	\sum_{i = 1}^{24}(c_i \, + \, 2y_i)^2 \overset{!}{=} 4
\end{equation}
a) if $c$ = $0^{24}$, where $0^{24}$ denotes the zero vector in $\Ftwo^{24}$, then since $(\pm2)^2 = 4$. We can choose one position to fill $\pm2$ and 0 elsewhere. This gives $24 \cdot 2$ choices\\
b) if $\wt{c} = 4$, then at the support(see definition \ref{coverSupportHoldDefinition}) of $c$, $y$ can be chosen $\pm1$ and 0 elsewhere. This gives $2^4 A_4$ choices.
Since for other cases, (\ref{sumCode_a1Eqn}) would be impossible. These are the only two possible cases. Combining the result, we have consequently (\ref{a1Eqn}).\\
For $a_2$, we do the similar analysis. First, we know that
\begin{equation}\label{sumCY_a2Eqn}
	\sum_{i = 1}^{24}(c_i \, + \, 2y_i)^2 \overset{!}{=} 8
\end{equation}
There are three cases:
a) $c = 0^{24}$, then since $(\pm2)^2 + (\pm2)^2 = 8$, we have $\binom{24}{2} \cdot 2^2$ choices for the $y_i$.\\
b) $\wt{c} = 4$, then $y$ could choose $\pm1$ at the support of $c$ and choose $\pm1$ at another position where $c$ is 0, since $(\pm1)^2 + (\pm1)^2+ (\pm1)^2+ (\pm1)^2 + (\pm2)^2 = 8$. This gives $A_4 \cdot 2^4 \cdot 20 \cdot 2$.\\
c) $wt{c} = 8$, then $y$ could only choose $\pm1$ at the support of $c$. This gives $A_8 \cdot 2^8$.\\
These are the only possible cases. Combining these results, it follows that 
\begin{equation}\label{a2A8A4Eqn}
	a_2 = 2^8A_8 \, + \,  16 A_4 \cdot 20 \cdot 2 \, + \, \binom{24}{2}\cdot 4
\end{equation}
\end{proof}
We state the following lemma without proof.
\begin{lemma}\label{a2a0a1}
Let $f$ be a modular form of weight 12 and
\[
	f(\tau) = \sum_{r = 0}^{\infty}a_rq^r
\]
its power series expansion in q. Then
\[
	a_2 = -24a_1 + 196560a_0
\]
\end{lemma}
\begin{prop}\label{a2a1}
Let $\Gamma \subset \Real^24$ be an even unimodular lattice. Then 
\[
	a_2 = 196560 - 24a_1
\] 
\end{prop}
\begin{proof}
We know that from theorem \ref{evenUniLattice}, that the theta function $\thetaFunction{\Gamma}$ is a modular form of weight 12. We now use lemma \ref{a2a0a1} combining the fact that $a_0 = 1$ for $\thetaFunction{\Gamma}$
\end{proof}
We now combine the proposition \ref{a2a1} to complete the proof of proposition \ref{A8A4}.

\begin{example}{exampleA8A4}
\begin{enumerate}
	\item
We consider the direct sum of extended Hamming code {\simpleCodes} = $\widetilde{H} \bigoplus \widetilde{H} \bigoplus \widetilde{H} \subset \Ftwo^{24}$. Since $\widetilde{H}$ is self-dual, so is the direct sum. Hence {\simpleCodes} is self-dual doubly even code with dimension 12.\\
\begin{align*}
	\weightEnumerator{\simpleCodes}{X}{Y} &= (X^8 + 14X^4Y^4 + Y^8)^3 \\
	&= X^{24} \,+\, 42X^{20}Y^4 \,+\, 591X^{16}Y^8 \, + \ldots
\end{align*}
Since 591 = 759 - 168. It conforms with what proposition \ref{A8A4} asserts.
\item
consider the extended Golay code {\cCodes} (see section \ref{golaySection} and \ref{sectionMacWill}).  {\cCodes} is a self-dual doubly even code with minimum distance 8. It follows that
\[
	A_8 = 759
\]
which we will see in section \ref{golaySection}
\end{enumerate}
\end{example}
Proposition \ref{A8A4} and \ref{a2a1} raises the question whether there exists code $\simpleCodes$ with $A_4 = 0$ and lattice where $a_1 = 0$. We will answer that question in the next section.
\newpage
\section{The Golay Code and thee Leech Lattice}\label{golaySection}
In this section, we will discuss a particular type of codes, the golay codes. We will show that there exists a unique doubly even linear code  $\GExtend \subset \Ftwo^{24}$ with $A_4 = 0$ 
The following discussion on designs is largely based on chapter 4, section 3 of \cite{roman1992coding}
We shall first need some definitions.
\begin{definition}\label{designDef} \hfill
	\begin{enumerate}
		\item A set with $t$ elements is called a \textbf{t-set}, a subset of a given set with $t$ elements is called a \textbf{t-subset}
		\item Let $V$ be a \textbf{v-set} and each element is called a \textbf{point}. Let $\mathscr{D}$ be a collection of disjoint \textbf{k-subset}. Every such \textbf{k-subset} is called a \textbf{block}. If for any \textbf{t-subset} $T$, there are exactly $\mathbf{\lambda}$ \textbf{k-subsets} $B$ from $S$, s.t, $T \subset B$ then $\mathscr{D}$ is called a \tCompleteDesign . We also sometimes simply call it a $\bm{t}$-design.
		\item A $\bm{t\mbox{-}(v,k,1)}$ design is called a \emph{Steiner System}, denoted as $\bm{S(t,k,v)}$
	\end{enumerate}
\end{definition}
\begin{example}\label{projPlane}
		The projective plane of order 2 is a $\bm{2\mbox{-}(7,3,1)}$ design, hence a $\bm{S(2,3,7)}$ \emph{Steiner System}
\end{example}

Before we begin with the proofs, let us discuss some propeties of the $\bm{t}$-design.
\begin{lemma}\label{propTdesign}
If $\mathscr{D}$ be a $\bm{t\mbox{-}design}$. Then $\mathscr{D}$ is also a $\bm{s\mbox{-}design}$, for $0 \leq s \leq t$
\end{lemma}
\begin{proof}
Let $\mathscr{D}$ be a \tCompleteDesign.
Let $S$ be a \textbf{s-subset}, we count the pairs $(T,B)$ with the property, that $S \subset T \subset B$, with $T$ a \textbf{t-subset} and $B$ a \textbf{block} of $\mathscr{D}$. We denote $\lambda_s$ as the number of \textbf{blocks} in $\mathscr{D}$ which contains $S$. We want to show that $\lambda_s$ is independent of the set $S$, but only depends on the size of the set. 
For each $\binom{v-s}{t-s}$ choices of $T$ that contains $S$, we have $\lambda$ {\pluralMyBlock} $B$ whicht contains $T$. In another way, for each of the $\lambda_s$ $\myBlock$ that contains $S$, there are $\binom{k-s}{t-s}$ choices for $T$. So we have:
\begin{align}
	\lambda_s \, \binom{k-s}{t-s} &= \lambda \, \binom{v-s}{t-s} \notag\\
	\Rightarrow  \quad \lambda_s &= \frac{\lambda \, \binom{v-s}{t-s}}{\binom{k-s}{t-s}} \label{sDesign}
\end{align}
In particular $\lambda_s$ is only dependent on the size of the subsets. So that $\mathscr{D}$ is a \xDesign{s}
\end{proof}

\begin{collary}\label{blockCount}
Let {\dDes} be a \tCompleteDesign.
Then {\dDes} has $b$ blocks, where
\[
	b = \frac{\lambda \, \binom{v}{t}}{\binom{k}{t}}
\]
\end{collary}

Let us denote $\lambda_0$ as $b$ (total number of blocks) and
$\lambda_1$ as $r$ (total number of blocks containg a given point)\\
Before proving the next lemma, we need a another fact from linear algebra.
\begin{lemma}\label{invertible}
Let $I$ be the identiy matrix in {$M := \myMatrixRing{\mathbb{R}}{n}$} Let $J \in M$ be the matrix with all entries equal to 1, and $k \in \mathbb{R}^*$ then $A := I + kJ$ is invertible
\end{lemma}
\begin{proof}
Since $(kJ)^2 = k^2 n J$ we have for the minimal polynomial of A :
\[
	\mu_A(x) := (x - 1)(x - 1 - (kn))
\]
In particular $\mu_A(0) \neq 0$, so A is invertible.
\end{proof}
\begin{lemma}\label{lambdaCommonPoints}
In a {\tParamDesign{2}{v}{k}{\lambda}} with $b = v$, $k = r$, any two blocks have exactly $\lambda$ common points.
\end{lemma}
\begin{proof}
Let $I,J$ have the same meaning as in the previous lemma for $n = v$.\\
We define the characteristic vectors of the blocks as the rows of a $b \times v$ matrix $M$. That is
\[
	M_{i,j} = 
	\begin{cases}
		1, &\text{if block $i$ contains the point $j$}\\
		0, &\text{otherwise}
	\end{cases}
\]
Denote the $i$-th block  as $B_i$, and the $j$-th point as $v_j$, $1 \leq i \leq b, 1 \leq j \leq v$
Now we have the two equations:
\begin{align}
	\sum_{j = 1}^{v} M_{i,j} &= \left|\{v\,|\,v \in B_i\} \right| \label{rowSum}\\
	\sum_{i = 1}^{b} M_{i,j} &= \left|\{i\,|\, v_j \in B_i\} \right| \label{colSum}
\end{align}
Based on these two equations, we can translate the condition that any block contains $k$ points and that any point lies in $r$ blocks in view of $k = r$ into following:
\[
	MJ = kJ = rJ = JM
\]
Meanwhile, if we denote $M = (w_1|w_2|\ldots|w_v)$ we have 
\begin{align}
	w_j^t w_i &= \,\left|\{i\,| \, v_j \in B_i \wedge v_i \in B_i\} \right|	\label{colVprod}\\
			&\overset{\wedge}{=} \,\text{``Number of blocks that contain both $v_i$ and $v_j$''}\notag
\end{align}
If we denote 
\[
 M = 
	\begin{pmatrix}
		u_1\\
		u_2\\
		\vdots\\
		u_b
	\end{pmatrix}
\]
Then we have 
\begin{align}
	u_i u_j^t &= \left|\{v_i\,|\, v_i \in B_i \wedge v_i \in B_j\} \right | \label{rowVprod}\\ 
	&\overset{\wedge}{=} \, \text{``Number of points that both block $B_i$ and $B_j$ contains''} \notag
\end{align}
The condition that we have a {\tParamDesign{2}{v}{k}{\lambda}}, i.e. any pair of points lies in $\lambda$ blocks can be expressed based on \eqref{colVprod} as follows:
\[
	M^tM = (r - \lambda)I + \lambda J
\] 
Since $M$ commutes with $J$ and $M^t = ((r - \lambda) I + \lambda J)M^{-1}$ (Lemma \ref{invertible}), $M$ also commutes with $M^t$ and thus:
\[
	MM^t = (r - \lambda)I + \lambda J
\]
Based on \eqref{rowVprod} we see that any two blocks have exactly $\lambda$ points in common.
\end{proof}

\begin{prop}\label{onlyOneDesign}
There is only one \tParamDesign{2}{11}{5}{2}
\end{prop}
\begin{proof}
We have in this case according to \eqref{sDesign}:
\begin{alignat}{3}
	b &= \frac{2\,\binom{11}{2}}{\binom{5}{2}} & = 11 &= v \notag\\
	r &= \frac{2 \, \binom{11-1}{2 - 1}}{\binom{5-1}{2 - 1}} & = 5 & \notag\phantom{v} 
\end{alignat}
Therefore, we can apply Lemma \ref{lambdaCommonPoints} and it follows that any two blocks have 2 common points.\\
{\wloe} we choose the characteristic vector of the first block as 
$(1 1 1 1 1 0 0 0 0 0 0)$. The remaining blocks coorespond to the {\tSubset{2}}\textbf{s} of the first five points.\\
we can choose the second as $(1 1 0 0 0 1 1 1 0 0 0)$ and the following 4. For the $6$-th row, there are two choices, namely
$(0 1 1 0 0 0 1 0 0 1 1)$ or $(0 1 1 0 0 0 0 1 1 0 1)$. And the rest are uniquely determined. \\
We have for the first choice the matrix:
\[
	\begin{pmatrix}
 1 & 1 & 1 & 1 & 1 & 0 & 0 & 0 & 0 & 0 & 0 \\
 1 & 1 & 0 & 0 & 0 & 1 & 1 & 1 & 0 & 0 & 0 \\
 1 & 0 & 1 & 0 & 0 & 1 & 0 & 0 & 1 & 1 & 0 \\
 1 & 0 & 0 & 1 & 0 & 0 & 1 & 0 & 1 & 0 & 1 \\
 1 & 0 & 0 & 0 & 1 & 0 & 0 & 1 & 0 & 1 & 1 \\
 0 & 1 & 1 & 0 & 0 & 0 & 1 & 0 & 0 & 1 & 1 \\
 0 & 1 & 0 & 1 & 0 & 0 & 0 & 1 & 1 & 1 & 0 \\
 0 & 1 & 0 & 0 & 1 & 1 & 0 & 0 & 1 & 0 & 1 \\
 0 & 0 & 1 & 1 & 0 & 1 & 0 & 1 & 0 & 0 & 1 \\
 0 & 0 & 1 & 0 & 1 & 0 & 1 & 1 & 1 & 0 & 0 
	\end{pmatrix}
	% \begin{tabular}{c|c|c|c|c|c|c|c|c|c|c}
 % 1 & 1 & 1 & 1 & 1 & 0 & 0 & 0 & 0 & 0 & 0 \\
 % 1 & 1 & 0 & 0 & 0 & 1 & 1 & 1 & 0 & 0 & 0
	% \end{tabular}
\]
\end{proof}
Now let us come back to codes and discuss the relationship between design and codes.\\
Let {\cCodes} be a binary code of length $n$.\\
First we define $S_d := \{c\,|\,wt(c) = d, c \in \cCodes\}$ as the set of codes of \cCodes, s.t the weight of the codes equal $d$

\begin{definition}\label{coverSupportHoldDefinition}
	Let $x$, $y$ two binary words of length $n$
	\begin{enumerate}
		\item The sphere around $x$ with radius $r$ is denoted as $B(x,r)$ with respect to the usual hamming distance.
		\item The \textbf{support} of $x$ is the set of positions in which $x$ has no zero entries.
		\item We say that $x$ \textbf{covers} $y$, if the support $y$ is a subset of $x$
		\item We say that $S_d$ \textbf{holds a } {\tParamDesign{t}{n}{d}{\lambda}}, if the supports of the codewords of $S_d$ form the {\pluralMyBlock} of a {\tParamDesign{t}{n}{d}{\lambda}}. In other words, for any {\tSet{t}} $T$, there are exactly $\lambda$ words in $S_d$ s.t the codes has $1$ in position given by $T$
	\end{enumerate}
\end{definition}

\begin{lemma}\label{perfectCdDesign}
Let {\cCodes} be a perfect binary {\code{n}{M}{d}}. Then the set $S_d$ of all codewords of minimum distances $d$ holds a \emph{Steiner system} \steinerSystem{t+1}{d}{n}, where $d = 2t + 1$
\end{lemma}

\begin{proof}
Since {\cCodes} is perfect. So all the spheres $B(c,t)$ are disjoint with one another. So given a binary word $x$ of length $t+1$, it must be included exclusively in one sphere, say $B(c,t)$. Now, since $wt(c) \leq wt(x) + d(c,x) = t + 1 + t = d$, so we deduce that $c \in S_d$. And we have
\begin{align*}
	2wt(x \cap c) &= wt(x) + wt(c) - d(x,c) \geq 2t+2 \\
	\Rightarrow \quad wt(x\cap c) &\geq t + 1
\end{align*}
So $c$ \textbf{covers} $x$. \\
Now if there is another $\tilde{c} \in S_d$ that also \textbf{covers} $x$. Then we must have 
\[
	wt(\tilde{c}) \geq wt(x\cap \tilde{c}) + d(x,\tilde{c}) \geq 2t + 2 = d + 1
\]
 which is impossible. \\
So $S_d$ is a \tParamDesign{t+1}{n}{d}{1}, hence \steinerSystem{t+1}{d}{n} \emph{Steiner System}
\end{proof}

\begin{collary}\label{perfectCodeAd}
Let {\cCodes} be a perfect binary \code{n}{M}{d}-code. Let $A_d$ denotes the number of words of {\cCodes} with weight $d$. Then
\[
	A_d = \frac{\binom{n}{t+1}}{\binom{d}{t+1}}
\]
\end{collary}
\begin{proof}
	Using Collary \ref{blockCount}  and Lemma \ref{perfectCdDesign}
\end{proof}
\begin{example}
For hamming \linearCode{7}{4}{3}. We have
\[
	A_3 = \frac{\binom{8-1}{2}}{\binom{3}{2}} = 7
\]
\end{example}
For the following, we define the $\langle , \rangle$ as the standard inner product of vectors with values in real number. 
For simplicity, we will denote $\langle u,v \rangle$ as $u \cdot v$, if it is clear in context that we are having a scalar product of two vectors. And we know that for two vectors $u,v$ in $\ftwoN{n}$, we have
\begin{equation}\label{distWeightEqn}
	\dist{u}{v} = \wt{u} + \wt{v} - 2\langle u, v\rangle 
\end{equation}
We come to the main theorem in this section.
\begin{theorem}
Let {\cCodes} be a binary {\code{24}{2^{12}}{8}} containing 0. Then {\cCodes} is a up to equivalence the only one doubly even self-dual linear {\linearCode{24}{12}{8}}. 
\end{theorem}
\begin{proof}
Let {\cCodes} be such a code. We first punctuate a poisition 
of each code word and denote the resulted code as {\tildcCodes}.
This code is has minimum distance 7 or 8. So it is a {\code{23}{2^{12}}{7}} or {\code{23}{2^{12}}{8}}. No matter the minimum distance is 7 or 8. The sphere $B(c,3)$ are disjoint from each other for all codes $c \in \tildcCodes$.\\
We show that the minimum distance is 7.  Indeed, we have
\[
	2^{12}(1 + \binom{23}{1} + \binom{23}{2} + \binom{23}{3}) = 2^{23}
\]
In particular this show that the spheres $B(c,3)$ covers the space {$\ftwoN{23}$} and it follows that the minimum distance is 7. (otherwise, there are ``leak points'' at the middle of the distance between these two points and these points are not covered, contradictory to the full packing) and {\tildcCodes} is perfect.\\
Since $0 \in \tildcCodes$, $A_0 = 1$
Now using Collary \ref{perfectCodeAd}, we have for {\tildcCodes} $A_7 = \binom{23}{4}$\\.
To calculate $A_8$, we know that the number of vectors in $\ftwoN{23}$ are given by $\binom{23}{5}$ and they may lie in the spheres of codes with weights 7 or 8. So we have
\[
	A_8 = \frac{\binom{23}{5} - A_7\binom{7}{5}}{\binom{8}{5}} = 506
\]
Using the same idea: we have for the other weight counts:
	\begin{align*}
		A_9 &= \frac{\binom{23}{6} - (A_7 \cdot(7 + 16 \cdot \binom{7}{5}) + A_8\cdot \binom{8}{6})}{\binom{9}{6}} = 0\\
		A_{10} &= \frac{\binom{23}{7} - (A_7\cdot(1 + 7\cdot 16) + A_8\cdot(8 + 15 \cdot \binom{8}{2}))}{\binom{10}{7}} = 0 \\
		A_{11} &= \frac{\binom{23}{8} - (A_7\cdot(16 + 7\cdot \binom{16}{2}) + A_8\cdot(1 + 8 * 15))}{\binom{11}{8}} = 1288\\
		A_{12} &= \frac{\binom{23}{9} - (A_7\cdot\binom{16}{2} + A_8\cdot(15 + 8 \cdot \binom{15}{2}) + A_{11}\cdot\binom{11}{2})}{\binom{12}{9}} = 1288 \\
		A_{13} &= \frac{\binom{23}{10} - (A_7\cdot\binom{16}{3} + A_8 \cdot \binom{15}{2} + A_{11} \cdot(11 + \binom{11}{2} \cdot 12) + A_{12} \cdot \binom{12}{2})}{\binom{13}{10}} = 0\\
		A_{14} &= \frac{\binom{23}{11} - (A_8 \cdot \binom{15}{3} + A_{11} \cdot(1 + 11 \cdot 12) + A_{12} \cdot(12 + \binom{12}{2} \cdot 11))}{\binom{14}{11}} = 0\\
		A_{15} &= \frac{\binom{23}{12} - (A_{11}(12 + 11 \cdot \binom{12}{2}) + A_{12}\cdot(1 + 12\cdot 11))}{\binom{15}{12}} = 506 \\
		A_{16} &= \frac{\binom{23}{13} - (A_{11} \cdot \binom{12}{2} + A_{12}\cdot(11 + 12 \cdot\binom{11}{2}))}{\binom{16}{13}} = 253\\
		A_{17} &= \ldots = 0 \\
		\vdots\\
		A_{23} &= \frac{\binom{23}{20}}{\binom{23}{20}} = 1
	\end{align*}
Now if {\cCodes} contains a word of weight $w$ not divisible by 4. Then by puncturing the {\cCodes} appropriately, we will have word of weight $w$ or $w - 1$ not equal to 0 or -1 (mod 4), which contradicts the weight enumerator coefficients calculated above. So we conclude that {\cCodes} is doubly even and we have thus the weight enumerator coefficients for {\cCodes}
\[
	A_0 = A_{24} = 1, \, A_{8} = A_{16} = 759,\, A_{12} = 2576
\]
In particular all the distances between two codewords are divisible by 4. We deduce from \ref{distWeightEqn} that for all
$u,v \in \cCodes$, $\langle u, v\rangle \in 2\Integer$, and in this case we have $\cCodes \in \cCodesVertical$. But since $\cCodesVertical$ is a linear subspace of dimensions 24 - dim$\langle \cCodes \rangle \leq 12$. It follows that $\langle \cCodes \rangle$ has less than $2^{12}$ elements. Since $\cCodes \subset \langle \cCodes \rangle$ we follow from the cardinality of $\cCodes$ that $\cCodes = \cCodesVertical$ and $\cCodes$ is linear.\\
We prove now that the uniqueness of such codes. \\
Consider a codeword $u$ of weight 12 in {\cCodes} and since $A_{24} = 1$, we have another codeword $\bar{u}$ s.t $u + \bar{u} = 1^{24}$ where here $1^{24}$ stands for the all-one vector in $\ftwoN{24}$. Let us denote $\cCodes_u$ as the subspace when we puncture all the positions of $\cCodes$ given by the support of $u$. Define the function $\pi_{u}$ as the canonical projection.
We have for $c \in \cCodes$, both $c$ and $c + u$ will be projected to the same image. Since $c$ is arbitrarily chosen, it follow that $|\cCodes_u| \leq \frac{2^{12}}{2} = 2^{11}$. Now if there are two different codewords $x,y \in \cCodes$ s.t $\pi_{u}(c) = \pi_{u}(y)$ and $x - y = v \neq u$. Then we know that $u$ covers $v$ and $v \in \cCodes$ so $\wt{v} \geq 8$. It follows that $\dist{u}{v} \leq 4$, contradicting the fact that the minimum distance is 8. So $\cCodes_u$ has exactly $2^{11}$ elements hence a linear subspace of dimension 11.\\
For all the codes $v \in \cCodes_u$, Let $x \in \cCodes$ s.t $\pi_u(x) = v$. It follows that $\wt{v} = x \cdot \bar{u} \in 2\Integer$ since $x, \bar{u} \in \cCodes$ and $\cCodes$ is doubly even. It follows that $\cCodes_u$ is of word length 12, dimension 11, and every codeword is of even length.
So by arranging the columns of the codeword appropriately, we have the generator matrix of \cCodes:
\[
	G := 
\renewcommand\arraystretch{1.3}
\mleft[
\begin{array}{c|c|c|c}
  1^{11} & 1 & 0 & 0^{11} \\
  \hline
  A & (0^{11})^t & (1^{11})^t & I_{11}
\end{array}
\mright]
\]
where $I_{11}$ is the $11 \times 11$ identity matrix and $a^k$ denotes a row vector of length $k$ and every entry is $a$. Since {\cCodes} has minimum distance 8, we deduce that each row
of the matrix $G$ has weight $\geq 8$.
The matrix $A$ has consequently two properties:
\begin{enumerate}
	\item each row has weight $\geq 6$ \label{ApropWt}
	\item every two rows have distance $\geq 6$ \label{ArowDist}
\end{enumerate}
We claim that we have actually in both properties equality.\\
First equality: take for example, $G_{1,-}$ and $G_{2,-}$. We have $\dist{G_{1,-}}{G_{2,-}} = \dist{1^{11}}{A_{1,-}} + 3 \geq 8$. So $\wt{A_{1,-}} \leq 6$. Combining the property \ref{ApropWt}, $\wt{A_{1,-}} = 6$. Similarly, all the row of $A$ must have weight 6.\\
Second equality: since {\cCodes} is doubly even and linear, $4 \,|\, \dist{G_{i,-}}{G_{j,-}}$, for $1 \leq i,j \leq 12$. Using the equality in property \ref{ApropWt}, we have $\dist{A_{i,-}}{A_{j,-}} \in \{ 6, 10 \}$. If $\dist{A_{i,-}}{A_{j,-}} = 10$, then it follows that $\dist{1^{11}}{A_{i,-} - A_{j,-}} = 1$, and hence $\dist{G_{1,-}}{G_{i,-}- G_{j,-}} = 4$, contradicting the fact that {\cCodes} has minimum weight 8. So equality holds.\\
Now, we only need to prove the uniqueness, where we resort to proposition \ref{onlyOneDesign}.\\
The idea is to prove that the $A$ submatrix in $G$ up to column and corresponding row permutation %(i.e the automorphism of {\cCodes} restricted on the ) 
unique. We define $B := J_{11} - A$, where $J$ is the $11 \times 11$ matrix with all entries 1. We have $\wt{B_{i,-}} = 5$, for all $1 \leq i \leq 11$. And $\dist{B_{i,-}}{B_{j,-}} = 6$. Using \ref{distWeightEqn}, it follows $\wt{B_{i,-} \cap B_{j,-}} = 2$. We define a set of elevent elements $P$ and a collection {\dDes} of eleven {\tSubset{5}} of $P$. For clarity, we denote each element of $P$ as point, and each element of {\dDes} as block. Note it is not a coincidence with the definition \ref{designDef}, since we will see {\dDes} is actually a design. We identify $B$ as the incidence matrix of the collection {\dDes}. We have $BJ = 5J$. Since each two blocks of {\dDes} has exactly two points in common and $|P| = |\dDes|$, there is a bijection between $M  := {D| D \subset \dDes, |D| = 2}$ and $N := {T|T \subset P, |T| = 2}$ by defining 
\[
	\phi: M \rightarrow N, \quad \{\,B_1,B_2\,\} \mapsto B_1 \bigcap B_2
\]
So it follows that every {\tSubset{2}} of points is contained in exactly two blocks. So {\dDes} is a {\tParamDesign{2}{11}{5}{2}}. So using proposition \ref{onlyOneDesign}, $B$ is unique up to renumbering of points. Hence it follows that the code {\cCodes} is unique up to equivalence.
\end{proof}

We now want to construct such a code. For that purpose, let us consider the regular icosahedron as a graph. Let $A$ be the adjacency matrix of this graph. By numbering the 12 vertices as shown in the figure \ref{icosahedron}. We define the entry of A as:
\[
	A_{i,j} = 
	\begin{cases}
		1, &\text{if there is an edge between point $i$ and point $j$}\\
		0, &\text{otherwise}
	\end{cases}
\]
It follows that $A has the form$

More over, define $B := J_{12} - A$. We have 
\begin{equation}\label{JMinusA}
	B^tB = I_{12} \,\in \,{\myMatrixRing{\Ftwo}{12}}.\\
\end{equation}
we have for the column vector $B_{-,i} = u + v_i$,where $u = J_{12\,-,i}$ and $v = A_{-,i}$:
\begin{align*}
	(u + v_i)^t(u + v_j) &= u^tu + u^t(v_i + v_j) + v_i^tv_j\\
						 &= 12 + a + b
\end{align*}
Now by observing the figure \ref{icosahedron}, it follows that
$v_i + v_j \in {10,8}, v_i^tv_j \in {2,0} in \Integer$, in particular, all the three terms above are 0 in $\Ftwo$. So 
Then the row of the matrix $G := (I_{12} B)$ generate the the {\linearCode{24}{12}{8}} \cCodes.
\begin{proof}
The generated code is certainly linear, of length 24 and has dimension 12. The only thing to prove is the minimum distance.\\
By observing the matrix of $G$ we find that each row of $G$ has weight 8, in particular divisable by 4. So using \ref{distWeightEqn}, we deduce that {\cCodes} is doubly even.\\
Observing that $G^tG = I_{24}$, we have {\cCodes} is self-dual.\\
For the linear combination of more than three rows of $G$ we have on the first 12 position at least four 1s. And by using \ref{JMinusA} it is impossible on the right half of the 12 position of codeword to be complete 0. (other wise $B$ would have nontrivial kernel which contradicts the fact that $B$ is invertible). In particular every such combination creates codeword with codeword $\geq$ 5, hence at least 8. So the only case to consider choice of two or three vectors. \\
Retrospecting the icosahedron, from which the adjacency matrix $A$ is derived.  We could make a similar interpretation of $B$, that is 
\[
	B_{i,j} = 
	\begin{cases}
		0, &\text{there is an edge between point $i$ and point $j$}\\
		1, &\text{otherwise}
	\end{cases}
\]
And for the sum $c_{i,j} := B_{i,-} + B_{j,-} $ of two row vectors of $B$
\[
	c_{i,j}(k) = 
	\begin{cases}
		1, &\text{there is an edge from point $i$ or $j$ to $k$, but not both}\\
		0, &\text{otherwise}
	\end{cases}
\]
% The symmetry group of icosahedron acts on the pair of points and have three orbits.
The possible results are $\{\,6,10\,\}$. With two 1s on the left 12 positions,  the weights are $\{\,8,12\,\}$
Similarly, for the sum of three vectors $v_i$ ,$v_j$ and $v_k$, there is a 1 at position $l$ if vertice $l$ is connected to exactly two of the three or none of the three.
  the possible results are $\{\,5,9\,\}$. With three 1s on the left 12 poisitions, the weights are $\{\,8,12\,\}$. 
\end{proof}
\begin{figure}
\tdplotsetmaincoords{60}{100}
    \begin{tikzpicture}[tdplot_main_coords,scale=1,line join=round]
    \pgfmathsetmacro\a{2}
    \pgfmathsetmacro{\phi}{\a*(1+sqrt(5))/2}
    \path 
    coordinate [label = right:3](A) at (0,\phi,\a)
    coordinate[label = right:8](B) at (0,\phi,-\a)
    coordinate[label = left:6](C) at (0,-\phi,\a)
    coordinate[label = left:10](D) at (0,-\phi,-\a)
    coordinate[label = above:2](E) at (\a,0,\phi)
    coordinate[label = left:12](F) at (\a,0,-\phi)
    coordinate[label = above:1](G) at (-\a,0,\phi)
    coordinate[label = left:9](H) at (-\a,0,-\phi)
    coordinate[label = below:7](I) at (\phi,\a,0)
    coordinate[label = below:11](J) at (\phi,-\a,0)
    coordinate[label = below:4](K) at (-\phi,\a,0)
    coordinate[label = below:5](L) at (-\phi,-\a,0); 
    \draw[dashed, thick]    (B) -- (H) -- (F) 
    (D) -- (L) -- (H) --cycle 
    (K) -- (L) -- (H) --cycle
    (K) -- (L) -- (G) --cycle
    (C) -- (L) (B)--(K) (A)--(K)
    ;

        \draw[ultra thick]
        (A) -- (I) -- (B) --cycle 
        (F) -- (I) -- (B) --cycle 
        (F) -- (I) -- (J) --cycle
        (F) -- (D) -- (J) --cycle
        (C) -- (D) -- (J) --cycle
        (C) -- (E) -- (J) --cycle
        (I) -- (E) -- (J) --cycle
        (I) -- (E) -- (A) --cycle
        (G) -- (E) -- (A) --cycle
        (G) -- (E) -- (C) --cycle
        ; 
         %\foreach \point/\position in {A/right,B/below,C/above,D/left,E/{above right},F/below,G/above,H/left,I/below,J/right,K/below,L/left}
%{
    %\fill (\point) circle (1.5pt);
    %\node[\position=3pt] at (\point) {$\point$};
%}
\end{tikzpicture}
\caption{icosahedron}
\label{icosahedron}
\end{figure}

Now we want to construct an even unimodular lattice with $a_1 = 0$, i.e., does not contains roots, using the extended golay code {\cCodes}.


\section{The MacWilliams Identity and Gleason's Theorem}\label{sectionMacWill}
We have till now studied the theta function for the lattices and the weight enumerator for the binary codes. And we have learned there is a relationship between the binary codes and the corresponding lattices. It may be interesting to study the relevance between the theta function and the weight enumerator.
We consider first two functions:
% \begin{equation}\label{Afunc}
% 	A(\tau) := \sum_{x \in \Integer}q^{x\cdot x} = 1 + 2q + 2q^4 + 2q^9 + \ldots
% \end{equation}
% We see that $A(\tau)$ is actually the theta function of the lattice $\Gamma = \sqrt{2}\Integer$.
Let $A(\tau)$ denote the theta function of the lattice $\Gamma = \sqrt{2}\Integer$, i.e.
	\begin{align}
	A(\tau) &= \sum_{x \in \Gamma}q^{\frac{1}{2}x\cdot x} \notag\\
			&= \sum_{x \in \Integer}q^{x\cdot x} \notag\\
			&= \sum_{x \in 2\Integer}q^{\frac{1}{4}(x\cdot x)}\label{Afunc}\\
			&= 1 + 2q + 2q^4 + 2q^9 + \ldots \notag
	\end{align}
Consider another function:
\begin{equation}\label{Bfunc}
	B(\tau) := \sum_{x \in 2\Integer + 1}q^{\frac{1}{4}(x\cdot x)}
\end{equation}
Note that $B(\tau)$ is not a theta function of a lattice, since the constant term is 0.
By adding the two functions together, it follows that
\begin{equation}\label{sumABfuncEq}
	A(\tau) + B(\tau) = \sum_{x \in \Integer}q^{\frac{1}{4}(x\cdot x)} = \sum_{x \in \frac{1}{2}\Integer}q^{\frac{1}{2}x \cdot x}
			= \sum_{x\in \Gamma^{*}}q^{\frac{1}{2}(x\cdot x)}
\end{equation}
So the sum of the two functions is the theta function of the dual lattice of $\Gamma$. Using proposition \ref{latticeAndDualThetaProp}, we have:
\begin{equation}\label{AMinusTau}
	A(-\frac{1}{\tau}) = (\frac{\tau}{\imaginary})^{1/2} \, \frac{1}{\sqrt{2}}(A(\tau) + B(\tau)).
\end{equation}
By replacing the $\tau$ into $-\frac{1}{\tau}$, we have from (\ref{AMinusTau})
\begin{align}
	[A(-\frac{1}{\tau}) + B(-\frac{1}{\tau})]\; \frac{1}{\sqrt{2}}\; (\frac{1}{-\tau \imaginary})^{\frac{1}{2}} &= A(\tau) \notag \\
	\Rightarrow \quad B(-\frac{1}{\tau}) &= \frac{\sqrt{2}}{(-\frac{1}{\tau \imaginary})^{\frac{1}{2}}} \; A(\tau) \,- \,(\frac{\tau}{\imaginary})^{1/2} \; \frac{1}{\sqrt{2}}\,(A(\tau) + B(\tau))\notag\\
	&= (\frac{\tau}{\imaginary})^{\frac{1}{2}} \;\frac{1}{\sqrt{2}}\;(A(\tau) - B(\tau)) \label{BMinusTau}
\end{align}
When we consider $A$ and $B$ as two variables, and consider now the ``rotation by $45^{\circ}$ followed by a reflection'',i.e.
\[
	Q := \frac{1}{\sqrt{2}}
	\left [
	\begin{matrix}
		1 & 1\\
		1 & -1
	\end{matrix}
	\right]
\] in the $(A,B)$-plane. For simplicity of notation we identify $A$ as $(A,0)^t$ and $B$ as $(0,B)$, then it follows that:
\begin{align*}
	QA &= \frac{1}{\sqrt{2}}\,(A + B)\\
	QB &= \frac{1}{\sqrt{2}}\,(A - B) 
\end{align*}
After doing elementary algebra, we find that the following homogeneous polynomial in $A$ and $B$ of degree 24 is invariant under such transformation.
\[
	A^4B^4 \,(A^2 - B^2)^4 \, (A^2 + B^2)^4 \; = \; A^4B^4\,(A^4 - B^4)^4
\]
For simplicity, we define $g := {\twoFourDegreeAB}$  and $p := \polyAB{X}{Y} \in \Complex[X,Y]_{hom,24}$
\begin{prop}\label{twoFourABInvarProp}
	{\twoFourDegreeAB} is invariant under the transformation 
	$\tau \mapsto -\frac{1}{\tau}$
\end{prop}
\begin{proof}
We have
\[
	g(-\frac{1}{\tau}) = \polyAB{A(-\frac{1}{\tau})}{B(-\frac{1}{\tau})}
\]
\begin{align*}
 &= \polyAB{((\frac{\tau}{\imaginary})^{\frac{1}{2}}\,Q\,A)}{((\frac{\tau}{\imaginary})^{\frac{1}{2}}\,Q\,B)}\\
 &= (\frac{\tau}{\imaginary})^{12} \, p(QA, QB) \\
 &= \tau^{12}p(A,B) = g(\tau)
\end{align*}
\end{proof}
Since $A$ is invariant under the transformation $\tau \mapsto \tau + 1$ and so does $B^4$. It follows that {\twoFourDegreeAB} is invariant under this transformation.\\
Combining these two results, we deduce that $\twoFourDegreeAB$ is a modular form of weight 12.
\begin{prop}\label{twoFourDegABDelta}
\[
	g := {\twoFourDegreeAB} \;= \; 16 \Delta \;= \; 16\,q\prod_{n = 1}^{\infty}\,(1 - q^n)^{24}
\]
\end{prop}
\begin{proof}
Since the constant term of $B$ is zero, so is the constant term of $g$. It follows that $g$ is a cusp form of weight 12. \\
According to theorem \ref{cuspFormTheo},  $g = k \Delta$ for a $k \in \Complex$. We could determine $k$ by looking at the coefficient of the term of the first order.\\
Since
\[
	B(\tau) = 2 q^{\frac{1}{4}} \; + \; 2q^{\frac{9}{4}} \;+\; 2q^{\frac{25}{4}} \; + \; \ldots 
\]
So, it follows that
\[
	B(\tau)^4 = 16 q \; + \; \mbox{high order terms}
\]
As a result
\[
	\twoFourDegreeAB = 16 q \; + \; \mbox{high order terms} 
\]
So, we deduce that $\twoFourDegreeAB$ = $16 \Delta$
\end{proof}
\begin{prop}\label{ABHammingProp}
Let {\simpleCodes} $\subset \Ftwo^n$ be a binary linear code with Hamming weight enumerator {\weightEnumerator{\simpleCodes}{X}{Y}} . Then
\[
	\thetaFunction{\buildLattice{\simpleCodes}} = \weightEnumerator{\simpleCodes}{A}{B}
\]
\end{prop}
\begin{proof}
Let $c \in \simpleCodes$, and let $\rho : \Integer^n \rightarrow \Ftwo^n$ the canonical reduction modulo 2 projection. Then
\begin{align*}
	\sum_{x \in \frac{1}{\sqrt{2}}\rho^{-1}(c)} q^{\frac{1}{2}(x\cdot x)} &= \sum_{x \in \rho^{-1}(c)} q^{\frac{1}{4}(x\cdot x)}\\
	&= \sum_{(y_1,y_2,\ldots,y_n) \in \Integer^n}q^{\frac{1}{4}\sum_{i = 1}^{n}(c_i + 2y_i)^2}\\
	&= \prod_{i = 1}^{n}\,(\sum_{y\in \Integer}q^{\frac{1}{4}(c_i + 2y)^2})
\end{align*}
Since
\begin{align*}
\sum_{y\in \Integer}q^{\frac{1}{4}(c_i + 2y)^2}) &= 
	\begin{cases}
	\sum_{y\in \Integer}q^{y^2}, &\text{if $c_i$ = 0}\\
	\sum_{y\in \Integer}q^{\frac{1}{4}(1 + 2y)^2}, &\text{if $c_i$ = 1} 
	\end{cases}\\
	&= 
	\begin{cases}
	A, &\text{if $c_i$ = 0} \\
	B, &\text{if $c_i$ = 1}
	\end{cases}
\end{align*}
It follows that
\[
\sum_{x \in \frac{1}{\sqrt{2}}\rho^{-1}(c)} q^{\frac{1}{2}(x\cdot x)} = A^{n - \wt{c}}\,B^{\wt{c}}
\] 
The rest follows by summing over all the codewords in $\simpleCodes$
\end{proof}
\begin{example}\label{e8Example}
For the extended Hamming code $\widetilde{H}$ we have
\[
\weightEnumerator{\widetilde{H}}{X}{Y} = X^8 + 14X^4Y^4 + Y^8
\]
We know that the corresponding lattice $\buildLattice{\widetilde{H}}$ is $E_8$ .
Thus it follows in consideration of proposition \ref{E8isomorphProp}
\[
	\thetaFunction{\buildLattice{\widetilde{H}}} = E_4 = A^8 + 14A^4B^4 + B^8
\]
\end{example}
We are now able to prove the \emph{MacWilliams identity}.
\begin{theorem}\label{MacWilliamsTheorem}
Let {\simpleCodes} $\subset$ $\Ftwo^n$ be a binary \linearCode{n}{k}{d} . Then
\[
	\weightEnumerator{\buildVertical{\simpleCodes}}{X}{Y} = \frac{1}{2^k} \weightEnumerator{\simpleCodes}{X}{Y}
\]
\end{theorem}
\begin{proof}
We observe first that
\begin{align*}
	\weightEnumerator{\buildVertical{\simpleCodes}}{A(-\frac{1}{\tau})}{B(-\frac{1}{\tau})} &= \thetaFunction{\buildLattice{\simpleCodes}}(-\frac{1}{\tau}) \tag{by proposition \ref{ABHammingProp}} \\ %&\overset{\text{Prop.(\ref{ABHammingProp})}}{=}
	&= (\frac{\tau}{\imaginary})^{\frac{n}{2}} \, \frac{1}{2^{\frac{n}{2}} - k} \,\thetaFunction{\buildLattice{\simpleCodes}^*}(\tau) \tag{by proposition \ref{latticeAndDualThetaProp}}\\  %&\overset{\text{Prop.(\ref{latticeAndDualThetaProp})}}{=}
		&= (\frac{\tau}{\imaginary})^{\frac{n}{2}} \, \frac{1}{2^{\frac{n}{2}} - k} \,\thetaFunction{\buildLattice{\buildVertical{\simpleCodes}}}(\tau)\tag{by Lemma \ref{latticeDualcodeVertLemma}}\\ %&\overset{\text{Lemma \ref{latticeDualcodeVertLemma}}}{=}
		&= (\frac{\tau}{\imaginary})^{\frac{n}{2}} \, \frac{1}{2^{\frac{n}{2}} - k} \,\weightEnumerator{\buildVertical{\simpleCodes}}{A(\tau)}{B(\tau)} \tag{by proposition \ref{ABHammingProp}}
\end{align*}
Since the weight enumerator is homogen, and by the transformation formula \ref{AMinusTau}, \ref{BMinusTau}. We have
\[
	\weightEnumerator{\simpleCodes}{A(-\frac{1}{\tau})}{B(-\frac{1}{\tau})} = (\frac{\tau}{\imaginary})^{\frac{n}{2}} \, \frac{1}{2^{\frac{n}{2}}} \, \weightEnumerator{\simpleCodes}{A(\tau)\, + \,B(\tau)}{A(\tau) \,- \,B(\tau)}
\]
Thus
\[
	\weightEnumerator{\buildVertical{\simpleCodes}}{A}{B} = \frac{1}{2^k} \weightEnumerator{\simpleCodes}{A \,+\, B}{A \, - \, B}
\]
We only need to prove that $A$ and $B$ are algebraic independent. But recall that from corollary \ref{E4E6collary}, all the modular forms is generated by the $E_4$ and $E_6$ (as algebra). This implies that the modular form with weight divisable by 4 is generated by $E_4$ and $\Delta$ . But recall example \ref{e8Example} and proposition \ref{twoFourDegABDelta}:
\begin{align*}
	E_4 &= A^8 + 14A^4B^4 + B^8\\
	\Delta &= \frac{1}{16}A^4B^4(A^4 - B^4)^4
\end{align*}
It follows that $A$ and $B$ are algebraically independent and we are done.
\end{proof}
\begin{example}\label{macWilliamsHamingExample}
Let's now reexamine the example \ref{hammingExample}.
We have the weight enumerator for $H$
\[
	\weightEnumerator{H}{X}{Y} = X^7 + 7X^4Y^3 + 7X^3Y^4 + 1
\]
Now we use \emph{MacWilliams Identity} to caculate the weight enumerator for the dual code. We have
\begin{align*}
	\weightEnumerator{\buildVertical{H}}{X}{Y} 
	&= \frac{1}{2^4}\weightEnumerator{H}{X+Y}{X-Y}\\
	&= X^7 + 7X^3Y^4 + Y^7
\end{align*}
which is the exact result we obtained in the previous example
\end{example}
\begin{collary}\label{macWilliamsSelfDualCoroll}
If {\simpleCodes} $\subset \Ftwo^n$ is a self-dual code, then 
\[
	\weightEnumerator{\simpleCodes}{X}{Y} = \weightEnumerator{\simpleCodes}{\frac{X + Y}{\sqrt{2}}}{\frac{X - Y}{\sqrt{2}}}
\]
\end{collary}
\begin{proof}
Since {\simpleCodes} is self-dual. So it is a {\linearCode{n}{\frac{n}{2}}{d}} . Since {\simpleCodes} = {\buildVertical{\simpleCodes}}, it follows by using \emph{MacWilliams Identity}
\begin{align*}
	\weightEnumerator{\simpleCodes}{X}{Y} &= \frac{1}{2^{\frac{n}{2}}} \, \weightEnumerator{\simpleCodes}{X+Y}{X-Y} \\
	&= \weightEnumerator{\simpleCodes}{\frac{X+Y}{\sqrt{2}}}{\frac{X-Y}{\sqrt{2}}} \tag{\weightEnumerator{\simpleCodes}{X}{Y} is homogen of degree n}
\end{align*}
\end{proof}
Another important theorem is the \emph{Gleason Theorem}, which characterizes the \emph{hamming weight enumerator} of a binary doubly self-dual even code.
\begin{theorem}\label{GleasonTheorem}
Let {\simpleCodes} $\subset \Ftwo^n$ be a doubly even self-dual code. Then the \emph{Hamming weight enumerator} {\weightEnumerator{\simpleCodes}{X}{Y}} is a polynomial in
\[
	\phi := \weightEnumerator{\widetilde{H}}{X}{Y} = X^8 + 14X^4Y^4 + Y^8
\]  
and
\[
	\xi := \polyAB{X}{Y}
\]
\end{theorem}
\begin{proof}
We consider the function \weightEnumerator{\simpleCodes}{A}{B}, where $A$ and $B$ are defined in (\ref{Afunc}) and (\ref{Bfunc}). We have seen in proposition \ref{ABHammingProp}, that 
\[
	\thetaFunction{\buildLattice{\simpleCodes}} = \weightEnumerator{\simpleCodes}{A}{B}
\]
Since {\simpleCodes} is doubly even and self-dual, it follows from proposition \ref{codeLatticeProp}, that $\buildLattice{\simpleCodes}$ is even and unimodular. The theorem \ref{evenUniLattice} can be applied. We deduce that {\weightEnumerator{\simpleCodes}{A}{B}} is a modular form of weight $\frac{n}{2}$ and $\frac{n}{2} \equiv 0 \pmod 4$.\\
We have seen in the proof of theorem \ref{MacWilliamsTheorem}, that the algebra of modular form of weight divisable by 4 is generated as algebra by $E_4$ and $\Delta$. Hence, {\weightEnumerator{\simpleCodes}{A}{B}} is generated by $\twoFourDegreeAB$ and $A^8 + 14A^4B^4 + B^8$. Replace $A$ and $B$ as $X$ and $Y$.
% The function is thus expressed as polynomial in $A$ and $B$ which is homogeneous of degree $n$. 
\end{proof}
\begin{example}\label{golayEnumerExample}
As an application of \emph{Gleason Theorem}, we would like to calculate the \emph{hamming weight enumerator} of the extended Golay code {\cCodes}.\\
We have known from section \ref{golaySection}, that {\cCodes} is self-dual and doubly even. So according to \emph{Gleason Theorem}, {\weightEnumerator{\cCodes}{X}{Y}} $\in \Integer[\phi,\rho]$ (since all the coefficients must be integer).\\
Since {\weightEnumerator{\cCodes}{X}{Y}} is homogeneous of degree 24, and the leading coefficient of $X^24$ is 1, it follows that:
\[
	\weightEnumerator{\cCodes}{X}{Y} = \phi^3 + k \cdot \xi, \quad \quad \mbox{for some k $\in \Integer$}
\]
To determine the value of $k$, we can make use of the fact that $A_4 = 0$ in $\cCodes$. So it follows that 
\[
	14 \cdot 3 \,+\, k \, =\, 0 \qquad \; \Rightarrow k \,=\, -42
\]
Thus, the \emph{hamming weight enumerator} of {\cCodes} is 
\[
	\weightEnumerator{\cCodes}{X}{Y} = (X^8 \,+ \, 14X^4 Y^4 \,+\, Y^8) ^3 \; - \; 42 X^4Y^4 \,(X^4 \, - \, Y^4)^4
\]
\end{example}
\section{Outlook}
Now we want to pose the question that given a certain $n = 24m + 8k$, $k = 0,1,2$ how much can we achieve such that given $\simpleCodes \subset \Ftwo^n$ be a doubly even self-dual code, the minimum distance is as large as possible.\\
With the help of theorem \ref{GleasonTheorem}, the weight enumerator of $\simpleCodes$ is given by
\[
	W_{\simpleCodes} = \sum_{j = 0}^{m}\,b_j\,\phi^{3(m-j)+k} \,\xi^j, \qquad b_j \in \Complex
\]
Now we can choose the coefficients of $b_j$, such that $A_{4l} = 0$, for $l = 1,\ldots,m$ (it's possible, by solving $m$ equations in $m$ variables). Then the possibly non-zero coefficient $A_{4m + 4}$ is uniquely determined by $b_j$, whcih we denote by $A_{4m+4}^{*}$ .The code, with such property is called an \emph{extremal code}. An extremal code has thus minimum distance at least $4m + 4$. It can be shown that the minimum distance is $4m + 4$ by showing that $A_{4m + 4}^* \neq 0$ for all $m \geq 1$, details see chapter 19, section 5 of \cite{macwilliams1977theory}\\
But this $m$ cannot be large, since it is shown by Mallows, that for $m$ sufficiently large, $n = 24m$
\[
	A_{4m + 8}^* < 0
\]
One knows that for $n \leq 64$, and some $n > 72$, there exists extremal doubly even self-dual code. It is still unknown whether this is true for $n = 72$\\
There are also similar results for lattices. Let $\Gamma$ be an even unimodular lattice in $\Real^n$, $n = 24 + 8k$ , $k = 0,1,2$. We know from the proof of theorem \ref{MacWilliamsTheorem}, that we can write the theta function of $\Gamma$ as:
\[
	\thetaFunction{\Gamma} = \sum_{j = 0}^m \, b_j E_4^{3(m - j) + k} \, \Delta^j,	\qquad b_j \in \Complex
\]
With the analog argument, we can find coefficients, such that we can cancel all the first $2m + 1$ term up to the constant term, i.e.
\[
	\thetaFunction{\Gamma}(\tau) = 1 \,+\, a_{2m + 2}^* q^{2m + 2}\, +\, \ldots
\]
Such lattice is called an \emph{extremal lattice}. One can also show that $m$ can not be too large. The last thing worth noting is that Frau Nebe has proven in 2010 the existence of an extremal even unimodular lattice in $\Real^{72}$

\newpage
\bibliography{seminarReference}
\bibliographystyle{ieeetr}
\end{document}