\documentclass{article}
\usepackage{amssymb,latexsym,amsmath,amsthm,mathtools,mathrsfs,bm}
\usepackage{tikz,tikz-3dplot} 
\usepackage{fouriernc}
\renewcommand{\baselinestretch}{1.5} 

\newtheorem{theorem}{Theorem}[section]
\newtheorem{lemma}[theorem]{Lemma}
\newtheorem{definition}[theorem]{Definition}
\newtheorem{Algorithmus}{Algorithmus}
\newtheorem{example}[theorem]{Example}
\newtheorem{collary}[theorem]{Collary}
\newtheorem{prop}[theorem]{Proposition}
\numberwithin{equation}{theorem}

\setcounter{MaxMatrixCols}{20}

\newcommand{\GExtend}{\ensuremath{\widetilde{G}}}
\newcommand{\Ftwo}{\ensuremath{\mathbb{F}_2}}
\newcommand{\tCompleteDesign}{\ensuremath{\bm{t\mbox{-}(v,k,\lambda)\; design}}}
\newcommand{\tParamDesign}[4]{\ensuremath{\bm{#1\mbox{-}(#2,#3,#4)\; design}}}
\newcommand{\steinerSystem}[3]{\ensuremath{\bm{S(#1,#2,#3)}}}
\newcommand{\myBlock}{\textbf{block}}
\newcommand{\pluralMyBlock}{\textbf{blocks}}
\newcommand{\xDesign}[1]{\ensuremath{\bm{#1\mbox{-}design}}}
\newcommand{\dDes}{\ensuremath{\mathscr{D}}}
\newcommand{\cCodes}{\ensuremath{\mathscr{C}}}
\newcommand{\tSet}[1]{\ensuremath{\bm{#1\mbox{-}{set}}}}
\newcommand{\tSubset}[1]{\ensuremath{\bm{#1\mbox{-}{subset}}}}
\newcommand{\code}[3]{\ensuremath{\bm{(#1,#2,#3)\mbox{-}code}}}
\newcommand{\linearCode}[3]{\ensuremath{\bm{(#1,#2,#3)\mbox{-}code}}}
\newcommand{\ftwoN}[1]{\ensuremath{\mathbb{F}_2}^{#1}}
\newcommand{\myMatrixRing}[2]{\ensuremath{#1^{#2\times#2}}}
\newcommand{\wloe}{w.l.o.e}
\title{Seminar}
\date{\today}
\author{Panwei Hu}

\begin{document}
\pagenumbering{gobble}
  \maketitle
  \newpage

\tableofcontents
	\newpage
  \newpage
  \pagenumbering{arabic}
\newpage
\section{Introduction}

This report is based on the book \emph{Lattices and Codes},from chapter 2.7 to 2.9
\newpage
\section{The Golay Code and thee Leech Lattice}
In this section, we will discuss a particular type of codes, the golay codes. We will show that there exists a unique doubly even linear code  $\GExtend \subset \Ftwo^{24}$ with $A_4 = 0$ 

We shall first need some definitions.
\begin{definition}\hfill
	\begin{enumerate}
		\item A set with $t$ elements is called a \textbf{t-set}, a subset of a given set with $t$ elements is called a \textbf{t-subset}
		\item Let $V$ be a \textbf{v-set} and each element is called a \textbf{point}. Let $\mathscr{D}$ be a collection of disjoint \textbf{k-subset}. Every such \textbf{k-subset} is called a \textbf{block}. If for any \textbf{t-subset} $T$, there are exactly $\mathbf{\lambda}$ \textbf{k-subsets} $B$ from $S$, s.t, $T \subset B$ then $\mathscr{D}$ is called a \tCompleteDesign . We also sometimes simply call it a $\bm{t}$-design.
		\item A $\bm{t\mbox{-}(v,k,1)}$ design is called a \emph{Steiner System}, denoted as $\bm{S(t,k,v)}$
	\end{enumerate}
\end{definition}
\begin{example}\label{projPlane}
		The projective plane of order 2 is a $\bm{2\mbox{-}(7,3,1)}$ design, hence a $\bm{S(2,3,7)}$ \emph{Steiner System}
\end{example}

Before we begin with the proofs, let us discuss some propeties of the $\bm{t}$-design.
\begin{lemma}\label{propTdesign}
If $\mathscr{D}$ be a $\bm{t\mbox{-}design}$. Then $\mathscr{D}$ is also a $\bm{s\mbox{-}design}$, for $0 \leq s \leq t$
\end{lemma}
\begin{proof}
Let $\mathscr{D}$ be a \tCompleteDesign.
Let $S$ be a \textbf{s-subset}, we count the pairs $(T,B)$ with the property, that $S \subset T \subset B$, with $T$ a \textbf{t-subset} and $B$ a \textbf{block} of $\mathscr{D}$. We denote $\lambda_s$ as the number of \textbf{blocks} in $\mathscr{D}$ which contains $S$. We want to show that $\lambda_s$ is independent of the set $S$, but only depends on the size of the set. 
For each $\binom{v-s}{t-s}$ choices of $T$ that contains $S$, we have $\lambda$ {\pluralMyBlock} $B$ whicht contains $T$. In another way, for each of the $\lambda_s$ $\myBlock$ that contains $S$, there are $\binom{k-s}{t-s}$ choices for $T$. So we have:
\begin{align}
	\lambda_s \, \binom{k-s}{t-s} &= \lambda \, \binom{v-s}{t-s} \notag\\
	\Rightarrow  \quad \lambda_s &= \frac{\lambda \, \binom{v-s}{t-s}}{\binom{k-s}{t-s}} \label{sDesign}
\end{align}
In particular $\lambda_s$ is only dependent on the size of the subsets. So that $\mathscr{D}$ is a \xDesign{s}
\end{proof}

\begin{collary}\label{blockCount}
Let {\dDes} be a \tCompleteDesign.
Then {\dDes} has $b$ blocks, where
\[
	b = \frac{\lambda \, \binom{v}{t}}{\binom{k}{t}}
\]
\end{collary}

Let us denote $\lambda_0$ as $b$ (total number of blocks) and
$\lambda_1$ as $r$ (total number of blocks containg a given point)\\
Before proving the next lemma, we need a another fact from linear algebra.
\begin{lemma}\label{invertible}
Let $I$ be the identiy matrix in {$M := \myMatrixRing{\mathbb{R}}{n}$} Let $J \in M$ be the matrix with all entries equal to 1, and $k \in \mathbb{R}^*$ then $A := I + kJ$ is invertible
\end{lemma}
\begin{proof}
Since $(kJ)^2 = k^2 n J$ we have for the minimal polynomial of A :
\[
	\mu_A(x) := (x - 1)(x - 1 - (kn))
\]
In particular $\mu_A(0) \neq 0$, so A is invertible.
\end{proof}
\begin{lemma}\label{lambdaCommonPoints}
In a {\tParamDesign{2}{v}{k}{\lambda}} with $b = v$, $k = r$, any two blocks have exactly $\lambda$ common points.
\end{lemma}
\begin{proof}
Let $I,J$ have the same meaning as in the previous lemma for $n = v$.\\
We define the characteristic vectors of the blocks as the rows of a $b \times v$ matrix $M$. That is
\[
	M_{i,j} = 
	\begin{cases}
		1, &\text{if block $i$ contains the point $j$}\\
		0, &\text{otherwise}
	\end{cases}
\]
Denote the $i$-th block  as $B_i$, and the $j$-th point as $v_j$, $1 \leq i \leq b, 1 \leq j \leq v$
Now we have the two equations:
\begin{align}
	\sum_{j = 1}^{v} M_{i,j} &= \left|\{v\,|\,v \in B_i\} \right| \label{rowSum}\\
	\sum_{i = 1}^{b} M_{i,j} &= \left|\{i\,|\, v_j \in B_i\} \right| \label{colSum}
\end{align}
Based on these two equations, we can translate the condition that any block contains $k$ points and that any point lies in $r$ blocks in view of $k = r$ into following:
\[
	MJ = kJ = rJ = JM
\]
Meanwhile, if we denote $M = (w_1|w_2|\ldots|w_v)$ we have 
\begin{align}
	w_j^t w_i &= \,\left|\{i\,| \, v_j \in B_i \wedge v_i \in B_i\} \right|	\label{colVprod}\\
			&\overset{\wedge}{=} \,\text{``Number of blocks that contain both $v_i$ and $v_j$''}\notag
\end{align}
If we denote 
\[
 M = 
	\begin{pmatrix}
		u_1\\
		u_2\\
		\vdots\\
		u_b
	\end{pmatrix}
\]
Then we have 
\begin{align}
	u_i u_j^t &= \left|\{v_i\,|\, v_i \in B_i \wedge v_i \in B_j\} \right | \label{rowVprod}\\ 
	&\overset{\wedge}{=} \, \text{``Number of points that both block $B_i$ and $B_j$ contains''} \notag
\end{align}
The condition that we have a {\tParamDesign{2}{v}{k}{\lambda}}, i.e. any pair of points lies in $\lambda$ blocks can be expressed based on \eqref{colVprod} as follows:
\[
	M^tM = (r - \lambda)I + \lambda J
\] 
Since $M$ commutes with $J$ and $M^t = ((r - \lambda) I + \lambda J)M^{-1}$ (Lemma \ref{invertible}), $M$ also commutes with $M^t$ and thus:
\[
	MM^t = (r - \lambda)I + \lambda J
\]
Based on \eqref{rowVprod} we see that any two blocks have exactly $\lambda$ points in common.
\end{proof}

\begin{prop}
There is only one \tParamDesign{2}{11}{5}{2}
\end{prop}
\begin{proof}
We have in this case according to \eqref{sDesign}:
\begin{alignat}{3}
	b &= \frac{2\,\binom{11}{2}}{\binom{5}{2}} & = 11 &= v \notag\\
	r &= \frac{2 \, \binom{11-1}{2 - 1}}{\binom{5-1}{2 - 1}} & = 5 & \notag\phantom{v} 
\end{alignat}
Therefore, we can apply Lemma \ref{lambdaCommonPoints} and it follows that any two blocks have 2 common points.\\
{\wloe} we choose the characteristic vector of the first block as 
$(1 1 1 1 1 0 0 0 0 0 0)$. The remaining blocks coorespond to the {\tSubset{2}}\textbf{s} of the first five points.\\
we can choose the second as $(1 1 0 0 0 1 1 1 0 0 0)$ and the following 4. For the $6$-th row, there are two choices, namely
$(0 1 1 0 0 0 1 0 0 1 1)$ or $(0 1 1 0 0 0 0 1 1 0 1)$. And the rest are uniquely determined. \\
We have for the first choice the matrix:
\[
	\begin{pmatrix}
 1 & 1 & 1 & 1 & 1 & 0 & 0 & 0 & 0 & 0 & 0 \\
 1 & 1 & 0 & 0 & 0 & 1 & 1 & 1 & 0 & 0 & 0 \\
 1 & 0 & 1 & 0 & 0 & 1 & 0 & 0 & 1 & 1 & 0 \\
 1 & 0 & 0 & 1 & 0 & 0 & 1 & 0 & 1 & 0 & 1 \\
 1 & 0 & 0 & 0 & 1 & 0 & 0 & 1 & 0 & 1 & 1 \\
 0 & 1 & 1 & 0 & 0 & 0 & 1 & 0 & 0 & 1 & 1 \\
 0 & 1 & 0 & 1 & 0 & 0 & 0 & 1 & 1 & 1 & 0 \\
 0 & 1 & 0 & 0 & 1 & 1 & 0 & 0 & 1 & 0 & 1 \\
 0 & 0 & 1 & 1 & 0 & 1 & 0 & 1 & 0 & 0 & 1 \\
 0 & 0 & 1 & 0 & 1 & 0 & 1 & 1 & 1 & 0 & 0 
	\end{pmatrix}
	% \begin{tabular}{c|c|c|c|c|c|c|c|c|c|c}
 % 1 & 1 & 1 & 1 & 1 & 0 & 0 & 0 & 0 & 0 & 0 \\
 % 1 & 1 & 0 & 0 & 0 & 1 & 1 & 1 & 0 & 0 & 0
	% \end{tabular}
\]
\end{proof}
Now let us come back to codes and discuss the relationship between design and codes.\\
Let {\cCodes} be a binary code of length $n$.\\
First we define $S_d := \{c\,|\,wt(c) = d, c \in \cCodes\}$ as the set of codes of \cCodes, s.t the weight of the codes equal $d$

\begin{definition}
	Let $x$, $y$ two binary words of length $n$
	\begin{enumerate}
		\item The sphere around $x$ with radius $r$ is denoted as $B(x,r)$ with respect to the usual hamming distance.
		\item The \textbf{support} of $x$ is the set of positions in which $x$ has no zero entries.
		\item We say that $x$ \textbf{covers} $y$, if the support $y$ is a subset of $x$
		\item We say that $S_d$ \textbf{holds a } {\tParamDesign{t}{n}{d}{\lambda}}, if the supports of the codewords of $S_d$ form the {\pluralMyBlock} of a {\tParamDesign{t}{n}{d}{\lambda}}. In other words, for any {\tSet{t}} $T$, there are exactly $\lambda$ words in $S_d$ s.t the codes has $1$ in position given by $T$
	\end{enumerate}
\end{definition}

\begin{lemma}\label{perfectCdDesign}
Let {\cCodes} be a perfect binary {\code{n}{M}{d}}. Then the set $S_d$ of all codewords of minimum distances $d$ holds a \emph{Steiner system} \steinerSystem{t+1}{d}{n}, where $d = 2t + 1$
\end{lemma}

\begin{proof}
Since {\cCodes} is perfect. So all the spheres $B(c,t)$ are disjoint with one another. So given a binary word $x$ of length $t+1$, it must be included exclusively in one sphere, say $B(c,t)$. Now, since $wt(c) \leq wt(x) + d(c,x) = t + 1 + t = d$, so we deduce that $c \in S_d$. And we have
\begin{align*}
	2wt(x \cap c) &= wt(x) + wt(c) - d(x,c) \geq 2t+2 \\
	\Rightarrow \quad wt(x\cap c) &\geq t + 1
\end{align*}
So $c$ \textbf{covers} $x$. \\
Now if there is another $\tilde{c} \in S_d$ that also \textbf{covers} $x$. Then we must have 
\[
	wt(\tilde{c}) \geq wt(x\cap \tilde{c}) + d(x,\tilde{c}) \geq 2t + 2 = d + 1
\]
 which is impossible. \\
So $S_d$ is a \tParamDesign{t+1}{n}{d}{1}, hence \steinerSystem{t+1}{d}{n} \emph{Steiner System}
\end{proof}

\begin{collary}\label{perfectCodeAd}
Let {\cCodes} be a perfect binary \code{n}{M}{d}-code. Let $A_d$ denotes the number of words of {\cCodes} with weight $d$. Then
\[
	A_d = \frac{\binom{n}{t+1}}{\binom{d}{t+1}}
\]
\end{collary}
\begin{proof}
	Using Collary \ref{blockCount}  and Lemma \ref{perfectCdDesign}
\end{proof}
\begin{example}
For hamming \linearCode{7}{4}{3}. We have
\[
	A_3 = \frac{\binom{8-1}{2}}{\binom{3}{2}} = 7
\]
\end{example}

\tdplotsetmaincoords{60}{100}
    \begin{tikzpicture}[tdplot_main_coords,scale=1,line join=round]
    \pgfmathsetmacro\a{2}
    \pgfmathsetmacro{\phi}{\a*(1+sqrt(5))/2}
    \path 
    coordinate [label = right:3](A) at (0,\phi,\a)
    coordinate[label = right:8](B) at (0,\phi,-\a)
    coordinate[label = left:6](C) at (0,-\phi,\a)
    coordinate[label = left:10](D) at (0,-\phi,-\a)
    coordinate[label = above:2](E) at (\a,0,\phi)
    coordinate[label = left:12](F) at (\a,0,-\phi)
    coordinate[label = above:1](G) at (-\a,0,\phi)
    coordinate[label = left:9](H) at (-\a,0,-\phi)
    coordinate[label = below:7](I) at (\phi,\a,0)
    coordinate[label = below:11](J) at (\phi,-\a,0)
    coordinate[label = below:4](K) at (-\phi,\a,0)
    coordinate[label = below:5](L) at (-\phi,-\a,0); 
    \draw[dashed, thick]    (B) -- (H) -- (F) 
    (D) -- (L) -- (H) --cycle 
    (K) -- (L) -- (H) --cycle
    (K) -- (L) -- (G) --cycle
    (C) -- (L) (B)--(K) (A)--(K)
    ;

        \draw[ultra thick]
        (A) -- (I) -- (B) --cycle 
        (F) -- (I) -- (B) --cycle 
        (F) -- (I) -- (J) --cycle
        (F) -- (D) -- (J) --cycle
        (C) -- (D) -- (J) --cycle
        (C) -- (E) -- (J) --cycle
        (I) -- (E) -- (J) --cycle
        (I) -- (E) -- (A) --cycle
        (G) -- (E) -- (A) --cycle
        (G) -- (E) -- (C) --cycle
        ; 
         %\foreach \point/\position in {A/right,B/below,C/above,D/left,E/{above right},F/below,G/above,H/left,I/below,J/right,K/below,L/left}
%{
    %\fill (\point) circle (1.5pt);
    %\node[\position=3pt] at (\point) {$\point$};
%}

\end{tikzpicture}





\end{document}