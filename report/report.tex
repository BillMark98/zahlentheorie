\documentclass{article}
\usepackage{amssymb,latexsym,amsmath,amsthm}

\newtheorem{theorem}{Satz}
\newtheorem{lemma}{Lemma}
\newtheorem{definition}{Definition}


%some defined variable
\newcommand{\myparaDent}{12pt}
\newcommand{\Prime}{\ensuremath{\mathbb{P}}}
\newcommand{\Nature}{\ensuremath{\mathbb{N}}}
\newcommand{\myMersen}{\emph{Mersennesche Zahl}}

\title{My first document}
\date{2019-6-7}
\author{Panwei Hu}

% set the contents of tables depth 
% here 2 ----  subsection
\setcounter{tocdepth}{2}
\begin{document}


  \pagenumbering{gobble}
  \maketitle
  \newpage
 \tableofcontents
	\newpage
  \newpage
  \pagenumbering{arabic}

  Hello World!
  \begin{equation}
  \label{quadric}
  	f(x) = x^2
  \end{equation}
Please see the equation \ref{quadric} for
definition of the function $f$
\newpage
\section{Einleitung}

\section{Mersennsche Primzahl}
Bevor wir zur Definition von \emph{Mersennesche Primzahl} kommen, wird es
hilfreich, wenn wir das folgende Lemma aus dem Skript~\cite{script} kennen.
\begin{lemma}\label{lem1}
	Sei $a,m \in \mathbb{N},m > 1$, so dass $a^m - 1$ eine Primzahl ist. Dann ist
	$a = 2$ und $m$ eine Primzahl.
\end{lemma}
\begin{proof}[Beweis]
Siehe \cite{script}
\end{proof}
Mithilfe dieses Lemmas f\"{u}hren wir die Definition von \emph{Mersennesche Primzahl} ein, die wir auch aus dem Skript~\cite{script} kennen:
\begin{definition}
Die Zahlen
\[
	M_{q} = 2^{q} - 1,
\]
$q \in \mathbb{P}$, hei{\ss}en \emph{Mersennesche Zahl} $M_{q}$ bzw. \emph{Mersennesche Primzahlen} falls sie zu $\mathbb{P}$ geh\"{o}rt.
\end{definition}
Wir sehen leicht, dass $M_{q}$ eine Primzahl ist, wenn $q$ eine Primzahl ist.
Somit gibt es eine Methode f\"{u}r eine gegebene Zahl $n$ zu testen, ob $n$ eine Primzahl ist, n\"{a}mlich, wir berechnen die $n-te$ \emph{Mersennesche Zahl} $M_{n}$. Wenn wir zeigen k\"{o}nnen, dass $M_{n}$ keine Primzahl ist, dann ist $n$ nach Lemma~\ref{lem1} auch nicht prim.\\[\myparaDent]
% Wichtig ist an der Stelle zu merken, dass die Umkehrung nicht richtig ist. Ein Gegenbeispiel ist $q = 11$ mit $M_{q} = 2^{11} -1 = 23 \cdot 89$\\[myparaDent]
Unsere Aufgabe ist jetzt, ein Algorithmus zu suchen, so dass f\"{u}r eine beliebige $n \in \Nature$, schnell eine Aussage \"{u}ber die $n-te$ \emph{Mersennesche Zahl} $M_{n}$ zu bekommen, ob die prim oder nicht prim ist. Dieses Algorithmus

\section{Der Lukas Lehmer Test}

\section{Beweis des Test}

\newpage
\begin{appendix}

\end{appendix}

Please see the \cite{zahlentheorie}
\newpage
\bibliography{reference}
\bibliographystyle{ieeetr}
\end{document}