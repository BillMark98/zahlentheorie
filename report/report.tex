\documentclass{article}
\usepackage{amssymb,latexsym,amsmath,amsthm,mathtools}

\renewcommand{\baselinestretch}{1.5} 

\newtheorem{theorem}{Satz}
\newtheorem{lemma}{Lemma}
\newtheorem{definition}{Definition}
\newtheorem{Algorithmus}{Algorithmus}
\newtheorem{example}{Beispiel}

%some defined variable
\newcommand{\myparaDent}{12pt}
\newcommand{\Prime}{\ensuremath{\mathbb{P}}}
\newcommand{\Nature}{\ensuremath{\mathbb{N}}}
\newcommand{\Rational}{\ensuremath{\mathbb{Q}}}
\newcommand{\Integer}{\ensuremath{\mathbb{Z}}}
\newcommand{\myMersen}{\emph{Mersennesche Zahl}}
\newcommand{\Qadjd}{\ensuremath{\Rational[\sqrt{d}\,]}}
\newcommand{\ZadjdOhnez}{\ensuremath{\Zadjd \setminus\! \{ 0 \}}}

\newcommand{\Zadjd}{\ensuremath{\Integer[\sqrt{d}\,]}}
\newcommand{\ZAdjdModq}{\ensuremath{\Integer[\sqrt{d}\,]/\langle q \rangle}}
\newcommand{\Zmodq}{\ensuremath{\Integer/q\Integer}}
\newcommand{\ZadjdModqOhnez}{\ensuremath{\ZAdjdModq \setminus\! \{ 0 \}}}
\newcommand{\QRmodq}{\ensuremath{\mbox{QR}\pmod{q}}}
\newcommand{\QNmodq}{\ensuremath{\mbox{QN} \! \pmod{q}}}
\newcommand{\Zmodn}{\ensuremath{\Integer/n\Integer}}
\newcommand{\ZadThreeModn}{\ensuremath{\Integer[\sqrt{3}\,]/\langle n \rangle}}

\makeatletter
\newcommand{\rmnum}[1]{\romannumeral #1}
\newcommand{\Rmnum}[1]{\expandafter\@slowromancap\romannumeral #1@}
\makeatother
\title{Der Lucas-Lehmer-Test}
\date{\today}
\author{Panwei Hu \quad 374218\\Kai Tangelder \quad 358033}

% set the contents of tables depth 
% here 2 ----  subsection
\setcounter{tocdepth}{2}

\renewcommand{\contentsname}{Inhalte}
\renewcommand{\refname}{Literatur}
\begin{document}


  \pagenumbering{gobble}
  \maketitle
  \newpage
 \tableofcontents
	\newpage
  \newpage
  \pagenumbering{arabic}
\newpage
\section{Einleitung}
Der \emph{Lucas-Lehmer-Test} ist ein Algorithmus, mit dem man testen kann, ob eine gegebene
\myMersen{} eine Primzahl ist. Der Test wurde erstmal von Édouard Lucas empfunden im Jahr 1856 und wurde sp\"{a}ter vom Lucas im Jahr 1878 und Derrick Henry Lehmer in den 1930s verbessert. In dieser Ausarbeitung stellen wir den Algorithmus vor und geben den Beweis f\"{u}r die Richtigkeit des Algorithmus dazu.
\section{Mersennsche Primzahlen}
Bevor wir zur Definition von \emph{Mersennesche Primzahlen} kommen, ist es
hilfreich, wenn wir uns an das folgende Lemma aus dem Skript~\cite{script} erinnern.
\begin{lemma}\label{lem1}
	Sei $a,m \in \mathbb{N},m > 1$, so dass $a^m - 1$ eine Primzahl ist. Dann ist
	$a = 2$ und $m$ eine Primzahl.
\end{lemma}
\begin{proof}[Beweis]
Siehe \cite{script}
\end{proof}
Mithilfe dieses Lemmas f\"{u}hren wir die Definition von \emph{Mersennesche Primzahlen} ein, die wir auch aus dem Skript~\cite{script} kennen:
\begin{definition}
Die Zahlen
\[
	M_{q} = 2^{q} - 1,
\]
$q \in \mathbb{P}$, hei{\ss}en \emph{Mersennesche Zahl} $M_{q}$ bzw. \emph{Mersennesche Primzahlen} falls sie zu $\mathbb{P}$ geh\"{o}ren.
\end{definition}
Wir sehen leicht, dass $M_{q}$ eine Primzahl ist, wenn $q$ eine Primzahl ist.
Somit gibt es eine Methode f\"{u}r eine gegebene Zahl $n$ zu testen, ob $n$ eine Primzahl ist, n\"{a}mlich, wir berechnen die $n-te$ \emph{Mersennesche Zahl} $M_{n}$. Wenn wir zeigen k\"{o}nnen, dass $M_{n}$ keine Primzahl ist, dann ist $n$ nach Lemma~\ref{lem1} auch nicht prim.\\[\myparaDent]
% Wichtig ist an der Stelle zu merken, dass die Umkehrung nicht richtig ist. Ein Gegenbeispiel ist $q = 11$ mit $M_{q} = 2^{11} -1 = 23 \cdot 89$\\[myparaDent]
Unsere Aufgabe ist jetzt, einen Algorithmus zu suchen, so dass f\"{u}r ein beliebiges $n \in \Nature$, schnell eine Aussage \"{u}ber die $n-te$ \emph{Mersennesche Zahl} $M_{n}$ zu bekommen, ob diese prim oder nicht prim ist. Dieser Algorithmus ist der \emph{Lucas-Lehmer-Test}. 

\section{Der Lucas-Lehmer-Test}
\begin{theorem}\label{lucas}
	Sei $p > 2$ eine Primzahl und die Folge $(S_n)_{n \in \Nature}$ rekursiv durch
	\[
		S_1 := 4,\, \mbox{und} \; S_n = S_{n-1}^2 - 2,\quad \mbox{f\"{u}r} \; n \in \Nature \; \mbox{mit} \;n \geq 2
	\]
	Dann ist $M_p := 2^p - 1$ eine Primzahl genau dann wenn $M_p$ ein Teiler von $S_{p-1}$ ist
\end{theorem}

\begin{example}
	Wir untersuchen $M_7 = 2^7 - 1 = 127$ mithilfe \emph{Lucas-Lehmer-Test},dazu:
	\begin{align*}
			S_1 &= 4 & \pmod{127}\\
			S_2 &= S_1^2 - 2 = 14 & \pmod{127}\\
			S_3 &= S_2^2 - 2 = 64 & \pmod{127}\\
			S_4 &= S_3^2 - 2 = 42 & \pmod{127}\\
			S_5 &= S_4^2 - 2 = 111 & \pmod{127}\\
			S_6 &= S_5^2 - 2 = 0 & \pmod{127}\\
	\end{align*}
	Also ist nach \emph{Lucas-Lehmer-Test} $M_7$ eine Primzahl, was unserer Erkenntnis entspricht.
\end{example}
\section{Beweis des Tests}
Wir f\"{u}hren den Beweis in zwei Teile durch.\\
\subsection{Vor\"{u}berlegung}
F\"{u}r den Beweis benutzen wir zwei Hilfsvariablen,
\[
	\psi = 2 + \sqrt{3}\quad \mbox{und die dazu konjugierte}\quad \bar{\psi} = 2- \sqrt{3}
\]
Es gilt $S_1 = 4 = \psi + \bar{\psi}$ und 
\begin{equation}\label{psiProp}
	\psi \cdot \bar{\psi} = 1
\end{equation}
und man zeigt durch Induktion leicht, dass 
\begin{equation}\label{Skpsi}
S_k \nobreak = \nobreak \psi^{2^{k-1}} \!+ \, \bar{\psi}^{2^{k-1}}
\end{equation} ist. F\"{u}r den Induktionsschritt:
\begin{align*}
	S_{k+1} &\overset{\text{def}}{=} S_{k}^2 - 2\\
			&\overset{\text{IV}}{=} (\psi^{2^{k-1}} + \bar{\psi}^{2^{k-1}})^2 - 2\\
			&= \psi^{2^{k}} + \bar{\psi}^{2^{k}} + 2\cdot \psi^{2^{k-1}} \cdot \bar{\psi}^{2^{k-1}} - 2\\
			&\overset{\mathclap{\text{(\ref{psiProp})}}}{=} \psi^{2^{k}} + \bar{\psi}^{2^{k}} + 2 - 2\\
			&= \psi^{2^{k}} + \bar{\psi}^{2^{k}}
\end{align*}
F\"{u}r den Beweis ben\"{o}tigen wir noch ein Lemma
\begin{lemma}\label{restRing}
	Sei $d$ eine quadratfreie ganze Zahl in \Integer, dann gilt 
	f\"{u}r die Menge \Integer$[\sqrt{d}\,]:=\{\, a + b\sqrt{d} \mid a,b\in \Integer \,\}$
	\begin{enumerate}
		\item \Integer$[\sqrt{d}\,]$ ist ein Ring mit der \"{u}blichen Addition und Multiplikation
		\item Sei $q \in \Nature$, dann ist der Restklassenring $\ZAdjdModq$ ein Ring mit der \"{u}blichen Addition und Multiplikation modulo $q$. Falls $q \in \Prime$  und $d$ kein quadratisches Rest in $\Zmodq$ ist, dann ist $\ZAdjdModq$ ein K\"{o}rper.
		\item Sei $q \in \Prime$, dann gilt f\"{u}r alle Elemente $a + b\sqrt{d} \in \ZAdjdModq$, mit $a,b\in \Integer$:
		\[
			(a + b\sqrt{d})^q  = a^q + (b\sqrt{d})^q
		\]
	\end{enumerate}
\end{lemma}
\begin{proof}[Beweis]
	\begin{enumerate}
		\item Man braucht nur die Abgeschlossenheit der Addition und Multiplikation zu zeigen, da alle andere Rechenregeln aus der Ringstruktur der ganzen Zahlen $\Integer$ folgen. \\ Wir k\"{o}nnen es aber auch anders machen, indem wir uns an die Algebraische Zahlentheorie erinnern. Diese besagt, dass $\Zadjd$ der Teilring der ganzen Zahlen (sogar gleich, falls $d \equiv {2,3} \pmod{4}$ ist) von dem K\"{o}rper $\Qadjd$ ist. Details, siehe z.B \cite{zahlentheorie}. Damit folgt
		die Behauptung
		\item 
		Dass $\ZAdjdModq$ ein Ring ist, folgt daraus, dass $\langle q \rangle$ ein Ideal in $\ZAdjdModq$ ist.\\
		Wir zeigen, dass jedes Element aus $\ZadjdModqOhnez$ eine Inverse hat.\\
		Sei $a + b\sqrt{d}$ mit $a,b\in \Integer$, $o.B.d.A$ nehmen wir an, dass $b \neq 0$ ist. Wir suchen ein $m + n\sqrt{d}$, mit $m,n \in \Integer$, so dass $(a + b\sqrt{d})(m + n\sqrt{d}) = 1$ Durch Ausmultplizieren bekommen wir zwei Gleichungen:
		\begin{align}
		am + dbn &= 1 \label{eqz}\\
		an + bm &= 0  \label{eqz2}
		\end{align} 
		Dies ist ein Lineares Gleichungssystem in $\Zmodq$, die Determinate der Matrix $A := $
		$\begin{pmatrix}
		a & db\\
		b & a
		\end{pmatrix}$
		ist $a^2 - db^2 =: t$. $t$ ist invertierbar in $\Zmodq$, genau dann, wenn $t\neq 0$ ist und 
		\begin{align*}
			t \neq 0 &\Leftrightarrow a^2 \neq db^2 \\
				&\Leftrightarrow (a/b)^2 \neq d\\
				&\Leftrightarrow d \; \QNmodq
		\end{align*}
		d.h, die Gleichungen (\ref{eqz}) (\ref{eqz2}) sind l\"{o}sbar, genau dann wenn
		$d$ kein quadratisches Rest in $\Zmodq$ ist. Es folgt also, dass
		alle Elemente aus $\ZadjdModqOhnez$ eine Inverse haben.
		\item
		Das folgt aus dem Skript \cite{script} Seite 31, Lemma(3.15)
	\end{enumerate}
\end{proof}
Zuerst zeigen wir die R\"{u}ckrichtung.
\subsection{Beweis der R\"{u}ckrichtung}
\begin{proof}[``$\Leftarrow$"]
Sei $n := M_p$ ein Teiler von $S_{p-1}$, und sei $q$ ein Teiler von $M_p$.
Aus der Definition von $M_p$ wissen wir, dass $q\neq2$ ist. Au\ss erdem nehmen wir an, dass $q^2 \leq n$ ist. (denn ansonsten, ist $\frac{n}{q}$ auch ein Teiler von n, was aber $(\frac{n}{q})^2 \leq n$ folgt).Wir rechnen im Ring $E  := \Integer[\sqrt{3}\,]/\langle q \rangle$ (der Ring ist nicht unbedingt ein K\"{o}rper)\\
Es gilt %&\phantom{\Rightarrow a} 
\begin{alignat*}{4}
	&\phantom{\Rightarrow a} & S_{p-1} \quad &&\equiv{\phantom{-}0} &\pmod{n} \\
	&\Rightarrow &S_{p-1} 		\quad   &&\equiv{\phantom{-}0} &\pmod{q}\\
	&\overset{\text{(\ref{Skpsi})}}{\Leftrightarrow} \quad \psi^{2^{p-2}} &+ \phantom{1}\bar{\psi}^{2^{p-2}} \quad &&\equiv{\phantom{-}0} &\pmod{q}\\
	&\overset{\text{(\ref{psiProp})}}{\Leftrightarrow} \quad \psi^{2^{p-1}} &+ \,1 \phantom{\bar{\psi}^{2^{p-2}}} \quad &&\equiv{\phantom{-}0}  &\pmod{q}\\
	&\Leftrightarrow &\psi^{2^{p-1}} \quad &&\equiv{-1} &\pmod{q}\\
	&\Rightarrow &\psi^{2^{p\phantom{-1}}} \quad && \equiv{\phantom{-}1} &\pmod{q}
\end{alignat*}
insbesonere folgt, dass ord($\psi$) $ = 2^p$ ist. Wir wissen, dass die Einheit von $E$ weniger als $q^2 - 1$ Elemente besitzt. Dann folgt aus dem Satz von Lagrange unter Ber\"{u}cksichtigung,dass $2^{p} = n + 1$:
\[
	n + 1 = 2^{p} \,\leq q^2-1 \,\leq n - 1
\]
Also ein Widerspruch. Insbesondere besitzt $n$ keinen Teiler, also ist $n = M_p$ eine Primzahl
\end{proof}
\subsection{Beweis der Hinrichtung}
\begin{proof}[``$\Rightarrow$'']
Sei $n := M_p = 2^{p} - 1$ jetzt eine Primzahl. Dann wissen wir aus Lemma(\ref{lem1}), dass $p \in \Prime.$ \\Es gilt
\begin{equation}\label{nPequ}
 p = \frac{n + 1}{2}
\end{equation}
F\"{u}r $p \leq 2$ rechnen wir einfach nach und pr\"{u}fen die Richtigkeit. Jetzt nehmen wir an, dass $p \geq 3 $ und $ p \in \Prime$ ist. Dann folgt zuerst:
\[
	n = 2^p - 1 \equiv{-1} \pmod{8}
\]
also gilt $(\frac{2}{n}) = 1$ nach \cite{script} \Rmnum{2} Lemma(3.13). Insbesondere
folgt aus dem Satz von Euler:
\begin{equation}\label{twoPo}
	2^{\frac{n-1}{2}} \equiv{1} \pmod{n}
\end{equation}
Wir zeigen jetzt, dass 3 $\mbox{QN}\pmod{n}$ ist.
Aus \cite{script} Seite 84, wissen wir dass
\[
	\left(\frac{3}{n} \right) = 
	\begin{cases}
	\,1, &\text{falls $p \equiv{\pm1} \pmod{12}$,}\\
	\,-1, &\text{falls $n \equiv{\pm5} \pmod{12}$.}
	\end{cases}
\]
Es gilt
\[
	2^4 \equiv{2^2} \pmod{12}
\]
Dann folgt, da $p \geq 3, p \in \Prime$ :
\begin{alignat*}{4}
	&\phantom{\Rightarrow a} 2^{p}     		&&\equiv{2^3}      &&		   &&\pmod{12}\\
	&\Leftrightarrow 2^{p} - 1 	\quad 		&&\equiv{2^3 - 1}  &&\equiv{7} &&\pmod{12}\\
	&\Rightarrow \left( \frac{3}{n} \right) &&= -1			   &&			   &
\end{alignat*}
also insbesondere mithilfe Satzes von Euler:
\begin{equation}\label{threePot}
	3^{\frac{n-1}{2}} \equiv{-1} \pmod{n}
\end{equation}
Da 2 ein quadratisches Rest in $\Zmodn$ ist, ist 2 auch ein quadratisches Rest in
$\ZadThreeModn$, also $\exists x \in \ZadThreeModn$, mit 
\begin{equation}\label{xQuTwo}
	x^2 = 2
\end{equation}
Aus dem Beweis von der R\"{u}ckrichtung wissen wir, dass es reicht,
\begin{equation}\label{*}
	\psi^{2^{p-1}} \equiv{-1} \pmod{n} \tag{*}
\end{equation}
zu pr\"{u}fen.
Durch Nachrechnen \"{u}berpr\"{u}ft man:
\begin{equation}\label{xPsi}
	\psi \,= \,2+ \sqrt{3}\, = \,\left( \frac{1 + \sqrt{3}}{x} \right)^2
\end{equation}
Insgesamt folgt:
\begin{align*}
	 \psi^{2^{p-1}} \quad &\overset{\mathclap{\text{(\ref{xPsi})}}}{=} \quad \left( \frac{1 + \sqrt{3}}{x} \right)	^{2^{p-1}}\\
	 &\overset{\mathclap{\text{(\ref{xQuTwo})}}}{\equiv} \quad \frac{(1 + \sqrt{3})^{2^{p}}}{2^{2^{p-1}}} &&\pmod{n} \\
	 &\overset{\text{(\ref{nPequ})}}{\equiv} \quad \frac{(1 + \sqrt{3})^{n+1}}{2^{\frac{n+1}{2}}}&&\pmod{n}\\
	 &\overset{\text{(\ref{twoPo})}}{\equiv} \quad \frac{(1 + \sqrt{3})^{n+1}}{2} &&\pmod{n}\\
	 &\overset{\mathclap{\text{Lemma\ref{restRing}}}}{\equiv} \quad \frac{(1 + \sqrt{3})(1 + \sqrt{3}^{n})}{2} &&\pmod{n} \\
	 &\equiv \quad \frac{(1 + \sqrt{3})(1 + 3^{\frac{n-1}{2}}\,\sqrt{3})}{2} &&\pmod{n}\\
	 &\overset{\text{(\ref{threePot})}}{\equiv} \quad \frac{(1 + \sqrt{3})(1 - \sqrt{3})}{2} &&\pmod{n}\\
	 &\equiv \quad \frac{1 - 3}{2} &&\pmod{n}\\
	 &\equiv \quad -1 &&\pmod{n}
\end{align*}
damit ist (\ref{*}) gezeigt. \\Also ist die Hinrichtung gezeigt.\\
\end{proof}
Insgesamt ist die Richtigkeit des \emph{Lucas-Lehmer-Tests} bewiesen.
\newpage

\newpage
\bibliography{reference}
\bibliographystyle{ieeetr}
\end{document}