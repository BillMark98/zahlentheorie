\documentclass{article}
\usepackage{amssymb,latexsym,amsmath,amsthm,mathtools}
\theoremstyle{definition}
\newtheorem{theorem}{Theorem}[section]
\newtheorem{lemma}[theorem]{Lemma}
\newtheorem{definition}[theorem]{Definition}
\newtheorem{Algorithmus}{Algorithmus}
\newtheorem{example}[theorem]{Example}
\newtheorem{corollary}[theorem]{Corollary}
\newtheorem{prop}[theorem]{Proposition}
\newtheorem{remark}[theorem]{Remark}
\numberwithin{equation}{theorem}
\numberwithin{figure}{theorem}

\newcommand{\Field}[1]{\ensuremath{\mathbb{F}_{#1}}}
\newcommand{\PolynomialRing}[2]{\ensuremath{#1[x_1,x_2,\ldots,x_{#2}]}}
\newcommand{\WLOG}{w.l.o.g}
\newcommand{\abs}[1]{\ensuremath{|#1|}}
\newcommand{\warningTheorem}{\emph{Warning Theorem}}
\newcommand{\polyDeg}[1]{deg(\ensuremath{#1})}

\begin{document}
    \section{Proof of Kemnitz Conjecture}
    \begin{corollary}\label{cor:origCountingJ3p}
        If $\abs{J} = 3p-3$, then
    \end{corollary}
    \begin{corollary}\label{cor:corCountingJ3p}
        If $|J| = 3p-2$, or $|J| = 3p-1$, then $1 - (p|J) + (2p|J)\equiv 0$ 
    \end{corollary}
    \begin{proof}
        Consider the polynomials in $\PolynomialRing{\Field{p}}{|J|}$,
        \WLOG, $|J| = 3p-2$, the case for $\abs{J} = 3p-1$ is analogous.

        \[\sum_{n=1}^{3p-3} x_n^{p-1}, \quad\sum_{n=1}^{3p-3}a_n x_n^{p-1} \hbox{and} \sum_{n = 1}^{3p-3}b_n x_n^{p-1}\]
        since the total degrees are $3p-3 < 3p-2$, the \warningTheorem{} is applicable.
        Their common zeros are counted as in Corollary~\ref{cor:origCountingJ3p},
        we obtain:
        \begin{alignat}{2}
            &\phantom{\Rightarrow}\quad\quad\quad&1 + (p-1)^p (p|J) + (p-1)^{2p}(2p-1|J) &\equiv 0 \\
            &\Rightarrow &1 - (p|J) + (2p|J) &\equiv 0
        \end{alignat}
    \end{proof}
    \begin{remark}
        Note that in the case where $\abs{J} = 3p-1$ or $3p-2$, the three polynomials we examined
        all have the same number of monomials, i.e. $\abs{J}$, diffrent from in the case in $\abs{J} = 3p-3$. This
        is due to the fact the total degrees of the three polynomials (denoted as $f_i, i \in \{1,2,3\}$) natuarlly satisfy $\sum_{i} \polyDeg{f_i} < \abs{J}$
    \end{remark}

    \begin{corollary}
        If $\abs{X} = 4p-3$, then
        \begin{enumerate}
            \item $-1 + (p|X) - (2p|X) + (3p|X) \equiv 0$
            \item $(p-1|X) - (2p-1|X) + (3p-1|X) \equiv 0$
        \end{enumerate}
    \end{corollary}
    \begin{proof}
        Consider the polynomials $f_1,f_2,f_3$
        \[\sum_{n = 1}^{4p-3}x_n^{p-1}, \quad \sum_{n=1}^{4p-3}a_nx_n^{p-1} \hbox{and}\sum_{n=1}^{4p-3}b_n x_n^{p-1}\]
        The polynomials satisfy $\deg_i \polyDeg{f_i} < 4p-3$, so the \warningTheorem{} is applicable.
        By doing the analogous as in Corollary~\ref{cor:origCountingJ3p}, we obtain the first equation.
        
    \end{proof}
\end{document}