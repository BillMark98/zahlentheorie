\documentclass{article}
\usepackage{amssymb,latexsym,amsmath,amsthm,mathtools}
\theoremstyle{definition}
\newtheorem{theorem}{Theorem}[section]
\newtheorem{lemma}[theorem]{Lemma}
\newtheorem{definition}[theorem]{Definition}
\newtheorem{Algorithmus}{Algorithmus}
\newtheorem{example}[theorem]{Example}
\newtheorem{corollary}[theorem]{Corollary}
\newtheorem{prop}[theorem]{Proposition}
\newtheorem{remark}[theorem]{Remark}
\numberwithin{equation}{theorem}
\numberwithin{figure}{theorem}


% ===============================
% For abs and set
\usepackage{mathtools,nccmath}%
\usepackage{ etoolbox, xparse} 

\DeclarePairedDelimiterX{\abs}[1]\lvert\rvert{\ifblank{#1}{\,\cdot\,}{#1}}

\let\oldabs\abs
\def\abs{\futurelet\testchar\MaybeOptArgAbs}
\def\MaybeOptArgAbs{\ifx[\testchar\let\next\OptArgAbs
\else \let\next\NoOptArgAbs\fi \next}
\def\OptArgAbs[#1]#2{\oldabs[#1]{#2}}
\def\NoOptArgAbs#1{\ifblank{#1}{\oldabs{}}{\oldabs[\big]{#1}}}

\def\Abs{\oldabs*}

\DeclarePairedDelimiterX{\set}[1]\{\}{\setargs{#1}}
\NewDocumentCommand{\setargs}{>{\SplitArgument{1}{;}}m}
{\setargsaux#1}
\NewDocumentCommand{\setargsaux}{mm}
{\IfNoValueTF{#2}{#1}{\nonscript\,#1\nonscript\;\delimsize\vert\nonscript\:\allowbreak #2\nonscript\,}}
%%% Syntaxe : \set{x ; P(x)})
\let\oldset\set
\def\set{\futurelet\testchar\MaybeOptArgSet}
\def\MaybeOptArgSet{\ifx[\testchar \let\next\OptArgSet
\else \let\next\NoOptArgSet \fi \next}
\def\OptArgSet[#1]#2{\oldset[#1]{#2}}
\def\NoOptArgSet#1{\OptArgSet[\big]{#1}}

\def\Set{\oldset*}
% ==============================


\newcommand{\Field}[1]{\ensuremath{\mathbb{F}_{#1}}}
\newcommand{\IntegerP}[1]{\ensuremath{\mathbb{Z}_{#1}}}
\newcommand{\Integer}{\ensuremath{\mathbb{Z}}}
\newcommand{\PolynomialRing}[2]{\ensuremath{#1[x_1,x_2,\ldots,x_{#2}]}}
\newcommand{\WLOG}{w.l.o.g}
% \newcommand{\abs}[1]{\ensuremath{|#1|}}
\newcommand{\setCondition}{\mid}
\newcommand{\warningTheorem}{\emph{Warning Theorem}}
\newcommand{\polyDeg}[1]{deg(\ensuremath{#1})}
\newcommand{\numSumSubset}[2]{\ensuremath{(#1|#2)}}
\newcommand{\circled}[1]{\ensuremath{#1}}


\begin{document}
    \section{Proof of Kemnitz Conjecture}
    For the following section, I use the notation $\equiv$ to denote the modulo in $\IntegerP{p}$.\\
    The $0$ denotes the usual neutral element of addition in the corresponding abelian group. In particular,
    $0$ denotes the standard $0$ in the abelian group $\IntegerP{p}$ and $(0,0)$ in the case of $\IntegerP{p}\times \IntegerP{p}$.
    \begin{corollary}\label{cor:origCountingJ3p}
        If $\abs{J} = 3p-3$, then $1 - \numSumSubset{p - 1}{J} - \numSumSubset{p}{J} + \numSumSubset{2p-1}{J} + \numSumSubset{2p}{J}\equiv 0$
    \end{corollary}
    \begin{corollary}\label{cor:corCountingJ3p}
        If $|J| = 3p-2$, or $|J| = 3p-1$, then $1 - (p|J) + (2p|J)\equiv 0$ 
    \end{corollary}
    \begin{proof}
        Consider the polynomials in $\PolynomialRing{\Field{p}}{|J|}$,
        \WLOG, $|J| = 3p-2$, the case for $\abs{J} = 3p-1$ is analogous.

        \[\sum_{n=1}^{3p-3} x_n^{p-1}, \quad\sum_{n=1}^{3p-3}a_n x_n^{p-1} \hbox{and} \sum_{n = 1}^{3p-3}b_n x_n^{p-1}\]
        since the total degrees are $3p-3 < 3p-2$, the \warningTheorem{} is applicable.
        Their common zeros are counted as in Corollary~\ref{cor:origCountingJ3p},
        we obtain:
        \begin{alignat}{2}
            &\phantom{\Rightarrow}\quad\quad\quad&1 + (p-1)^p (p|J) + (p-1)^{2p}(2p-1|J) &\equiv 0 \\
            &\Rightarrow &1 - (p|J) + (2p|J) &\equiv 0
        \end{alignat}
    \end{proof}
    \begin{remark}
        Note that in the case where $\abs{J} = 3p-1$ or $3p-2$, the three polynomials we examined
        all have the same number of monomials, i.e. $\abs{J}$, diffrent from in the case in $\abs{J} = 3p-3$. This
        is due to the fact the total degrees of the three polynomials (denoted as $f_i, i \in \{1,2,3\}$) natuarlly satisfy $\sum_{i} \polyDeg{f_i} < \abs{J}$
    \end{remark}

    \begin{corollary}\label{cor:3pM13pM2_pJE0Imply2P}
        If $\abs{J} = 3p-2$ or $\abs{J} = 3p-1$, then $\numSumSubset{p}{J} = 0$ implies $\numSumSubset{2p}{J}\equiv -1$.
    \end{corollary}
    \begin{proof}
        This follows directly from Corollary~\ref{cor:corCountingJ3p} by setting $\numSumSubset{p}{J} = 0$.
    \end{proof}
    \begin{corollary}\label{cor:Exact3pZeroSumHasPzeroSeq}
        If $J$ contains exactly $3p$ elements, and $\sum_{x \in J} x \equiv 0$, then $\numSumSubset{p}{J} > 0$.
    \end{corollary}
    \begin{proof}
        Assume that $\numSumSubset{p}{J} = 0$, this means that there are no subsets in $J$, of cardinality $p$ which 
        builds a zero-sum sequence. In particular, if we randomly pick an $x \in J$ and consider the set $J-x$, it follows that
        $\numSumSubset{p}{J-x} = 0$. Now, since $\abs{J-x} = 3p-1$, we can apply Corollary~\ref{cor:3pM13pM2_pJE0Imply2P} to obtain that 
        $\numSumSubset{2p}{J-x} \equiv -1$, in particular, this implies that $\numSumSubset{2p}{J-x} > 0$. \\
        We observe that $\forall A \subset J$, s.t. $\sum_{a \in A} a \equiv 0$,
        \begin{align*}
            \sum_{a\in A} a + \sum_{b\in J-A} b = \sum_{j \in J} j &\equiv 0\\
            \Rightarrow \sum_{b \in J-A} b &\equiv 0
        \end{align*}
        We could see that the map $T$
        \begin{align*}
        T: \set{A\subset J; \abs[]{A} = p, \sum_{a \in A} a \equiv 0}
        &\rightarrow \set{A\subset J; \abs[]{A} = 2p, \sum_{a\in A}a \equiv 0}\\
            A &\mapsto J - A
        \end{align*}
        is a bijection. It follows that:
        \begin{equation*}
            \numSumSubset{p}{J} = \numSumSubset{2p}{J} \geq \numSumSubset{2p}{J - x} > 0,
        \end{equation*}
        which is a contradiction to the assumption $\numSumSubset{p}{J} = 0$
    \end{proof}
    \begin{corollary}\label{cor:4pM3_twoEquations}
        If $\abs{X} = 4p-3$, then
        \begin{enumerate}
            \item $-1 + (p|X) - (2p|X) + (3p|X) \equiv 0$
            \item $(p-1|X) - (2p-1|X) + (3p-1|X) \equiv 0$
        \end{enumerate}
    \end{corollary}
    \begin{proof}
        Consider the polynomials $f_1,f_2,f_3$
        \[\sum_{n = 1}^{4p-3}x_n^{p-1}, \quad \sum_{n=1}^{4p-3}a_nx_n^{p-1} \hbox{and}\sum_{n=1}^{4p-3}b_n x_n^{p-1}\]
        The polynomials satisfy $\deg_i \polyDeg{f_i} < 4p-3$, so the \warningTheorem{} is applicable.
        By doing the analogous as in Corollary~\ref{cor:origCountingJ3p}, we obtain the first equation.
        
    \end{proof}
    \begin{corollary}\label{cor:4pM3_multiSum}
        If $\abs{X} = 4p-3$, then $3 - 2\numSumSubset{p-1}{X} - 2\numSumSubset{p}{X} + \numSumSubset{2p-1}{X} + \numSumSubset{2p}{X} \equiv 0$.
    \end{corollary}

    \begin{proof}
        We deduce from Corollary~\ref{cor:origCountingJ3p} that:
        \[\sum_I 1 - \numSumSubset{p-1}{I} - \numSumSubset{p}{I} + \numSumSubset{2p-1}{I} + \numSumSubset{2p}{I} \equiv 0,\]
        where the sum is over $I\subset X$, s.t, $\abs{I} = 3p-3$. 
        For a given subset $Y \subset X$,s.t $\abs{Y} = p$ and $\sum_{y\in Y} y \equiv 0$, we want to find out the number of pairs
        $(Y,I)$, s.t $Y\subset I, \abs{I} = 3p-3$. \\We could see that 
        \[\abs{\{(Y,I)| Y\subset I, \abs{I} = 3p-3}\} = \binom{3p-3}{2p-3},\]
        since once we have chosen $p$ elements $Y$, we need to further choose $\abs{I} - p = 3p-3 - p = 2p-3$ elements from
        total $\abs{X} - p = 4p-3 -p = 3p-3$ elements.
        Note that
        \[ \sum_{ \substack{Y \subset X,\\\sum_{y\in Y} y= 0}} \sum_{\substack{I,s.t. \\Y\subset I,\\\abs{I} = 3p-3}} 1\]
        \begin{align*}
            \binom{3p-3}{2p-3} \numSumSubset{p}{X} = \sum_{ \substack{Y\subset X\\ \sum_{y\in Y} y = 0}} \binom{3p-3}{2p-3} &= \sum_{ \substack{Y \subset X,\\\sum_{y\in Y} y= 0}} \sum_{\substack{I,s.t. \\Y\subset I,\\\abs{I} = 3p-3}} 1\\
                &=\sum_{\substack{I,s.t.\\\abs{I} = 3p-3}}\;\sum_{ \substack{Y \subset I,\\\sum_{y\in Y} y= 0}}  1 \\
                &= \sum_{I} \numSumSubset{p}{I}
        \end{align*}
        Performing similar calculation, we get
        \begin{align}
            \binom{4p-3}{3p-3} &- \binom{3p-2}{2p-2}\numSumSubset{p-1}{X} - \binom{3p-3}{2p-3}\numSumSubset{p}{X}\nonumber\\ 
            &+ \binom{2p-2}{p-2}\numSumSubset{2p-1}{X} + \binom{2p-3}{p-3}\numSumSubset{2p}{X} \equiv 0 \label{eqn:4p_3BinomModulo}               
        \end{align}
        We finally prove that 
        \begin{equation}\label{eqn:binomModulo1}
            \binom{4p-3}{3p-3} \equiv 3, \binom{3p-2}{2p-2}\equiv 2,            
        \end{equation}
        because:
        \begin{align*}
            \binom{4p-3}{3p-3} &\equiv \frac{(4p-3) \cdots (4p-(p-1)) \cdot 3p \cdot (3p-1) \cdot (3p-2)}{p!}\\
            &\equiv \frac{(4p-3) \cdots (4p-(p-1)) \cdot 3 \cdot (3p-1) \cdot (3p-2)}{(p-1)!} \\
            &\equiv 3 \cdot \frac{(-3) \cdot (-4) \cdots (-(p-1)) \cdot(-1)\cdot(-2)}{(p-1)!}\\
            &\equiv 3 \cdot \frac{(p-1)!}{(p-1)!}\\
            &\equiv 3
        \end{align*}        
        and
        \begin{align*}
            \binom{3p-2}{2p-2} &\equiv \frac{(3p-2) \cdots (3p-(p-1)) \cdot 2p \cdot (2p-1)}{p!}\\
            &\equiv \frac{(3p-2) \cdots (3p-(p-1)) \cdot 2 \cdot (2p-1) \cdot (2p-2)}{(p-1)!} \\
            &\equiv 2 \cdot \frac{(-2) \cdot (-4) \cdots (-(p-1)) \cdot(-1)}{(p-1)!}\\
            &\equiv 2 \cdot \frac{(p-1)!}{(p-1)!}\\
            &\equiv 2
        \end{align*}
        Note that we have used the fact that $p$ is an odd prime, so that $(-1)^{p-1} \equiv 1$.
        Similarly, one can prove that
        \begin{equation}\label{eqn:binomModulo2}
            \binom{3p-3}{2p-3}\equiv 2, \binom{2p-2}{p-2} \equiv 1, \binom{2p-3}{p-3}\equiv 1.
        \end{equation}
        Combining the modulo equivalence in~\eqref{eqn:binomModulo1} and~\eqref{eqn:binomModulo2}, \eqref{eqn:4p_3BinomModulo} can be simplified to
        \begin{equation}
            3 - 2\numSumSubset{p-1}{X} - 2\numSumSubset{p}{X} + \numSumSubset{2p-1}{X} + \numSumSubset{2p}{X} \equiv 0, 
        \end{equation}
        which is what we want to prove.
        \end{proof}

    \begin{lemma}\label{lem:4pM3_pX0_impliesPM1_3PM1}
        If $\abs{X} = 4p- 3$  and $\numSumSubset{p}{X} = 0$, then $\numSumSubset{p-1}{X} \equiv \numSumSubset{3p-1}{X}$.
    \end{lemma}
    \begin{proof}
        We consider the partition of $X = A \bigcup B \bigcup C$, where 
        \[\abs{A} = p-1, \quad \abs{B} = p-2, \quad \abs{C} = 2p.\] and 
        \[\sum_{a \in A} a \equiv 0, \quad \sum_{b \in B} b \equiv \sum_{x \in X} x, \quad \sum_{c \in C} c \equiv 0\]
        Let $\chi$ denote the number of such partition. We use two ways to compute the number $\chi$, the first one fixes $A$ and 
        find out the possible set $C$:
        \[\chi \equiv \sum_{A} \numSumSubset{2p}{X-A} \equiv \sum_{A} -1 \equiv -\numSumSubset{p-1}{X},\]
        where we have used Corollary~\ref{cor:3pM13pM2_pJE0Imply2P}, for $J = X-A$, with $\abs{J} = 3p-2$ and the fact that 
        \[0 \leq \numSumSubset{p}{J} \leq \numSumSubset{p}{X} = 0,\]
        which leads to $\numSumSubset{p}{J} = 0$.
        Now by fixing $B$ and count the possible set $C$, we get:
        \[\chi \equiv \sum_{B} \numSumSubset{2p}{X-B} \stackrel{\circled{1}}{\equiv} \sum_{B} -1 
        \stackrel{\circled{2}}{\equiv} \sum_{X-B}-1 \stackrel{\circled{3}}{\equiv} -\numSumSubset{3p-1}{X}\]
        \begin{enumerate}
            \item  We use the similar argumentation as before, since $\abs{X-B} = 3p-1$ and apply Corollary~\ref{cor:3pM13pM2_pJE0Imply2P}
            leads to $\numSumSubset{2p}{X-B} \equiv -1$.    
            \item The map T defined by:
            \begin{align*}
                T: \set{B\subset X; \abs[]{B} = p - 1, \sum_{b \in B} a \equiv \sum_{x \in X} x}
                &\rightarrow \set{ J \subset X; \abs[]{J} = 3p-1, \sum_{j\in J}j \equiv 0}\\
                    X &\mapsto X - B
            \end{align*}
            is a bijection, so that 
            \[\sum_{B} 1 \equiv \sum_{X -B} 1\]
            \item Since $\sum_{b\in B} b \equiv \sum_{x \in X} x$, it follows that 
            \[\sum_{x \in X - B} x \equiv 0,\]
            in particular 
            \[\sum_{X-B} -1 \equiv -1 \cdot \numSumSubset{3p-1}{X}\]
        \end{enumerate}
        Now we are able to prove the \emph{Kemnitz conjecture}:        
    \end{proof}
    \begin{theorem}
        Any choice of $4p-3$ lattice points in the plane $\Integer \times \Integer$ contains
        a subset of cardinality $p$ whose centroid is a lattice point.
    \end{theorem}
    \begin{proof}
        For the proof of the theorem, we recall the equations we obatined from Corollary~\ref{cor:4pM3_twoEquations} and Corollary~\ref{cor:4pM3_multiSum}:
        \begin{align}
            -1 + (p|X) - (2p|X) + (3p|X) &\equiv 0\\
            (p-1|X) - (2p-1|X) + (3p-1|X) &\equiv 0\\
            3 - 2\numSumSubset{p-1}{X} - 2\numSumSubset{p}{X} + \numSumSubset{2p-1}{X} + \numSumSubset{2p}{X} &\equiv 0.
        \end{align}
        Adding the three above equations, we obtain:
        \begin{equation}\label{eqn:kemnitzSum3Eqn}
            2 - \numSumSubset{p-1}{X} - \numSumSubset{p}{X} + \numSumSubset{3p-1}{X} + \numSumSubset{3p}{X} \equiv 0            
        \end{equation}
        Assume there is a set $X$, with $\abs{X} = 4p-3$ which contradicts the theorem, that is $\numSumSubset{p}{X} = 0$.
        Using the previous Lemma~\ref{lem:4pM3_pX0_impliesPM1_3PM1}, we obtain $\numSumSubset{p-1}{X} \equiv \numSumSubset{3p-1}{X}$
        Then \eqref{eqn:kemnitzSum3Eqn} simplifies to
        \begin{equation}
            2 - \numSumSubset{p}{X} + \numSumSubset{3p}{X} \equiv 0
        \end{equation}
        Since $p$ is odd, we see that $\numSumSubset{p}{X}$ and $\numSumSubset{3p}{X}$ could not both be $0$.
        Since we assume that $\numSumSubset{p}{X} = 0$, it follows that $\numSumSubset{3p}{X} > 0$, i.e., there is a subset $J \subset X$,
        $\abs{J} = 3p$ and $\sum_{j \in J} j \equiv 0$. 
        But from Corollary~\ref{cor:Exact3pZeroSumHasPzeroSeq}, we see that $\numSumSubset{p}{J} > 0$, in particular $\numSumSubset{p}{X} > 0$,
        which is a contradiction.
    \end{proof}
\end{document}