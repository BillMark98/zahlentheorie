%%%%%%%%%%%%%%%%%%%%%%%%%%%%%%%%%%%%%%%%%
% Beamer Presentation
% LaTeX Template
% Version 1.0 (10/11/12)
%
% This template has been downloaded from:
% http://www.LaTeXTemplates.com
%
% License:
% CC BY-NC-SA 3.0 (http://creativecommons.org/licenses/by-nc-sa/3.0/)
%
%%%%%%%%%%%%%%%%%%%%%%%%%%%%%%%%%%%%%%%%%

%----------------------------------------------------------------------------------------
%	PACKAGES AND THEMES
%----------------------------------------------------------------------------------------

\documentclass[notheorems, envcountsect]{beamer}

\mode<presentation> {

% The Beamer class comes with a number of default slide themes
% which change the colors and layouts of slides. Below this is a list
% of all the themes, uncomment each in turn to see what they look like.

%\usetheme{default}
%\usetheme{AnnArbor}
%\usetheme{Antibes}
%\usetheme{Bergen}
%\usetheme{Berkeley}
%\usetheme{Berlin}
%\usetheme{Boadilla}
%\usetheme{CambridgeUS}
%\usetheme{Copenhagen}
%\usetheme{Darmstadt}
%\usetheme{Dresden}
\usetheme{Frankfurt}
% \usetheme{Goettingen}
%\usetheme{Hannover}
%\usetheme{Ilmenau}
%\usetheme{JuanLesPins}
% \usetheme{Luebeck}
% \usetheme{Madrid}
%\usetheme{Malmoe}
%\usetheme{Marburg}
%\usetheme{Montpellier}
%\usetheme{PaloAlto}
%\usetheme{Pittsburgh}
%\usetheme{Rochester}
%\usetheme{Singapore}
%\usetheme{Szeged}
%\usetheme{Warsaw}

% As well as themes, the Beamer class has a number of color themes
% for any slide theme. Uncomment each of these in turn to see how it
% changes the colors of your current slide theme.

%\usecolortheme{albatross}
%\usecolortheme{beaver}
%\usecolortheme{beetle}
%\usecolortheme{crane}
%\usecolortheme{dolphin}
%\usecolortheme{dove}
%\usecolortheme{fly}
%\usecolortheme{lily}
%\usecolortheme{orchid}
%\usecolortheme{rose}
%\usecolortheme{seagull}
%\usecolortheme{seahorse}
%\usecolortheme{whale}
%\usecolortheme{wolverine}

%\setbeamertemplate{footline} % To remove the footer line in all slides uncomment this line
%\setbeamertemplate{footline}[page number] % To replace the footer line in all slides with a simple slide count uncomment this line

%\setbeamertemplate{navigation symbols}{} % To remove the navigation symbols from the bottom of all slides uncomment this line
}
\usepackage{amsmath,amsfonts}
\usepackage{graphicx} % Allows including images
\usepackage{booktabs} % Allows the use of \toprule, \midrule and \bottomrule in tables
\usepackage{algorithmic} % algorithm
\usepackage{pgfgantt}
\usepackage{tikz}
\usepackage{tikz-cd}
\usetikzlibrary{arrows.meta,positioning}
\usetikzlibrary{matrix,backgrounds}
\newcommand{\newKeyword}{\textbf{new }}
\newcommand{\MWIS}[1]{\hbox{MWIS(\ensuremath{#1})}}
\newcommand*{\QEDA}{\hfill\ensuremath{\blacksquare}}
\newcommand{\obdA}{o.B.d.A}
\newcommand{\myEmptySet}{\ensuremath{(\emptyset, -1, 0)}}
\newcommand\textganttbar[4]{%
    \ganttbar{#1}{#3}{#4}
    \ganttbar[inline,bar label font=\footnotesize]
    {#2}{#3}{#4}
}
\newcommand{\MAXFLOW}{\hbox{MAXFLOW}}
\newcommand{\maxFlowFunc}[1]{\ensuremath{\hbox{MAXFLOW}(#1)}}


% \newcommand{\minus}{-}

%======================================================
%======================================================

% import from thesis.tex

%-------------------------------------------------------

\usepackage{amssymb,latexsym,amsmath, mathtools}
% \usepackage{amsthm}
% \setbeamertemplate{thm}[ams style]
% \setbeamertemplate{theorems}[numbered]


% \theoremstyle{defn}
% \newtheorem{thm}{Theorem}[section]
% \newtheorem{mainthm}{Main Theorem}
% \newtheorem*{mainTheoremNoNumber}{Main Theorem}

% \newtheorem{lem}[thm]{Lemma}
% \newtheorem{prop}[thm]{Proposition}
% \newtheorem{property}[thm]{Property}
% \newtheorem{defn}[thm]{Definition}
% \newtheorem{Algorithmus}{Algorithmus}
% \newtheorem{example}[thm]{Example}
% \newtheorem{cor}[thm]{Corollary}
% \newtheorem{prop}[thm]{Proposition}
% \newtheorem{remark}[thm]{Remark}
% \theoremstyle{defn}
% % \newtheorem{prop}[thm]{Proposition}
% % \newtheorem{defn}[thm]{Definition}
% % \newtheorem{cor}[thm]{Corollary}
% \newtheorem*{goal}{Goal}
% \newtheorem*{observation}{Observation}
% \newtheorem{prob}[thm]{Problem}
% \newtheorem{notation}[thm]{Notation}
% % \newtheorem{prob}{Problem}
% % \newtheorem{property}{Property}
% \numberwithin{equation}{thm}
% \numberwithin{figure}{thm}


% tweak \newtheorem
\let\originalnewtheorem\newtheorem
\RenewDocumentCommand{\newtheorem}{ommo}{%
  \originalnewtheorem{#2inner}{#3\thisthmnumber}
  \NewDocumentEnvironment{#2}{od()}
   {%
    \IfValueT{##2}{\renewcommand{\thisthmnumber}{ ##2}}%
    \IfValueTF{##1}{\begin{#2inner}[##1]}{\begin{#2inner}}%
   }
   {\end{#2inner}}
}
\newcommand{\thisthmnumber}{}


\theoremstyle{plain}
\newtheorem{thm}{Theorem}[section]
\newtheorem[thm]{lem}{Lemma}
\newtheorem[thm]{prop}{Proposition}
\newtheorem[thm]{cor}{Corollary}
\newtheorem[thm]{mainthm}{Main Theorem}

\newtheorem[thm]{mainTheoremNoNumber}{Main Theorem}


\theoremstyle{definition}
\newtheorem[thm]{defn}{Definition}

\newtheorem{goal}{Goal}
\newtheorem{observation}{Observation}
\newtheorem[thm]{prob}{Problem}
\newtheorem[thm]{remark}{Remark}
\newtheorem[thm]{notation}{Notation}
\newtheorem{property}{Property}




% \theoremstyle{plain}
% \newtheorem{thm}{Theorem}[section]
% \newtheorem[thm]{lem}{Lemma}
% \newtheorem[thm]{prop}{Proposition}
% \newtheorem[thm]{cor}{Corollary}
% \newtheorem{mainthm}{Main Theorem}

% \newtheorem[thm]{mainTheoremNoNumber}{Main Theorem}


% \theoremstyle{defn}
% \newtheorem[thm]{defn}{Definition}

% \newtheorem{goal}{Goal}
% \newtheorem{observation}{Observation}
% \newtheorem[thm]{prob}{Problem}



% 
% for enumerate item
% \usepackage{enumitem}
\usepackage{enumerate}



% ================================
% redefine thm for frame breaks
% \setbeamertemplate{thm begin}
% {%
%   \par\vskip\medskipamount%
%   \begin{beamercolorbox}[colsep*=.75ex]{block title}
%     \usebeamerfont*{block title}%
%       \inserttheoremname
%       \ifx\inserttheoremaddition\empty\else\ (\inserttheoremaddition)\fi%
%   \end{beamercolorbox}%
%   {\parskip0pt\par}%
%   \ifbeamercolorempty[bg]{block title}
%   {}
%   {\ifbeamercolorempty[bg]{block body}{}{\nointerlineskip\vskip-0.5pt}}%
%   \usebeamerfont{block body}%
%   \vskip-.25ex\vbox{}%
% }
% \setbeamertemplate{thm end}{}

\newcommand*{\theorembreak}{\usebeamertemplate{thm end}\framebreak\usebeamertemplate{thm begin}}

\makeatletter
\newenvironment<>{proofs}[1][\proofname]{%
    \par
    \def\insertproofname{#1\@addpunct{.}}%
    \usebeamertemplate{proof begin}#2}
  {\usebeamertemplate{proof end}}
\newenvironment<>{proofc}{%
  \setbeamertemplate{proof begin}{\begin{block}{}}
    \par
    \usebeamertemplate{proof begin}}
  {\usebeamertemplate{proof end}}
\newenvironment<>{proofe}{%
    \par
    \pushQED{\qed}
    \setbeamertemplate{proof begin}{\begin{block}{}}
    \usebeamertemplate{proof begin}}
  {\popQED\usebeamertemplate{proof end}}
\makeatother
% ===================


% ===============================
% For abs and set
\usepackage{mathtools,nccmath}%
\usepackage{ etoolbox, xparse} 

% ===============
% for contradiction symbol
% \usepackage{wasysym}  % \lightning
\usepackage{marvosym} %\Lightning
\usepackage{stmaryrd} % $\lightning

%========================
% for bibtex url
\usepackage{url}

%=================================
%================================
% title page

% \newcommand{\ThesisTitle}[1]{\newcommand{\thesistitle}{#1}}

% \newcommand{\ThesisDate}[3]{\newcounter{thesisday}

% \newcommand{\ThesisAuthor}[1]{\newcommand{\thesisauthor}{#1}}
% \newcommand{\FirstExaminerName}[1]{\newcommand{\firstexaminername}{#1}}
% \newcommand{\SecondExaminerName}[1]{\newcommand{\secondexaminername}{#1}}
% \newcommand{\SupervisorNames}[1]{\newcommand{\supervisornames}{#1}}



% \AtBeginDocument{%
% \hypersetup{
% 	pdftitle={\thesistitle},
% 	pdfauthor={\thesisauthor},
% 	% pdfsubject={\thesistype, Institute for Networked Systems, RWTH Aachen University},
% 	bookmarksnumbered=true,
% 	bookmarksopen=true,
% 	bookmarksopenlevel=1,
% 	pdfborder={0 0 0},
% 	plainpages=false,
% 	pdfcreator={\thesisauthor},
% }

%================================
\DeclarePairedDelimiterX{\abs}[1]\lvert\rvert{\ifblank{#1}{\,\cdot\,}{#1}}

\let\oldabs\abs
\def\abs{\futurelet\testchar\MaybeOptArgAbs}
\def\MaybeOptArgAbs{\ifx[\testchar\let\next\OptArgAbs
\else \let\next\NoOptArgAbs\fi \next}
\def\OptArgAbs[#1]#2{\oldabs[#1]{#2}}
\def\NoOptArgAbs#1{\ifblank{#1}{\oldabs{}}{\oldabs[\big]{#1}}}

\def\Abs{\oldabs*}

\DeclarePairedDelimiterX{\set}[1]\{\}{\setargs{#1}}
\NewDocumentCommand{\setargs}{>{\SplitArgument{1}{;}}m}
{\setargsaux#1}
\NewDocumentCommand{\setargsaux}{mm}
{\IfNoValueTF{#2}{#1}{\nonscript\,#1\nonscript\;\delimsize\vert\nonscript\:\allowbreak #2\nonscript\,}}
%%% Syntaxe : \set{x ; P(x)})
\let\oldset\set
\def\set{\futurelet\testchar\MaybeOptArgSet}
\def\MaybeOptArgSet{\ifx[\testchar \let\next\OptArgSet
\else \let\next\NoOptArgSet \fi \next}
\def\OptArgSet[#1]#2{\oldset[#1]{#2}}
\def\NoOptArgSet#1{\OptArgSet[\big]{#1}}

\def\Set{\oldset*}

% =====================
% really wide hat
\usepackage{scalerel,stackengine}
\stackMath
\newcommand\reallywidehat[1]{%
\savestack{\tmpbox}{\stretchto{%
  \scaleto{%
    \scalerel*[\widthof{\ensuremath{#1}}]{\kern-.6pt\bigwedge\kern-.6pt}%
    {\rule[-\textheight/2]{1ex}{\textheight}}%WIDTH-LIMITED BIG WEDGE
  }{\textheight}% 
}{0.5ex}}%
\stackon[1pt]{#1}{\tmpbox}%
}
\parskip 1ex


% ==============================

% =========
% theroem
% \newcommand{\warningTheorem}{\ensuremath{\operatorname{\hbox{\emph{Chevalley-Warning Theorem}}}}}
% \newcommand{\warningTheorem}{\ensuremath{\textbf{Chevalley-Warning Theorem}}}
\newcommand{\warningTheorem}{\emph{Chevalley\,-Warning Theorem}}
\newcommand{\kemnitzConjecture}{\emph{Kemnitz' Conjecture}}
\newcommand{\alonDubinerTheorem}{\emph{Alon-Dubiner Theorem}}

% ========
% number ring, fields
\newcommand{\Field}[1]{\ensuremath{\mathbb{F}_{#1}}}
\newcommand{\IntegerP}[1]{\ensuremath{\mathbb{Z}_{#1}}}
\usepackage{faktor}
\newcommand{\modularInteger}[1]{\ensuremath{\faktor{\mathbb{Z}}{#1\-\mathbb{Z}}}}
% \newcommand{\modularInteger}[1]{\ensuremath{\mathbb{Z}/#1\mathbb{Z}}}
\newcommand{\NaturalNumber}{\ensuremath{\mathbb{N}}}
\newcommand{\Real}{\ensuremath{\mathbb{R}}}
\newcommand{\Complex}{\ensuremath{\mathbb{C}}}
\newcommand{\ComplexUnit}{\ensuremath{\mathbb{C}^{*}}}

\newcommand{\Integer}{\ensuremath{\mathbb{Z}}}
\newcommand{\PolynomialRing}[2]{\ensuremath{#1[x_1,x_2,\ldots,x_{#2}]}}


% ================
% german alphabet
\newcommand{\oUmlaut}{{\"o}}
\newcommand{\uUmlaut}{{\"u}}
\newcommand{\aUmlaut}{{\"a}}

% ======
% notations
\newcommand{\zeroSumSeq}[1]{$0$-sum $#1$-subsequence}
\newcommand{\cayleyGraph}[2]{\ensuremath{\Gamma_{#1,#2}}}
\newcommand{\adjacencyMatrixCayley}[2]{\ensuremath{A_{#1,#2}}}
\newcommand{\characterGroup}[1][G]{\ensuremath{\reallywidehat{#1}}}
\DeclarePairedDelimiter{\roundCeil}\lceil\rceil

% =========
% math symbols
\newcommand{\means}{\ensuremath{\reallywidehat{=}}}


% =======
% some functions
\newcommand{\fnd}[2]{\ensuremath{f(#1,#2)}}
\newcommand{\bigO}[1]{\ensuremath{\mathcal{O}(#1)}}
\newcommand{\myAlphabetSubSupscript}[3]{\ensuremath{#1_{#2}^{#3}}}
\newcommand{\mySup}[1]{\sup{#1}}
\newcommand{\myInf}[1]{\inf{#1}}
\newcommand{\bilinearForm}[2]{\ensuremath{\langle#1,#2\rangle}}
\newcommand{\composition}[2]{\ensuremath{#1\circ#2}}
\newcommand{\sothat}{s.t.\ }
\newcommand{\TODO}{\textbf{!!!!!!! To Do !!!!!!!!!}}
\newcommand{\Dom}[1]{\ensuremath{\hbox{Dom}(#1)}}
% ===========
% environment
\newenvironment{case}
    {\begin{enumerate}[label = \textbf{Case }{\arabic* }:]}
        {\end{enumerate}}

\newenvironment{enumeratei}{\begin{enumerate}[label = (\roman{enumii})]}
            {\end{enumerate}}

\newcommand{\WLOG}{w.l.o.g}
% \newcommand{\abs}[1]{\ensuremath{|#1|}}

% ======
% math symbol
\newcommand{\setCondition}{\mid}
\newcommand{\mySetMinus}{\setminus}
\newcommand{\minus}{-}
\newcommand{\myMin}[1]{\ensuremath{\min \{#1\}}}
\newcommand{\myMax}[1]{\ensuremath{\max \{#1\}}}

\newcommand{\contradiction}{\ensuremath{\lightning}}
% \newcommand{\negative}{-}

\newcommand{\polyDeg}[1]{deg(\ensuremath{#1})}
\newcommand{\numSumSubset}[2]{\ensuremath{(#1|#2)}}
\newcommand{\circled}[1]{\ensuremath{#1}}

%------------------------------------------



% ---------------------------------
% for presentation added
\newcommand*{\Scale}[2][4]{\scalebox{#1}{$#2$}}%
\newcommand*{\Resize}[2]{\resizebox{#1}{!}{$#2$}}%




%----------------------------------------------------------------------------------------
%	TITLE PAGE
%----------------------------------------------------------------------------------------

\title{Proof of Kemnitz' Conjecture and a generalization to higher dimensions} % The short title appears at the bottom of every slide, the full title is only on the title page


\author{Panwei Hu} % Your name
% \institute[UCLA] % Your institution as it will appear on the bottom of every slide, may be shorthand to save space
% {
% University of California \\ % Your institution for the title page
% \medskip
% \textit{john@smith.com} % Your email address
% }
\date{} % Date, can be changed to a custom date

\begin{document}

\begin{frame}
\titlepage % Print the title page as the first slide
\end{frame}

% \begin{frame}
% \frametitle{Overview} % Table of contents slide, comment this block out to remove it
% \tableofcontents % Throughout your presentation, if you choose to use \section{} and \subsection{} commands, these will automatically be printed on this slide as an overview of your presentation
% \end{frame}

%----------------------------------------------------------------------------------------
%	PRESENTATION SLIDES
%----------------------------------------------------------------------------------------
%------------------------------------------------
% \section{Abstract} % Sections can be created in order to organize your presentation into discrete blocks, all sections and subsections are automatically printed in the table of contents as an overview of the talk
%------------------------------------------------

% \subsection{Subsection Example} % A subsection can be created just before a set of slides with a common theme to further break down your presentation into chunks


\section{Introduction}
\begin{frame}
\frametitle{Introduction}
We define a $d$-dimensional affine space $V$ and consider the set of points which lie in the set
    \[V_d := \set{\sum_{i = 1}^d \alpha_i v_i; \alpha_i \in \Integer, \quad 1 \leq i \leq d},\]
    where $\set{v_i; 1 \leq i \leq d}$ are linear independent vectors in $V$. 
    We also call the points in $V_d$ as \emph{lattice} points.
\begin{prob}(1)\label{prob:centroid}
    Find out the minimum of the number $f$ \sothat given $f$ sequences in $V_d$, we can guarantee to find out a subsequence of length $n$, \sothat the centroid of this 
    subsequence is also a lattice point. We define such minimum number as $f(n,d)$.
\end{prob}
    % \begin{goal}
    %     Find the exact value of $f(n,d)$, if not possible, study the upper and lower bound of $f(n,d)$
    % \end{goal} 
\end{frame}

% -------------------

\begin{frame}
\frametitle{Goal}
\begin{goal}
Find the exact value of $f(n,d)$, if not possible, study the upper and lower bound of $f(n,d)$
\end{goal}
\pause
Consider the additive group $G:= \IntegerP{n}^d$.
We call a subsequence of length $l$, which 
    sums to a $0$ in $\IntegerP{n}^d$ as 
    \zeroSumSeq{l} (w.r.t $G$), where $0$ denotes the zero vector
     in $G$.  
\pause
\begin{prob}(2)\label{prob:zeroSumSeq}
    Find the 
    number $f(n,d)$, \sothat 
    for any sequences of elements in $G$, with 
    length $l \geq f(n,d)$, there exists a \zeroSumSeq{n}.
\end{prob}
\pause
\begin{observation}
    Problem~\ref{prob:centroid} and Problem~\ref{prob:zeroSumSeq} are equivalent
\end{observation}
\end{frame}

% %------------------

% \begin{frame}
% \frametitle{Multiset operation}
% % $f : A \rightarrow K$, where $K$ is the cardinal number. We thus extend the operation on sets to the multisets as following:
% \begin{defn}(2.1)[via examples]\label{def:multisetOperation}
%     Let $A = \set{1,1,1,2,2,3,3,3,3}, B = \set{1,1,2,2,2,3}$ be two multisets.\\
%     The \emph{union},\emph{intersection},\emph{difference} between $A$ and $B$ is given by:\\
%     \begin{itemize}
%         \item $A \cup B = \set{1,1,1,2,2,2,3,3,3,3}$
%         \item $A \cap B = \set{1,1,2,2,3}$
%         \item $A \setminus B = \set{1,3,3,3}$
%     \end{itemize}
% \end{defn}\end{frame}

%------------------

\subsection{Bounds on $\fnd{n}{d}$}

\begin{frame}
    \frametitle{A natural bound on $f(n,d)$}
    \begin{lem}(2.2)\label{lem:f_n_d_naturalBound}
        \begin{equation}\label{eqn:fndProp1}
            (n-1) 2^d + 1 \leq \fnd{n}{d} \leq (n-1)n^d + 1
        \end{equation}
    \end{lem}
    \pause
    \begin{proof}
        Left inequality: 
        Construct $(n-1) 2^d$ vectors, which include all the vectors in $\IntegerP{n}^d$, which has $0$ or $1$ in their entry, so there are in all $2^d$ different vectors.
        Each vector appear exactly $n-1$ times.  It is impossible to find a \zeroSumSeq{n} among these vectors.
\pause

        Right inequality: pigeon hole principle

        Since $\abs{G} = n^d$, given $(n-1)n^d + 1$ elements, there are at least one vector $v$ which has 
        multiplicity
        \[\roundCeil{\frac{(n-1) n^d + 1}{n^d}} = n.\]
    \end{proof}    
\end{frame}

%     %------------------
\begin{frame}
    \frametitle{Decomposition of $f(n,d)$ (1)}
    \begin{lem}(2.3)
        \begin{equation}\label{eqn:fndProp2}
            \fnd{pq}{d} \leq \fnd{p}{d} + p(\fnd{q}{d} - 1) \tag{2.3.1}
        \end{equation}
    \end{lem}
    \begin{proof}
        For the convenience of notation, we define 
        \[f_1 := \fnd{p}{d}, f_2 := \fnd{q}{d}, f := f_1 + p(f_2 - 1).\] 
        since 
        \[f = f_1 + p (f_2 - 1),\]
        we would obtain $f_2$ \zeroSumSeq{p} (w.r.t $\IntegerP{p}^d$). 
        Among all these $f_2$ vectors, there exists a \zeroSumSeq{q} (w.r.t $\IntegerP{q}^d$), which means 
        they sum to a vector $z$, with each component divisible by $q$. Since each summand
        has components all divisible by $p$, the resultant vectors will be divisible by $pq = n$, thus we obtain a \zeroSumSeq{n}.
    \end{proof}
\end{frame}
%------------------------------------------------

\begin{frame}
    \frametitle{Decomposition of $f(n,d)$ (2)}
    Due to the symmetry we would obtain similarly:
    \begin{equation}\label{eqn:fndProp2_prime}
        \fnd{pq}{d} \leq \fnd{q}{d} + q(\fnd{p}{d} - 1 ) \tag{2.3.1'} %\tag{$\ref{eqn:fndProp2}^\prime$}
    \end{equation}
    \pause
    Combining \eqref{eqn:fndProp2} and \eqref{eqn:fndProp2_prime}, we obtain the following upper bound.
    \begin{cor}(2.5)\label{cor:fnd_pq_Bound}
        \begin{equation}\label{eqn:fnd_pq_Bound}
            \fnd{pq}{d} \leq \min \{\fnd{p}{d} + p(\fnd{q}{d} - 1),  \fnd{q}{d} + q(\fnd{p}{d} - 1)\}
        \end{equation}
    \end{cor}
    \pause
    \begin{thm}(2.5)[Cauchy-Davenport]\label{thm:cauchy_davenport}
        Let $p$ be a prime number. If $A, B \subset \IntegerP{p}$ are nonempty, then 
        \[\abs{A + B} \geq \myMin{p, \abs{A} + \abs{B} - 1},\]
        where $A + B := \set{ a + b ; a \in A, b \in B }$
    \end{thm}
\end{frame}

% %------------------------------------------------

\begin{frame}
    \frametitle{Examples}
    \begin{thm}(2.7)[Erd\oUmlaut s-Ginsburg-Ziv]\label{thm:Erdos2NM1}
        \begin{equation*}
            \fnd{n}{1} = 2n - 1  
        \end{equation*}
    \end{thm}  
    \begin{proof}[Proof sketches]
        From \eqref{eqn:fndProp1}:
        \begin{equation}\label{ieqn:fnd_n_1_get_2n_1}
            \fnd{n}{1} \geq 2 n - 1
        \end{equation}
        Only need to show
        \begin{equation}\label{ieqn:fnd_n_1_leq_2n_1}
            \fnd{n}{1} \leq 2n - 1.
        \end{equation}   
    \pause         
    Recall $\qquad\fnd{pq}{d} \leq \fnd{p}{d} + p(\fnd{q}{d} - 1)$
\pause
        \begin{enumerate}
            \item restrict to prime number ($\fnd{pq}{1} \leq 2pq -1$)
            \item application of Theorem~\ref{thm:cauchy_davenport} to prove 
        \[\fnd{p}{1} \leq 2p - 1\]
        \end{enumerate}
    \end{proof}
\end{frame}

%------------------------------------------------

\begin{frame}
    \frametitle{Proof of \eqref{ieqn:fnd_n_1_leq_2n_1}(1)}
\end{frame}

%------------------------------------------------

\begin{frame}
    \frametitle{Proof of \eqref{ieqn:fnd_n_1_leq_2n_1}(2)}
\end{frame}
%------------------------------------------------

\begin{frame}
    \frametitle{Further examples}
    \begin{lem}(2.8)\label{lem:f_nd_2_potent}
        \begin{equation*}
            \fnd{2^n}{d} = (2^n - 1) 2^d + 1
        \end{equation*}
    \end{lem}
    % \begin{lem}\label{lem:f_3_2_Value_Harborth}
    %     \begin{equation*}
    %         \fnd{3}{2} = 9
    %     \end{equation*}
    % \end{lem}
    \pause
    \begin{lem}(2.9)\label{lem:f_3Potent_2_value}
        \begin{equation*}
            \fnd{3^n}{2} = 4 \cdot 3^n - 3
        \end{equation*}
    \end{lem}
    \pause
    % \begin{remark}
    %     By setting $d = 2$ in Lemma~\ref{lem:f_nd_2_potent}, we see that
    %     \[\fnd{2^n}{2} = 4 \cdot 2^n - 3\]
    %     we see that there is a simlar structure as in $\fnd{3^n}{2}$. 
    %     A natural question emerges that if this is the case for all $n \in \Integer$, that is
    % \end{remark}
    \begin{prob}
        Is it true for all $n \in \NaturalNumber$,
        \[\fnd{n}{2} = 4n - 3\]
        This is the well-known \kemnitzConjecture{}
    \end{prob}
\end{frame}

\section{Proof of \kemnitzConjecture{}}
%------------------------------------------------
\begin{frame}
    \begin{remark}
        Notation: $\equiv$ means modulo in $\IntegerP{p}$.\\
        The $0$ denotes the usual neutral element of addition in the corresponding abelian group. In particular,
        $0$ denotes the standard $0$ in the abelian group $\IntegerP{p}$ and $(0,0)$ in the case of $\IntegerP{p}\times \IntegerP{p}$.
    \end{remark}
    \pause
    \begin{thm}[\warningTheorem](3.1)\label{thm:warning}
        Let $p$ be a prime number and $q = p^t, t \in \NaturalNumber$. We use $\Field{q}$ to denote the finite field of $q$ elements.
        Let $p_1,\ldots,p_m$ be $m$ polynomials in $\PolynomialRing{\Field{q}}{n}$, with degree $d_i$. Denote the number of common
        zeros of the $m$ polynomials as $N$.
        If 
        \[\sum_{i = 1}^{m} d_i < n,\]
        then 
        \[N \equiv 0 \pmod{p}\]
    \end{thm}    
\end{frame}
%------------------------------------------------

% %------------------------------------------------
% \begin{frame}
%     \begin{mainthm}[\kemnitzConjecture{}]
%         Any choice of $4p-3$ lattice points in the plane $\Integer \times \Integer$ contains
%         a subset of cardinality $p$ whose centroid is a lattice point. In other words:
%         \[\fnd{n}{2} = 4n - 3\]
%     \end{mainthm}
% \end{frame}
% ------------------------------------------------

%------------------------------------------------
\begin{frame}
    \frametitle{First reduction}
    \begin{mainTheoremNoNumber}[\kemnitzConjecture{}]
     \[\fnd{n}{2} = 4n - 3\]
    \end{mainTheoremNoNumber}
    \begin{prop}
        It suffices to consider the case for $n$ is an odd prime number.
    \end{prop}
    \pause
    \begin{notation}
        Denote $J,X$ and other capital alphabets
        as a multiset of $G := \IntegerP{p} \times \IntegerP{p}$.\\
            \[
                \numSumSubset{m}{J} \quad \means{} \quad \text{number of \zeroSumSeq{m} (w.r.t G) in J}.
            \]
    \end{notation}
\end{frame}
%------------------------------------------------

%------------------------------------------------
\begin{frame}
    \begin{cor}(3.2)\label{cor:origCountingJ3p}
        If $\abs{J} = 3p-3$, then $1 - \numSumSubset{p - 1}{J} - \numSumSubset{p}{J} + \numSumSubset{2p-1}{J} + \numSumSubset{2p}{J}\equiv 0$
    \end{cor}
    \pause
    We specify $J$ as the multiset $\set{(a_n,b_n); 1 \leq n \leq 3p-3}$.\\

    We consider three polynomials
        \[p_1 := \sum_{n=1}^{3p-3} x_n^{p-1} + x_{3p-2}^{p-1}, \quad p_2 := \sum_{n=1}^{3p-3}a_n x_n^{p-1}, \quad p_3:= \sum_{n = 1}^{3p-3}b_n x_n^{p-1}\]
    \pause
    Since the total sum of degress are $3(p-1) = 3p-3 < 3p-2$, we can apply the \warningTheorem{}.\\
    \pause
    Since three polynomials have $0$ as a common zero, the \warningTheorem{} states that there are non-trivial common zeros.
        We consider the common zeros depending on the term $x_{3p-2}$ in $p_1$.
            
\end{frame}
%------------------------------------------------

%------------------------------------------------
\begin{frame}
    \frametitle{Proof of Corollary 3.2(1) $x_{3n-2} = 0$}
    $x_{3n-2} = 0$\\
\pause
    \begin{columns}
        \begin{column}{0.4\textwidth}
            \begin{small}
            \begin{align*}
                p_1 :&= \sum_{n=1}^{3p-3} x_n^{p-1} + x_{3p-2}^{p-1} \\
                p_2 :&= \sum_{n=1}^{3p-3}a_n x_n^{p-1}\\
                p_3:&= \sum_{n = 1}^{3p-3}b_n x_n^{p-1}
            \end{align*}
        \end{small}
        % \[\Scale[0.5]{
        %     \begin{align*}
        %         p_1 :&= \sum_{n=1}^{3p-3} x_n^{p-1} + x_{3p-2}^{p-1} \\
        %         p_2 :&= \sum_{n=1}^{3p-3}a_n x_n^{p-1}\\
        %         p_3:&= \sum_{n = 1}^{3p-3}b_n x_n^{p-1}
        %     \end{align*}}
        %     \]
        % \small
        %     \begin{align*}
        %         p_1 :&= \sum_{n=1}^{3p-3} x_n^{p-1} + x_{3p-2}^{p-1} \\
        %         p_2 :&= \sum_{n=1}^{3p-3}a_n x_n^{p-1}\\
        %         p_3:&= \sum_{n = 1}^{3p-3}b_n x_n^{p-1}
        %     \end{align*}
        
            \end{column}
            \pause
        
        \begin{column}{0.6\textwidth}
            % $x_{3n-2} = 0$\\
            In this case for the zero of $p_1$, it needs to satisfy
            \[\sum_{n = 1}^{3p-3} x_{n}^{p-1} \equiv 0\]
    
        \end{column}
        
    \end{columns}
    % $x_{3n-2} = 0$\\
    %         In this case for the zero of $p_1$, it needs to satisfy
    %         \[\sum_{n = 1}^{3p-3} x_{n}^{p-1} \equiv 0\]
            We know that 
            \begin{equation*}
                x^{p-1} \equiv 
                \begin{cases}
                    1, &\text{if $x \not\equiv 0$;}\\
                    0, &\text{if $x \equiv 0$.}
                \end{cases}
            \end{equation*}
            Since there are in all $3p-3$ variables left, there could only be three cases:
\end{frame}
%------------------------------------------------

%------------------------------------------------
\begin{frame}
    \frametitle{Proof of Corollary 3.2(2)}
    \begin{columns}
        \begin{column}{0.4\textwidth}
            \begin{small}
            \begin{align*}
                p_1 :&= \sum_{n=1}^{3p-3} x_n^{p-1} + x_{3p-2}^{p-1} \\
                p_2 :&= \sum_{n=1}^{3p-3}a_n x_n^{p-1}\\
                p_3:&= \sum_{n = 1}^{3p-3}b_n x_n^{p-1}
            \end{align*}
        \end{small}
    \end{column}
    \pause

\begin{column}{0.6\textwidth}
    \begin{enumeratei}
        \item $0$ of them are $1$: \# $1$
        \pause
        \item $p$ of them are $1$: 
        $x_{i_1},\ldots,x_{i_p}$.
        \pause
        \[\sum_{j = 1}^{p} a_{i_j} \equiv 0, \quad \sum_{j = 1}^{p} b_{i_j} \equiv 0.\]
        \# $(p-1)^p \numSumSubset{p}{J}$.
        \item $2p$ of them are $1$:\\
        \# $(p-1)^{2p}\numSumSubset{2p}{J}$.
    \end{enumeratei}
\end{column}

\end{columns}

\end{frame}
%------------------------------------------------



%------------------------------------------------
\begin{frame}
    \frametitle{Proof of Corollary 3.2(3)}
    \begin{columns}
        \begin{column}{0.4\textwidth}
            \begin{small}
            \begin{align*}
                p_1 :&= \sum_{n=1}^{3p-3} x_n^{p-1} + x_{3p-2}^{p-1} \\
                p_2 :&= \sum_{n=1}^{3p-3}a_n x_n^{p-1}\\
                p_3:&= \sum_{n = 1}^{3p-3}b_n x_n^{p-1}
            \end{align*}
        \end{small}
    \end{column}
    \pause

\begin{column}{0.6\textwidth}
    \begin{enumeratei}
        \item $p-1$ of them are $1$. \\
            \# $(p-1)^p \numSumSubset{p-1}{J}$
        \pause
        \item $2p-1$ of them are $1$.\\
        \# $(p-1)^{2p} \numSumSubset{2p-1}{J}$
    \end{enumeratei}

\end{column}

\end{columns}
\pause
Collecting all the number of common zeros:
            \begin{align*}
                1 + (p-1)^p\numSumSubset{p}{J} + (p-1)^{2p} \numSumSubset{2p}{J} &+ 
                (p-1)^p\numSumSubset{p-1}{J}\\
                + (p-1)^{2p}\numSumSubset{2p-1}{J}&\equiv 0
            \end{align*}
Simplifying the equation above we obtain
            \[1 - \numSumSubset{p - 1}{J} - \numSumSubset{p}{J} + \numSumSubset{2p-1}{J} + \numSumSubset{2p}{J}\equiv 0\]
\end{frame}
%------------------------------------------------

% %------------------------------------------------
% \begin{frame}
%     \frametitle{Proof of Corollary 3.2(3)}
    
% \end{frame}
% %------------------------------------------------

%------------------------------------------------
\begin{frame}
    \begin{cor}(3.3)\label{cor:corCountingJ3p}
        If $|J| = 3p-2$, or $|J| = 3p-1$, then $1 - (p|J) + (2p|J)\equiv 0$ 
    \end{cor}
\pause
    \begin{cor}(3.5)\label{cor:3pM13pM2_pJE0Imply2P}
        If $\abs{J} = 3p-2$ or $\abs{J} = 3p-1$, then $\numSumSubset{p}{J} = 0$ implies $\numSumSubset{2p}{J}\equiv -1$.
    \end{cor}
    \pause

    \begin{cor}(3.6)\label{cor:Exact3pZeroSumHasPzeroSeq}
        If $J$ contains exactly $3p$ elements, and $\sum_{x \in J} x \equiv 0$, then $\numSumSubset{p}{J} > 0$.
    \end{cor}
\end{frame}
%------------------------------------------------

%------------------------------------------------
\begin{frame}
    % \begin{cor}(3.6)\label{cor:Exact3pZeroSumHasPzeroSeq}
    %     If $J$ contains exactly $3p$ elements, and $\sum_{x \in J} x \equiv 0$, then $\numSumSubset{p}{J} > 0$.
    % \end{cor}
    \frametitle{Proof of Corollary 3.6}
    \begin{proofs}
        If $\numSumSubset{p}{J} = 0$ $ \Rightarrow $ 
        \[\forall x \in J, \quad\numSumSubset{p}{J-x} = 0.\]
        \pause
        $\abs{J-x} = 3p-1 \Rightarrow$
        \[\numSumSubset{2p}{J-x} \equiv -1.\]
        In particular
        \[\numSumSubset{2p}{J-x} > 0\]
        \pause
        $\forall A \subset J$, \sothat $\sum_{a \in A} a \equiv 0$,
        \begin{align*}
            \sum_{a\in A} a + \sum_{b\in J-A} b = \sum_{j \in J} j &\equiv 0\\
            \Rightarrow \sum_{b \in J-A} b &\equiv 0
        \end{align*}
       
    \end{proofs}
    
\end{frame}
%------------------------------------------------

%------------------------------------------------

\begin{frame}
    
    \begin{proofe}
        The map $T$
        \begin{align*}
        T: \set{A\subset J; \abs[]{A} = p, \sum_{a \in A} a \equiv 0}
        &\rightarrow \set{A\subset J; \abs[]{A} = 2p, \sum_{a\in A}a \equiv 0}\\
            A &\mapsto J - A
        \end{align*}
        is a bijection. It follows that:
        \begin{equation*}
            \numSumSubset{p}{J} = \numSumSubset{2p}{J} \geq \numSumSubset{2p}{J - x} > 0,
        \end{equation*}
        which is a contradiction to the assumption $\numSumSubset{p}{J} = 0$
    \end{proofe}
\end{frame}

%------------------------------------------------


%------------------------------------------------
\begin{frame}
    \begin{cor}(3.7)\label{cor:4pM3_twoEquations}
        If $\abs{X} = 4p-3$, then
        \begin{enumerate}
            \item $-1 + (p|X) - (2p|X) + (3p|X) \equiv 0$
            \item $(p-1|X) - (2p-1|X) + (3p-1|X) \equiv 0$
        \end{enumerate}
    \end{cor}
    \begin{cor}(3.8)\label{cor:4pM3_multiSum}
        If $\abs{X} = 4p-3$, then $3 - 2\numSumSubset{p-1}{X} - 2\numSumSubset{p}{X} + \numSumSubset{2p-1}{X} + \numSumSubset{2p}{X} \equiv 0$.
    \end{cor}
    
\end{frame}
%------------------------------------------------

%------------------------------------------------
\begin{frame}
    \begin{proofs}
        We deduce from Corollary 3.2 that:
        \[\sum_I 1 - \numSumSubset{p-1}{I} - \numSumSubset{p}{I} + \numSumSubset{2p-1}{I} + \numSumSubset{2p}{I} \equiv 0,\]
        where the sum is over $I\subset X$, s.t, $\abs{I} = 3p-3$. 
        % For a given subset $Y \subset X$,s.t $\abs{Y} = p$ and $\sum_{y\in Y} y \equiv 0$, we want to find out the number of pairs
        % $(Y,I)$, s.t $Y\subset I, \abs{I} = 3p-3$. \\
        % We could see that 
        \pause
        \[\abs{\{(Y,I)| Y\subset I, 
        \sothat \sum_{y\in Y} y \equiv 0, \;\abs{I} = 3p-3\}}
        = \binom{3p-3}{2p-3},\]
        % since once we have chosen $p$ elements $Y$, we need to further choose $\abs{I} - p = 3p-3 - p = 2p-3$ elements from
        % total $\abs{X} - p = 4p-3 -p = 3p-3$ elements.
        We consider the sum
        \[ \sum_{ \substack{Y \subset X,\\\sum_{y\in Y} y= 0}} \sum_{\substack{I,\sothat \\Y\subset I,\\\abs{I} = 3p-3}} 1\]
    \end{proofs}
\end{frame}
%------------------------------------------------

\begin{frame}
    \begin{proofc}
        \begin{align*}
            \binom{3p-3}{2p-3} \numSumSubset{p}{X} &= \sum_{ \substack{Y\subset X\\ \sum_{y\in Y} y = 0}} \binom{3p-3}{2p-3} &&= \sum_{ \substack{Y \subset X,\\\sum_{y\in Y} y= 0}} \sum_{\substack{I,\sothat \\Y\subset I,\\\abs{I} = 3p-3}} 1\\
                &=\sum_{\substack{I,\sothat\\\abs{I} = 3p-3}}\;\sum_{ \substack{Y \subset I,\\\sum_{y\in Y} y= 0}}  1
                &&= \sum_{I} \numSumSubset{p}{I}
        \end{align*}
            Performing similar calculation, we get
        \begin{align}
            \binom{4p-3}{3p-3} &- \binom{3p-2}{2p-2}\numSumSubset{p-1}{X} - \binom{3p-3}{2p-3}\numSumSubset{p}{X}\nonumber\\ 
            &+ \binom{2p-2}{p-2}\numSumSubset{2p-1}{X} + \binom{2p-3}{p-3}\numSumSubset{2p}{X} \equiv 0 \label{eqn:4p_3BinomModulo}               
        \end{align}
    \end{proofc}
\end{frame}

%-----------------------------------------------
\begin{frame}
    \begin{proofc}
        We finally prove that 
        \begin{equation}\label{eqn:binomModulo1}
            \binom{4p-3}{3p-3} \equiv 3, \binom{3p-2}{2p-2}\equiv 2,            
        \end{equation}
        because:
        \begin{align*}
            \binom{4p-3}{3p-3} &\equiv \frac{(4p-3) \cdots (4p-(p-1)) \cdot 3p \cdot (3p-1) \cdot (3p-2)}{p!}\\
            &\equiv \frac{(4p-3) \cdots (4p-(p-1)) \cdot 3 \cdot (3p-1) \cdot (3p-2)}{(p-1)!} \\
            &\equiv 3 \cdot \frac{(-3) \cdot (-4) \cdots (-(p-1)) \cdot(-1)\cdot(-2)}{(p-1)!}\\
            &\equiv 3 \cdot \frac{(p-1)!}{(p-1)!}\\
            &\equiv 3
        \end{align*}        
        Note that we have used the fact that $p$ is an odd prime, \sothat $(-1)^{p-1} \equiv 1$.
    \end{proofc}    
\end{frame}

%----------------------------------------------

%------------------------------------------------
\begin{frame}
    \begin{proofe}
        Similarly, one can prove that
        \begin{equation}\label{eqn:binomModulo2}
            \binom{3p-3}{2p-3}\equiv 2, \binom{2p-2}{p-2} \equiv 1, \binom{2p-3}{p-3}\equiv 1.
        \end{equation}
        Combining the modulo equivalence in~\eqref{eqn:binomModulo1} and~\eqref{eqn:binomModulo2}, \eqref{eqn:4p_3BinomModulo} can be simplified to
        \begin{equation}
            3 - 2\numSumSubset{p-1}{X} - 2\numSumSubset{p}{X} + \numSumSubset{2p-1}{X} + \numSumSubset{2p}{X} \equiv 0, 
        \end{equation}
        which is what we want to prove.
    \end{proofe}
\end{frame}

%------------------------------------------------

%------------------------------------------------
\begin{frame}
    \begin{lem}(3.9)\label{lem:4pM3_pX0_impliesPM1_3PM1}
        If $\abs{X} = 4p- 3$  and $\numSumSubset{p}{X} = 0$, then $\numSumSubset{p-1}{X} \equiv \numSumSubset{3p-1}{X}$.
    \end{lem}
\end{frame}
%------------------------------------------------

%------------------------------------------------
\begin{frame}
    \begin{proofs}
        We consider the partition of $X = A \cup B \cup C$, where 
        \[\abs{A} = p-1, \quad \abs{B} = p-2, \quad \abs{C} = 2p.\] and 
        \[\sum_{a \in A} a \equiv 0, \quad \sum_{b \in B} b \equiv \sum_{x \in X} x, \quad \sum_{c \in C} c \equiv 0\]
        Let $\chi$ denote the number of such partition. We use two ways to compute the number $\chi$, the first one fixes $A$ and 
        find out the possible set $C$:
        \[\chi \equiv \sum_{A} \numSumSubset{2p}{X-A} \equiv \sum_{A} -1 \equiv -\numSumSubset{p-1}{X},\]
        where we have used Corollary 3.5, for $J = X-A$, with $\abs{J} = 3p-2$ and the fact that 
        \[0 \leq \numSumSubset{p}{J} \leq \numSumSubset{p}{X} = 0,\]
        which leads to $\numSumSubset{p}{J} = 0$.
    \end{proofs}
\end{frame}
%------------------------------------------------

\begin{frame}
    \begin{proofc}
        Now by fixing $B$ and count the possible set $C$, we get:
        \[\chi \equiv \sum_{B} \numSumSubset{2p}{X-B} \stackrel{\circled{1}}{\equiv} \sum_{B} -1 
        \stackrel{\circled{2}}{\equiv} \sum_{X-B}-1 \stackrel{\circled{3}}{\equiv} -\numSumSubset{3p-1}{X}\]
        For the three equivalences, we have used the following facts:

            1:  We use the similar argumentation as before, since $\abs{X-B} = 3p-1$ and apply Corollary 3.5
            leads to $\numSumSubset{2p}{X-B} \equiv -1$.    
            
            2:
            Consider the two sets
            \begin{align*}
                S :&=\set{B\subset X; \abs[]{B} = p - 2, \sum_{b \in B} b \equiv \sum_{x \in X} x}\\
                W :&=\set{ J \subset X; \abs[]{J} = 3p-1, \sum_{j\in J}j \equiv 0}
                \end{align*}
    \end{proofc}
\end{frame}

% --------------------------------------
\begin{frame}
    \begin{proofe}
        The map $T$ defined by:
        \begin{align*}
            T: S &\rightarrow W\\
            B &\mapsto X - B
        \end{align*}
        is a bijection, \sothat 
        \[\sum_{B} 1 \equiv \sum_{X -B} 1\]

        3: Since $\sum_{b\in B} b \equiv \sum_{x \in X} x$, it follows that 
        \[\sum_{x \in X - B} x \equiv 0,\]
        in particular 
        \[\sum_{X-B} -1 \equiv -1 \cdot \numSumSubset{3p-1}{X}\]
    \end{proofe}
\end{frame}


%------------------------------------------------
\begin{frame}
    \frametitle{Proof of \kemnitzConjecture{}}
    \begin{proofs}
        \begin{align}
            -1 + (p|X) - (2p|X) + (3p|X) &\equiv 0\\
            (p-1|X) - (2p-1|X) + (3p-1|X) &\equiv 0\\
            3 - 2\numSumSubset{p-1}{X} - 2\numSumSubset{p}{X} + \numSumSubset{2p-1}{X} + \numSumSubset{2p}{X} &\equiv 0.
        \end{align}
        \pause
        Adding the three above equations, we obtain:
        \begin{equation}\label{eqn:kemnitzSum3Eqn}
            2 - \numSumSubset{p-1}{X} - \numSumSubset{p}{X} + \numSumSubset{3p-1}{X} + \numSumSubset{3p}{X} \equiv 0            
        \end{equation}
        Assume there is a set $X$, with $\abs{X} = 4p-3$ which contradicts the theorem, that is $\numSumSubset{p}{X} = 0$.
        \pause

        Using the previous Lemma 3.9, we obtain $\numSumSubset{p-1}{X} \equiv \numSumSubset{3p-1}{X}$.
        Then \eqref{eqn:kemnitzSum3Eqn} simplifies to
        \begin{equation}
            2 - \numSumSubset{p}{X} + \numSumSubset{3p}{X} \equiv 0
        \end{equation}

    \end{proofs}
\end{frame}
%------------------------------------------------

%------------------------------------------------
\begin{frame}
\begin{proofe}
    Since $p$ is odd, we see that $\numSumSubset{p}{X}$ and $\numSumSubset{3p}{X}$ could not both be $0$.
    Since we assume that $\numSumSubset{p}{X} = 0$, it follows that $\numSumSubset{3p}{X} > 0$, i.e., there is a subset $J \subset X$,
    $\abs{J} = 3p$ and $\sum_{j \in J} j \equiv 0$. 
    But from Corollary 3.6, we see that $\numSumSubset{p}{J} > 0$, in particular $\numSumSubset{p}{X} > 0$,
    which is a contradiction.
\end{proofe}
\end{frame}
%------------------------------------------------

\section{Generalization}
%------------------------------------------------
\begin{frame}
    \begin{thm}[\alonDubinerTheorem]\label{thm:AlonDubiner}
        $\exists c > 0$, \sothat $\forall n \in \NaturalNumber$, 
        \[\fnd{n}{d} < (cd \log_2 d)^d n\]
    \end{thm}
\end{frame}


%----------------------------------------------------------------------------------------
\begin{frame}
    \centering \Huge
    \emph{Thank You}
\end{frame}
% -----------------------------------


\end{document} 