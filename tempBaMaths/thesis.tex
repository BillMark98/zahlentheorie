\documentclass{article}
\usepackage{amssymb,latexsym,amsmath,amsthm,mathtools}
% for enumerate item
\usepackage{enumitem}
\theoremstyle{definition}
\newtheorem{theorem}{Theorem}[section]
\newtheorem{lemma}[theorem]{Lemma}
\newtheorem{proposition}[theorem]{Proposition}
\newtheorem{property}[theorem]{Property}
\newtheorem{definition}[theorem]{Definition}
\newtheorem{Algorithmus}{Algorithmus}
\newtheorem{example}[theorem]{Example}
\newtheorem{corollary}[theorem]{Corollary}
\newtheorem{prop}[theorem]{Proposition}
\newtheorem{remark}[theorem]{Remark}
% \newtheorem{property}{Property}
\numberwithin{equation}{theorem}
\numberwithin{figure}{theorem}


% ===============================
% For abs and set
\usepackage{mathtools,nccmath}%
\usepackage{ etoolbox, xparse} 

% ===============
% for contradiction symbol
% \usepackage{wasysym}  % \lightning
\usepackage{marvosym} %\Lightning
\usepackage{stmaryrd} % $\lightning

\DeclarePairedDelimiterX{\abs}[1]\lvert\rvert{\ifblank{#1}{\,\cdot\,}{#1}}

\let\oldabs\abs
\def\abs{\futurelet\testchar\MaybeOptArgAbs}
\def\MaybeOptArgAbs{\ifx[\testchar\let\next\OptArgAbs
\else \let\next\NoOptArgAbs\fi \next}
\def\OptArgAbs[#1]#2{\oldabs[#1]{#2}}
\def\NoOptArgAbs#1{\ifblank{#1}{\oldabs{}}{\oldabs[\big]{#1}}}

\def\Abs{\oldabs*}

\DeclarePairedDelimiterX{\set}[1]\{\}{\setargs{#1}}
\NewDocumentCommand{\setargs}{>{\SplitArgument{1}{;}}m}
{\setargsaux#1}
\NewDocumentCommand{\setargsaux}{mm}
{\IfNoValueTF{#2}{#1}{\nonscript\,#1\nonscript\;\delimsize\vert\nonscript\:\allowbreak #2\nonscript\,}}
%%% Syntaxe : \set{x ; P(x)})
\let\oldset\set
\def\set{\futurelet\testchar\MaybeOptArgSet}
\def\MaybeOptArgSet{\ifx[\testchar \let\next\OptArgSet
\else \let\next\NoOptArgSet \fi \next}
\def\OptArgSet[#1]#2{\oldset[#1]{#2}}
\def\NoOptArgSet#1{\OptArgSet[\big]{#1}}

\def\Set{\oldset*}

% =====================
% really wide hat
\usepackage{scalerel,stackengine}
\stackMath
\newcommand\reallywidehat[1]{%
\savestack{\tmpbox}{\stretchto{%
  \scaleto{%
    \scalerel*[\widthof{\ensuremath{#1}}]{\kern-.6pt\bigwedge\kern-.6pt}%
    {\rule[-\textheight/2]{1ex}{\textheight}}%WIDTH-LIMITED BIG WEDGE
  }{\textheight}% 
}{0.5ex}}%
\stackon[1pt]{#1}{\tmpbox}%
}
\parskip 1ex


% ==============================

% =========
% theroem
% \newcommand{\warningTheorem}{\ensuremath{\operatorname{\hbox{\emph{Chevalley-Warning Theorem}}}}}
% \newcommand{\warningTheorem}{\ensuremath{\textbf{Chevalley-Warning Theorem}}}
\newcommand{\warningTheorem}{\emph{Chevalley\,-Warning Theorem}}
\newcommand{\kemnitzConjecture}{\emph{Kemnitz' Conjecture}}
\newcommand{\alonDubinerTheorem}{\emph{Alon-Dubiner Theorem}}

% ========
% number ring, fields
\newcommand{\Field}[1]{\ensuremath{\mathbb{F}_{#1}}}
\newcommand{\IntegerP}[1]{\ensuremath{\mathbb{Z}_{#1}}}
\newcommand{\modularInteger}[1]{\ensuremath{\mathbb{Z}/#1\mathbb{Z}}}
\newcommand{\NaturalNumber}{\ensuremath{\mathbb{N}}}
\newcommand{\Real}{\ensuremath{\mathbb{R}}}
\newcommand{\Complex}{\ensuremath{\mathbb{C}}}
\newcommand{\ComplexUnit}{\ensuremath{\mathbb{C}^{*}}}

\newcommand{\Integer}{\ensuremath{\mathbb{Z}}}
\newcommand{\PolynomialRing}[2]{\ensuremath{#1[x_1,x_2,\ldots,x_{#2}]}}


% ================
% german alphabet
\newcommand{\oUmlaut}{{\"o}}
\newcommand{\uUmlaut}{{\"u}}
\newcommand{\aUmlaut}{{\"a}}

% ======
% notations
\newcommand{\zeroSumSeq}[1]{$0$-sum $#1$-subsequence}
\newcommand{\cayleyGraph}[2]{\ensuremath{\Gamma_{#1,#2}}}
\newcommand{\adjacencyMatrixCayley}[2]{\ensuremath{A_{(#1,#2)}}}
\newcommand{\characterGroup}[1][G]{\ensuremath{\reallywidehat{#1}}}
\DeclarePairedDelimiter{\roundCeil}\lceil\rceil
% =======
% some functions
\newcommand{\fnd}[2]{\ensuremath{f(#1,#2)}}
\newcommand{\bigO}[1]{\ensuremath{\mathcal{O}(#1)}}
\newcommand{\myAlphabetSubSupscript}[3]{\ensuremath{#1_{#2}^{#3}}}
\newcommand{\mySup}[1]{\sup{#1}}
\newcommand{\myInf}[1]{\inf{#1}}
\newcommand{\bilinearForm}[2]{\ensuremath{\langle#1,#2\rangle}}
\newcommand{\composition}[2]{\ensuremath{#1\circ#2}}
\newcommand{\TODO}{\textbf{!!!!!!! To Do !!!!!!!!!}}

% ===========
% environment
\newenvironment{case}
    {\begin{enumerate}[label = \textbf{Case }{\arabic* }:]}
        {\end{enumerate}}
\newenvironment{enumeratei}{\begin{enumerate}[label = (i)]}%
    {\end{enumerate}}

\newcommand{\WLOG}{w.l.o.g}
% \newcommand{\abs}[1]{\ensuremath{|#1|}}

% ======
% math symbol
\newcommand{\setCondition}{\mid}
\newcommand{\mySetMinus}{\setminus}
\newcommand{\minus}{-}
\newcommand{\myMin}[1]{\ensuremath{\min \{#1\}}}
\newcommand{\contradiction}{\ensuremath{\lightning}}
% \newcommand{\negative}{-}

\newcommand{\polyDeg}[1]{deg(\ensuremath{#1})}
\newcommand{\numSumSubset}[2]{\ensuremath{(#1|#2)}}
\newcommand{\circled}[1]{\ensuremath{#1}}


\title{Proof of Kemnitz' Conjecture and generalized to higher dimensions}
\date{\today}
\author{Panwei Hu}

% set the contents of tables depth 
% here 2 ----  subsection
\setcounter{tocdepth}{2}

% \renewcommand{\contentsname}{Inhalte}
% \renewcommand{\refname}{Literatur}

\begin{document}

\pagenumbering{gobble}
\maketitle
\newpage
\tableofcontents
  \newpage
\newpage
\pagenumbering{arabic}
\newpage

\section{Abstract}
In this thesis, I worked with a lattice point problem which deals with the smallest number $f(n,d)$
of a sequence of lattice point to guarantee the centroid is also a lattice point. The thesis first introduces
this concept and gives some examples in the low-dimensional case. The next chapter deals with the \kemnitzConjecture.
The thesis will state a detailed proof which is based on \cite{Reiher_2007}. The last chapter deals with the \alonDubinerTheorem{} which
gives a general upper bound for $f(n,d)$.

\newpage
    \section{Introduction}
    We define a $d$-dimensional affine space $V$ and consider the set of points which lie in the set
    \[V_d := \set{\sum_{i = 1}^d a_i v_i; a_i \in \Integer, \quad 1 \leq i \leq d},\]
    where $\set{v_i; 1 \leq i \leq d}$ are linear independent vectors in $V$. 
    We also call the point in $V_d$ as \emph{lattice} points.
    We now want to find out the minimum of the number $f$ so that given $f$ sequences in $V_d$, we can guarantee to find out a subsequence of length $n$, s.t. the centroid of this 
    subsequence is also a lattice point. We define such minimum number of $f(n,d)$.
    % The element of $G$ will be treated as a lattice point. We define the number $f(n,d)$ as the smallest number s.t. for all 
    % sequence of elements in $G$, with length greater or equal to $f(n,d)$, we can find a 

    

    Consider the additive group $G:= \IntegerP{n}^d$.
    We call a subsequence of length $n$, which 
    sums to a $0$ in $\IntegerP{n}^d$ as 
    \zeroSumSeq{n}, where $0$ denotes the zero vector
     in $G$.  
    It is not hard to see that the above problem 
    can be equivalently formulated as finding the 
    number $f(n,d)$, so that 
    given a sequence of elements in $G$, with 
    length $l \geq f(n,d)$, there exists a \zeroSumSeq{n}.

    A natural bound on the number $f(n,d)$ is given in the following Lemma.
    % \begin{lemma}
    %     \begin{alignat}{2}
    %         (n-1) 2^d + 1 &\leq &\fnd{n}{d} &\leq (n-1)n^d + 1 \label{eqn:fndProp1}\\
    %         \fnd{pq}{d} &\leq &\fnd{p}{d} &+ p(\fnd{q}{d} - 1) \label{eqn:fndProp2}
    %     \end{alignat}
    % \end{lemma}
    \begin{lemma}\label{lem:f_n_d_naturalBound}
        \begin{equation}\label{eqn:fndProp1}
            (n-1) 2^d + 1 \leq \fnd{n}{d} \leq (n-1)n^d + 1
        \end{equation}
    \end{lemma}
    \begin{proof}
        The left inequality can be seen as following:

        We construct $(n-1) 2^d$ vectors, which includes all the vectors in $\IntegerP{n}^d$, which has $0$ or $1$ in their entry, so there are in all $2^d$ different vectors.
        Each vector appear exactly $n-1$ times.  It is impossible to find a \zeroSumSeq{n} among these vectors. Thus we must have 
        \[(n-1) 2^d + 1 \leq \fnd{n}{d}.\]

        To prove the right inequality we use the pigeon hole principle. 
        Since $\abs{G} = n^d$, given $(n-1)n^d + 1$ elements, there are at least one vector $v$ which has 
        multiplicity
        \[\roundCeil{\frac{(n-1) n^d + 1}{n^d}} = n.\]
        Choosing these $n$ vectors $v$, we obtain a \zeroSumSeq{n}.
        We thus deduce that 
        \[\fnd{n}{d} \leq (n-1)n^d + 1.\]
    \end{proof}
    It is generally very difficult to examine $f(n,d)$ for every possible $n$, thus we would like to restrict $n$ to a smaller set of numbers, for example the prime number. The 
    following Lemma helps us with such reduction.
    \begin{lemma}
        \begin{equation}\label{eqn:fndProp2}
            \fnd{pq}{d} \leq \fnd{p}{d} + p(\fnd{q}{d} - 1) 
        \end{equation}
    \end{lemma}
    \begin{proof}
        For convenience of notation, we define 
        \[f_1 := \fnd{p}{d}, f_2 := \fnd{q}{d}, f := f_1 + p(f_2 - 1).\]
        Given a sequence $S$ of $f$ elements, 
        since $f > f_1$, we can find a \zeroSumSeq{p}, 
        which we denote as $A_1$.
        We then consider the $S \setminus A_1$, since 
        the rest is still larger than $f_1$ we can 
        further find a \zeroSumSeq{p}, which we denote 
        as $A_2$.
        We perform this procedure recursively. Finally, 
        since 
        \[f = f_1 + p (f_2 - 1),\]
        we would obtain $f_2$ \zeroSumSeq{p}s. According to the definition, each such subsequence $A_l$ sums to a vector $w_l$ of the form
        \begin{equation*}\label{eqn:fnd_pq_p_sumVec}
            w_l = p \cdot \begin{pmatrix}
            v_1\\
            \vdots\\
            v_d
        \end{pmatrix}, \quad v_i \in \Integer, \; 1 \leq i \leq d
        \end{equation*}
        Among all the $f_2$ $w_l$ vectors, there exist a \zeroSumSeq{q}, which means 
        they sum to a vector $z$, with
        \[z = q \cdot \begin{pmatrix}
            v_1\\
            \vdots\\
            v_d
        \end{pmatrix}, \quad v_i \in \Integer, \; 1 \leq i \leq d\]
        Since $z$ are the sum of vectors of the form $w_l$, we see that $z$ can also be written as
        \[
            z = pq \cdot \begin{pmatrix}
                \tilde{v}_1\\
                \vdots\\
                \tilde{v}_d
            \end{pmatrix}, \quad \tilde{v}_i \in \Integer, \; 1 \leq i \leq d\]
        Combining the results we obtain $pq = n$ sequences which sums to a vector $z$, whose every 
        entry is a multiple of $n$, thus we obtain a \zeroSumSeq{n}.
    \end{proof}
    \begin{remark}
        Due to the symmetry we would obtain similarly:
        \begin{align}\label{eqn:fndProp2_prime}
            \fnd{pq}{d} \leq \fnd{q}{d} + q(\fnd{p}{d} - 1)\tag{$\ref{eqn:fndProp2}^\prime$}
        \end{align}
        Combining \eqref{eqn:fndProp2} and \eqref{eqn:fndProp2_prime}, we obtain the following upper bound.
    \end{remark}
    \begin{corollary}\label{cor:fnd_pq_Bound}
        \begin{equation}\label{eqn:fnd_pq_Bound}
            \fnd{pq}{d} \leq \min \{\fnd{p}{d} + p(\fnd{q}{d} - 1),  \fnd{q}{d} + q(\fnd{p}{d} - 1)\}
        \end{equation}
    \end{corollary}
    
    We can use \eqref{eqn:fnd_pq_Bound} to find out some values of $f(n,d)$.
    Before we state the famous theorem first discovered by Erd\oUmlaut s ,Ginsburg and Ziv,
    we state the \emph{Cauchy-Davenport} Theorem which would help the proof.
    \begin{theorem}[Cauchy-Davenport]\label{theorem:cauchy_davenport}
        Let $p$ be a prime number. If $A, B \subset \IntegerP{p}$ are nonempty, then 
        \[\abs{A + B} \geq \myMin{p, \abs{A} + \abs{B} - 1},\]
        where $A+B$ denotes the set of the addition of elements in $A$ and $B$, see\\ Definition~\ref{def:sum_of_sets}.
    \end{theorem}
    \begin{proof}
        See for example \cite{nathanson1996additive}.
    \end{proof}
    \begin{theorem}[Erd\oUmlaut s-Ginsburg-Ziv]\label{theorem:Erdos2NM1}
        \begin{equation*}
            \fnd{n}{1} = 2n - 1  
        \end{equation*}
    \end{theorem}  
    \begin{proof}
        From \eqref{eqn:fndProp1}, we obtain 
        \begin{equation}\label{ieqn:fnd_n_1_get_2n_1}
            \fnd{n}{1} \geq 2 n - 1
        \end{equation}            
        So we only need to prove that 
        \begin{equation}\label{ieqn:fnd_n_1_leq_2n_1}
            \fnd{n}{1} \leq 2n - 1.
        \end{equation}            
        From \eqref{eqn:fnd_pq_Bound}, we could restrict to the case when $n$ is prime, 
        because if we have proved the theorem for all 
        the prime numbers,
        then for two given prime numbers $p,q$:
        \begin{equation}\label{eqn:erdos_fnd_pq_inequality}
        \begin{aligned}[b]
            \fnd{pq}{1} &\leq \fnd{p}{1} + p (\fnd{q}{1} - 1) \\
            &\leq 2 p - 1 + p (2 q - 2) \\
            &= 2pq - 1         
        \end{aligned}
        \end{equation}
        Applying \eqref{eqn:erdos_fnd_pq_inequality} recursively 
        to the prime factorization of any integer 
        $n$ leads to \eqref{ieqn:fnd_n_1_leq_2n_1}.
    
        We thus prove \eqref{ieqn:fnd_n_1_leq_2n_1} for the case $p$, $p$ a prime number.
        Assume there is a sequence with $2p - 1$ elements, if there are already $p$ identical elemnent, we are done.
        So we could assume that there are no elements which has multiplicity larger than $p-1$.
        We arrange all the element increasingly (which is possible in the $1$ dimensional case), and enumerate the element so that
        \[a_1 \leq a_2 \leq \cdots \leq a_{2p-1}.\]
        We define the set $A_i := \set{a_i, a_{i + p - 1}}, 1 \leq i \leq p-1$. Note that $\abs{A_i} = 2, \forall 1 \leq i \leq p-1$, because 
        we assume that no elements appear more than $p - 1$  times.
        Apply Theorem~\ref{theorem:cauchy_davenport} recursively, we obtain
        \begin{align*}
            \abs{A_1 + \cdots + A_{p-1}} &\geq \myMin{p, \abs{A_2 + \cdots + A_{p-1}} + \abs{A_1} - 1} \\
            &\geq \myMin{p, \abs{A_2 + \cdots + A_{p-1}} + 1} \\
            &\geq \myMin{p, \abs{A_3 + \cdots + A_{p-1}} + 2}\\
            &\vdots \\
            &\geq \myMin{p, \abs{A_{p-2}, A_{p-1}} + p - 3}\\
            &\geq \myMin{p, 2 + 2 - 1 + p - 3}\\
            &\geq \myMin{p,p} = p.
        \end{align*}
        This implies that 
        \[A_1 + \cdots + A_{p-1} = \IntegerP{p}\]
        In particular we can pick $p-1$ elements from the first $2p-2$ elements $a_{i_1},\ldots, a_{i_{p-1}}$, with $a_{i_k} \in A_k, 1 \leq k \leq p - 1$.
        s.t.
        \[a_{i_1} + \cdots + a_{i_{p-1}} = - a_{2p-1}.\]
        Moving $a_{2p-1}$ to the left handside of the equation, we thus obtain a \zeroSumSeq{p} $a_{i_1},\ldots, a_{i_{p-1}}, a_{2p-1}$, which means
        \[\fnd{p}{1} \leq 2p - 1\]
    \end{proof}

    \begin{lemma}\label{lem:f_nd_2_potent}
        \begin{equation*}
            \fnd{2^n}{d} = (2^n - 1) 2^d + 1
        \end{equation*}
    \end{lemma}
    \begin{proof}
        We prove the Lemma through induction on $n$.
        For $n = 1$, we know from \eqref{eqn:fndProp1} that
        \begin{equation*}
            (2 - 1)2^d + 1 \leq \fnd{2}{d} \leq (2 - 1)2^d + 1
        \end{equation*}
        Thus $\fnd{2}{d} = 2^d + 1$

        For the general case $n$, we have $f(2^{n-1},d) = (2^{n-1} - 1)2^d + 1$ from induction.
        Using \eqref{eqn:fnd_pq_Bound}, we obtain:
        \begin{align*}
            \fnd{2^n}{d} = \fnd{2 \cdot 2^{n-1}}{d} &\leq \fnd{2^{n-1}}{d} + 2^{n-1} \cdot (\fnd{2}{d} - 1)\\
            &= (2^{n-1} - 1)2^d + 1 + 2^{n-1} \cdot 2^d\\
            &= (2^n - 1) 2^d + 1
        \end{align*}
    \end{proof}
    It is proved in \cite{Harborth1973} that the following holds
    \begin{lemma}\label{lem:f_3_2_Value_Harborth}
        \begin{equation*}
            \fnd{3}{2} = 9
        \end{equation*}
    \end{lemma}
    Applying the trick as in Lemma~\ref{lem:f_nd_2_potent}, using the base case in Lemma~\ref{lem:f_3_2_Value_Harborth}, we obtain 
    \begin{lemma}\label{lem:f_3Potent_2_value}
        \begin{equation*}
            \fnd{3^n}{2} = 4 \cdot 3^n - 2
        \end{equation*}
    \end{lemma}

    By setting $d = 2$ in Lemma~\ref{lem:f_nd_2_potent}, we see that
    \[\fnd{2^n}{2} = 4 \cdot 2^n - 2\]
    we see that there is a simlar structure as in $\fnd{3^n}{2}$. 
    A natural question emerges that if this is the case for all $n \in \Integer$, that is
    \[\fnd{n}{2} = 4n - 3\]
    This is the well-known \kemnitzConjecture{}, which we will discuss in the next chapter.
\newpage
    \section{Proof of Kemnitz Conjecture}
    For the following section, I use the notation $\equiv$ to denote the modulo in $\IntegerP{p}$.\\
    The $0$ denotes the usual neutral element of addition in the corresponding abelian group. In particular,
    $0$ denotes the standard $0$ in the abelian group $\IntegerP{p}$ and $(0,0)$ in the case of $\IntegerP{p}\times \IntegerP{p}$.
    \begin{theorem}[\warningTheorem]\label{theorem:warning}
        Let $p$ be a prime number and $q = p^t$. We use $\Field{q}$ to denote the finite field of $q$ elements.
        Let $p_1,\ldots,p_m$ be $m$ polynomials in $\PolynomialRing{\Field{q}}{n}$, with degree $d_i$. Denote the number of common
        zeros of the $m$ polynomials as $N$.
        If 
        \[\sum_{i = 1}^{m} d_i < n,\]
        then 
        \[N \equiv 0 \pmod{p}\]
    \end{theorem}
    \begin{proof}
        See for example \cite{nathanson1996additive}.
    \end{proof}
    We would like to give a proof of the \kemnitzConjecture{}. First we note that from if $\fnd{p}{2} = 4p-3$ and $\fnd{q}{2} = 4q-3$,
    then applying \eqref{eqn:fnd_pq_Bound}
    \begin{align*}
        \fnd{pq}{2} &\leq \fnd{p}{2} + p (\fnd{q}{2} - 1)\\
        &\leq 4p-3 + p (4q - 3 - 1)\\
        &= 4pq - 3.
    \end{align*}
    Thus it suffices to consider the case when $p$ is prime. In addition from Lemma~\ref{lem:f_nd_2_potent}, we already know that
    \[\fnd{2}{2} = 4 \cdot 2 - 3 = 5,\]
    it suffices to consider the odd prime number. Thus we assume in the following paragraph that $p$ is an odd prime number. We denote $J,X$ and other capital alphabet
    as multiset of $\IntegerP{p} \times \IntegerP{p}$.
    We use the notation $\numSumSubset{p}{J}$ to denote the number of the \zeroSumSeq{p} in $J$.
    First we obtain the following Corollary from Theorem~\ref{theorem:warning}.
    \begin{corollary}\label{cor:origCountingJ3p}
        If $\abs{J} = 3p-3$, then $1 - \numSumSubset{p - 1}{J} - \numSumSubset{p}{J} + \numSumSubset{2p-1}{J} + \numSumSubset{2p}{J}\equiv 0$
    \end{corollary}
    \begin{proof}
        We specify $J$ as the set $\set{(a_n,b_n); 1 \leq n \leq 3p-3}$.\\
        We consider three polynomials
        \[p_1 := \sum_{n=1}^{3p-3} x_n^{p-1} + x_{3p-2}^{p-1}, \quad p_2 := \sum_{n=1}^{3p-3}a_n x_n^{p-1} \quad \hbox{and }  p_3:= \sum_{n = 1}^{3p-3}b_n x_n^{p-1}\]
        Since the total sum of degress are $3(p-1) = 3p-3 < 3p-2$, we can apply the \warningTheorem{}.
        Since three polynomials has $0$ as a common zero. The \warningTheorem{} states that there are non-trivial common zeros.
        We consider the common zeros depending on the term $x_{3p-2}$ in $p_1$.
        \begin{case}
            \item $x_{3n-2} = 0$\\
            In this case for the zero of $p_1$, it needs to satisfy
            \[\sum_{n = 1}^{3p-3} x_{n}^{p-1} \equiv 0\]
            We know that 
            \begin{equation*}
                x^{p-1} \equiv 
                \begin{cases}
                    1, &\text{if $x \not\equiv 0$;}\\
                    0, &\text{if $x \equiv 0$.}
                \end{cases}
            \end{equation*}
            Since there are in all $3p-3$ entries, there could only be three cases:
            \begin{enumeratei}
                \item $0$ of them are $1$
                This corresponds to the $0$, since all $x_i$ has to be $0$.
                \item $p$ of them are $1$
                This corresponds to $p$ $x_i$ which is non-zero. We enumerate them as $x_{i_1},\ldots,x_{i_p}$.
                To be the common zero, then this assignment nees to satisfy
                \[\sum_{j = 1}^{j = p} a_{i_j} \equiv 0, \quad \sum_{j = 1}^{j = p} b_{i_j} \equiv 0.\]
                The total number the set $\{i_1,\ldots, i_p \}$ is exactly $\numSumSubset{p}{J}$.
                Combining the fact that there are $p-1$ possible value for each $x_i$ s.t $x_i^{p-1} \equiv 1$,
                the total number of common zeros with $p$ non-zero entries is $(p-1)^p \numSumSubset{p}{J}$.
                \item $2p$ of them are $1$
                Arguing similarly as in the second case, we obtain the total number of common zeros with $2p$ non-zero entries 
                is given by $(p-1)^{2p}\numSumSubset{2p}{J}$.
            \end{enumeratei}
            \item $x_{3n-2} \neq 0$\\
            Since $x_{3n-2} \neq 0$, then $x_{3n-2}^{p-1} \equiv 1$. The other terms of $p_1$ must thus satisfy
            \[\sum_{n = 1}^{3p-3} x_{n}^{p-1} \equiv p-1.\]
            There are two possible cases for the other entries of $x$.
            \begin{enumeratei}
                \item $p-1$ of them are $1$.
                This corresponds to the case where there needs to be a $p-1$ elements in $J$ which form a zero-sum.
                There are in all 
                \[(p-1) \cdot (p-1)^{p-1} \cdot \numSumSubset{p-1}{J} = (p-1)^p \numSumSubset{p-1}{J},\]
                possible choices for $x$, where the first term corresponds to the choice of $x_{3n-2}$, the second to the rest of $x_i$.
                \item $2p-1$ of them are $1$.
                Similarly, the number of possible common zeros in this case is given by
                \[(p-1)^{2p} \numSumSubset{2p-1}{J}\]
            \end{enumeratei}
        \end{case}
            Collecting all the number of common zeros considered in different cases, we obtain, under the usage of \warningTheorem{}
            \begin{equation*}
                1 + (p-1)^p\numSumSubset{p}{J} + (p-1)^{2p} \numSumSubset{2p}{J} + (p-1)^p\numSumSubset{p-1}{J}
                + (p-1)^{2p}\numSumSubset{2p-1}{J}\equiv 0
            \end{equation*}
            Simplifying the equation above we obtain
            \[1 - \numSumSubset{p - 1}{J} - \numSumSubset{p}{J} + \numSumSubset{2p-1}{J} + \numSumSubset{2p}{J}\equiv 0\]
    \end{proof}
    \begin{corollary}\label{cor:corCountingJ3p}
        If $|J| = 3p-2$, or $|J| = 3p-1$, then $1 - (p|J) + (2p|J)\equiv 0$ 
    \end{corollary}
    \begin{proof}
        Consider the polynomials in $\PolynomialRing{\Field{p}}{|J|}$,
        \WLOG, $|J| = 3p-2$, the case for $\abs{J} = 3p-1$ is analogous.

        \[\sum_{n=1}^{3p-3} x_n^{p-1}, \quad\sum_{n=1}^{3p-3}a_n x_n^{p-1} \hbox{and} \sum_{n = 1}^{3p-3}b_n x_n^{p-1}\]
        since the total degrees are $3p-3 < 3p-2$, the \warningTheorem{} is applicable.
        Their common zeros are counted as in Corollary~\ref{cor:origCountingJ3p},
        we obtain:
        \begin{alignat}{2}
            &\phantom{\Rightarrow}\quad\quad\quad&1 + (p-1)^p (p|J) + (p-1)^{2p}(2p-1|J) &\equiv 0 \\
            &\Rightarrow &1 - (p|J) + (2p|J) &\equiv 0
        \end{alignat}
    \end{proof}
    \begin{remark}
        Note that in the case where $\abs{J} = 3p-1$ or $3p-2$, the three polynomials we examined
        all have the same number of monomials, i.e. $\abs{J}$, diffrent from in the case in $\abs{J} = 3p-3$. This
        is due to the fact the total degrees of the three polynomials (denoted as $f_i, i \in \{1,2,3\}$) natuarlly satisfy $\sum_{i} \polyDeg{f_i} < \abs{J}$
    \end{remark}

    \begin{corollary}\label{cor:3pM13pM2_pJE0Imply2P}
        If $\abs{J} = 3p-2$ or $\abs{J} = 3p-1$, then $\numSumSubset{p}{J} = 0$ implies $\numSumSubset{2p}{J}\equiv -1$.
    \end{corollary}
    \begin{proof}
        This follows directly from Corollary~\ref{cor:corCountingJ3p} by setting $\numSumSubset{p}{J} = 0$.
    \end{proof}
    \begin{corollary}\label{cor:Exact3pZeroSumHasPzeroSeq}
        If $J$ contains exactly $3p$ elements, and $\sum_{x \in J} x \equiv 0$, then $\numSumSubset{p}{J} > 0$.
    \end{corollary}
    \begin{proof}
        Assume that $\numSumSubset{p}{J} = 0$, this means that there are no subsets in $J$, of cardinality $p$ which 
        builds a zero-sum sequence. In particular, if we randomly pick an $x \in J$ and consider the set $J-x$, it follows that
        $\numSumSubset{p}{J-x} = 0$. Now, since $\abs{J-x} = 3p-1$, we can apply Corollary~\ref{cor:3pM13pM2_pJE0Imply2P} to obtain that 
        $\numSumSubset{2p}{J-x} \equiv -1$, in particular, this implies that $\numSumSubset{2p}{J-x} > 0$. \\
        We observe that $\forall A \subset J$, s.t. $\sum_{a \in A} a \equiv 0$,
        \begin{align*}
            \sum_{a\in A} a + \sum_{b\in J-A} b = \sum_{j \in J} j &\equiv 0\\
            \Rightarrow \sum_{b \in J-A} b &\equiv 0
        \end{align*}
        We could see that the map $T$
        \begin{align*}
        T: \set{A\subset J; \abs[]{A} = p, \sum_{a \in A} a \equiv 0}
        &\rightarrow \set{A\subset J; \abs[]{A} = 2p, \sum_{a\in A}a \equiv 0}\\
            A &\mapsto J - A
        \end{align*}
        is a bijection. It follows that:
        \begin{equation*}
            \numSumSubset{p}{J} = \numSumSubset{2p}{J} \geq \numSumSubset{2p}{J - x} > 0,
        \end{equation*}
        which is a contradiction to the assumption $\numSumSubset{p}{J} = 0$
    \end{proof}
    \begin{corollary}\label{cor:4pM3_twoEquations}
        If $\abs{X} = 4p-3$, then
        \begin{enumerate}
            \item $-1 + (p|X) - (2p|X) + (3p|X) \equiv 0$
            \item $(p-1|X) - (2p-1|X) + (3p-1|X) \equiv 0$
        \end{enumerate}
    \end{corollary}
    \begin{proof}
        Consider the polynomials $f_1,f_2,f_3$
        \[\sum_{n = 1}^{4p-3}x_n^{p-1}, \quad \sum_{n=1}^{4p-3}a_nx_n^{p-1} \hbox{and}\sum_{n=1}^{4p-3}b_n x_n^{p-1}\]
        The polynomials satisfy $\deg_i \polyDeg{f_i} < 4p-3$, so the \warningTheorem{} is applicable.
        By doing the analogous as in Corollary~\ref{cor:origCountingJ3p}, we obtain the first equation.
        
    \end{proof}
    \begin{corollary}\label{cor:4pM3_multiSum}
        If $\abs{X} = 4p-3$, then $3 - 2\numSumSubset{p-1}{X} - 2\numSumSubset{p}{X} + \numSumSubset{2p-1}{X} + \numSumSubset{2p}{X} \equiv 0$.
    \end{corollary}

    \begin{proof}
        We deduce from Corollary~\ref{cor:origCountingJ3p} that:
        \[\sum_I 1 - \numSumSubset{p-1}{I} - \numSumSubset{p}{I} + \numSumSubset{2p-1}{I} + \numSumSubset{2p}{I} \equiv 0,\]
        where the sum is over $I\subset X$, s.t, $\abs{I} = 3p-3$. 
        For a given subset $Y \subset X$,s.t $\abs{Y} = p$ and $\sum_{y\in Y} y \equiv 0$, we want to find out the number of pairs
        $(Y,I)$, s.t $Y\subset I, \abs{I} = 3p-3$. \\We could see that 
        \[\abs{\{(Y,I)| Y\subset I, \abs{I} = 3p-3}\} = \binom{3p-3}{2p-3},\]
        since once we have chosen $p$ elements $Y$, we need to further choose $\abs{I} - p = 3p-3 - p = 2p-3$ elements from
        total $\abs{X} - p = 4p-3 -p = 3p-3$ elements.
        Note that
        \[ \sum_{ \substack{Y \subset X,\\\sum_{y\in Y} y= 0}} \sum_{\substack{I,s.t. \\Y\subset I,\\\abs{I} = 3p-3}} 1\]
        \begin{align*}
            \binom{3p-3}{2p-3} \numSumSubset{p}{X} = \sum_{ \substack{Y\subset X\\ \sum_{y\in Y} y = 0}} \binom{3p-3}{2p-3} &= \sum_{ \substack{Y \subset X,\\\sum_{y\in Y} y= 0}} \sum_{\substack{I,s.t. \\Y\subset I,\\\abs{I} = 3p-3}} 1\\
                &=\sum_{\substack{I,s.t.\\\abs{I} = 3p-3}}\;\sum_{ \substack{Y \subset I,\\\sum_{y\in Y} y= 0}}  1 \\
                &= \sum_{I} \numSumSubset{p}{I}
        \end{align*}
        Performing similar calculation, we get
        \begin{align}
            \binom{4p-3}{3p-3} &- \binom{3p-2}{2p-2}\numSumSubset{p-1}{X} - \binom{3p-3}{2p-3}\numSumSubset{p}{X}\nonumber\\ 
            &+ \binom{2p-2}{p-2}\numSumSubset{2p-1}{X} + \binom{2p-3}{p-3}\numSumSubset{2p}{X} \equiv 0 \label{eqn:4p_3BinomModulo}               
        \end{align}
        We finally prove that 
        \begin{equation}\label{eqn:binomModulo1}
            \binom{4p-3}{3p-3} \equiv 3, \binom{3p-2}{2p-2}\equiv 2,            
        \end{equation}
        because:
        \begin{align*}
            \binom{4p-3}{3p-3} &\equiv \frac{(4p-3) \cdots (4p-(p-1)) \cdot 3p \cdot (3p-1) \cdot (3p-2)}{p!}\\
            &\equiv \frac{(4p-3) \cdots (4p-(p-1)) \cdot 3 \cdot (3p-1) \cdot (3p-2)}{(p-1)!} \\
            &\equiv 3 \cdot \frac{(-3) \cdot (-4) \cdots (-(p-1)) \cdot(-1)\cdot(-2)}{(p-1)!}\\
            &\equiv 3 \cdot \frac{(p-1)!}{(p-1)!}\\
            &\equiv 3
        \end{align*}        
        and
        \begin{align*}
            \binom{3p-2}{2p-2} &\equiv \frac{(3p-2) \cdots (3p-(p-1)) \cdot 2p \cdot (2p-1)}{p!}\\
            &\equiv \frac{(3p-2) \cdots (3p-(p-1)) \cdot 2 \cdot (2p-1) \cdot (2p-2)}{(p-1)!} \\
            &\equiv 2 \cdot \frac{(-2) \cdot (-4) \cdots (-(p-1)) \cdot(-1)}{(p-1)!}\\
            &\equiv 2 \cdot \frac{(p-1)!}{(p-1)!}\\
            &\equiv 2
        \end{align*}
        Note that we have used the fact that $p$ is an odd prime, so that $(-1)^{p-1} \equiv 1$.
        Similarly, one can prove that
        \begin{equation}\label{eqn:binomModulo2}
            \binom{3p-3}{2p-3}\equiv 2, \binom{2p-2}{p-2} \equiv 1, \binom{2p-3}{p-3}\equiv 1.
        \end{equation}
        Combining the modulo equivalence in~\eqref{eqn:binomModulo1} and~\eqref{eqn:binomModulo2}, \eqref{eqn:4p_3BinomModulo} can be simplified to
        \begin{equation}
            3 - 2\numSumSubset{p-1}{X} - 2\numSumSubset{p}{X} + \numSumSubset{2p-1}{X} + \numSumSubset{2p}{X} \equiv 0, 
        \end{equation}
        which is what we want to prove.
        \end{proof}

    \begin{lemma}\label{lem:4pM3_pX0_impliesPM1_3PM1}
        If $\abs{X} = 4p- 3$  and $\numSumSubset{p}{X} = 0$, then $\numSumSubset{p-1}{X} \equiv \numSumSubset{3p-1}{X}$.
    \end{lemma}
    \begin{proof}
        We consider the partition of $X = A \cup B \cup C$, where 
        \[\abs{A} = p-1, \quad \abs{B} = p-2, \quad \abs{C} = 2p.\] and 
        \[\sum_{a \in A} a \equiv 0, \quad \sum_{b \in B} b \equiv \sum_{x \in X} x, \quad \sum_{c \in C} c \equiv 0\]
        Let $\chi$ denote the number of such partition. We use two ways to compute the number $\chi$, the first one fixes $A$ and 
        find out the possible set $C$:
        \[\chi \equiv \sum_{A} \numSumSubset{2p}{X-A} \equiv \sum_{A} -1 \equiv -\numSumSubset{p-1}{X},\]
        where we have used Corollary~\ref{cor:3pM13pM2_pJE0Imply2P}, for $J = X-A$, with $\abs{J} = 3p-2$ and the fact that 
        \[0 \leq \numSumSubset{p}{J} \leq \numSumSubset{p}{X} = 0,\]
        which leads to $\numSumSubset{p}{J} = 0$.
        Now by fixing $B$ and count the possible set $C$, we get:
        \[\chi \equiv \sum_{B} \numSumSubset{2p}{X-B} \stackrel{\circled{1}}{\equiv} \sum_{B} -1 
        \stackrel{\circled{2}}{\equiv} \sum_{X-B}-1 \stackrel{\circled{3}}{\equiv} -\numSumSubset{3p-1}{X}\]
        \begin{enumerate}
            \item  We use the similar argumentation as before, since $\abs{X-B} = 3p-1$ and apply Corollary~\ref{cor:3pM13pM2_pJE0Imply2P}
            leads to $\numSumSubset{2p}{X-B} \equiv -1$.    
            \item The map T defined by:
            \begin{align*}
                T: \set{B\subset X; \abs[]{B} = p - 1, \sum_{b \in B} a \equiv \sum_{x \in X} x}
                &\rightarrow \set{ J \subset X; \abs[]{J} = 3p-1, \sum_{j\in J}j \equiv 0}\\
                    X &\mapsto X - B
            \end{align*}
            is a bijection, so that 
            \[\sum_{B} 1 \equiv \sum_{X -B} 1\]
            \item Since $\sum_{b\in B} b \equiv \sum_{x \in X} x$, it follows that 
            \[\sum_{x \in X - B} x \equiv 0,\]
            in particular 
            \[\sum_{X-B} -1 \equiv -1 \cdot \numSumSubset{3p-1}{X}\]
        \end{enumerate}
        Now we are able to prove the \kemnitzConjecture:        
    \end{proof}
    \begin{theorem}
        Any choice of $4p-3$ lattice points in the plane $\Integer \times \Integer$ contains
        a subset of cardinality $p$ whose centroid is a lattice point.
    \end{theorem}
    \begin{proof}
        For the proof of the theorem, we recall the equations we obatined from Corollary~\ref{cor:4pM3_twoEquations} and Corollary~\ref{cor:4pM3_multiSum}:
        \begin{align}
            -1 + (p|X) - (2p|X) + (3p|X) &\equiv 0\\
            (p-1|X) - (2p-1|X) + (3p-1|X) &\equiv 0\\
            3 - 2\numSumSubset{p-1}{X} - 2\numSumSubset{p}{X} + \numSumSubset{2p-1}{X} + \numSumSubset{2p}{X} &\equiv 0.
        \end{align}
        Adding the three above equations, we obtain:
        \begin{equation}\label{eqn:kemnitzSum3Eqn}
            2 - \numSumSubset{p-1}{X} - \numSumSubset{p}{X} + \numSumSubset{3p-1}{X} + \numSumSubset{3p}{X} \equiv 0            
        \end{equation}
        Assume there is a set $X$, with $\abs{X} = 4p-3$ which contradicts the theorem, that is $\numSumSubset{p}{X} = 0$.
        Using the previous Lemma~\ref{lem:4pM3_pX0_impliesPM1_3PM1}, we obtain $\numSumSubset{p-1}{X} \equiv \numSumSubset{3p-1}{X}$
        Then \eqref{eqn:kemnitzSum3Eqn} simplifies to
        \begin{equation}
            2 - \numSumSubset{p}{X} + \numSumSubset{3p}{X} \equiv 0
        \end{equation}
        Since $p$ is odd, we see that $\numSumSubset{p}{X}$ and $\numSumSubset{3p}{X}$ could not both be $0$.
        Since we assume that $\numSumSubset{p}{X} = 0$, it follows that $\numSumSubset{3p}{X} > 0$, i.e., there is a subset $J \subset X$,
        $\abs{J} = 3p$ and $\sum_{j \in J} j \equiv 0$. 
        But from Corollary~\ref{cor:Exact3pZeroSumHasPzeroSeq}, we see that $\numSumSubset{p}{J} > 0$, in particular $\numSumSubset{p}{X} > 0$,
        which is a contradiction.
    \end{proof}
\newpage
    \section{Alon-Dubiner Theorem}
    In this chapter, I would like to state the proof the \alonDubinerTheorem{} which gives a general upper bound
    of the number $\fnd{n}{d}$.
    \subsection{Preliminaries}


    \begin{definition}
        We define a \emph{hyperplane} of $\IntegerP{p}^d$ to be the set of all vectors $v \in \IntegerP{p}^d$ so that
        $\exists u \in \IntegerP{p}^d$, s.t. $ u\cdot v = c$ for a constant $c \in \IntegerP{p}$. where $\cdot$ denotes the standard
        scalar product. An \emph{affine basis} is a set $V \subset \IntegerP{p}^d$ with $\abs{V} = d+1$, s.t. $V$ is not contained in a hyperplane.
    \end{definition}
        
    \subsection{Some inequalities from additive combinatorics}
    \begin{definition}\label{def:sum_of_sets}
        Let $A$ and $B$ two subsets of a finite abelian group $G$, define
        the addition and subtraction of the two sets as following:
        \begin{align*}
            A + B &:= \set{ a + b ; a \in A, b \in B }\\
            A \minus B &:= \set{ a \minus b ; a \in A, b \in B }        
        \end{align*}
        \end{definition}
    \begin{definition}
        Let A be a subset of a finite abelian group $G$, $g \in G$, the
        addition and subtraction of the element $g$ and subset $A$ is defined as following:
        \begin{align*}
            g + A &:= \set{ g + a ; a \in A }\\
            g \minus A &:= \set{ g \minus a ; a \in A }
        \end{align*}

    \end{definition}

    \begin{lemma}\label{lem:A_B_SumToG}
        Let G be a finite abelian group, A, B are two subsets of G,
        satisfying
        \[\abs{A} > \frac{\abs{G}}{2}, \abs{B} > \frac{\abs{G}}{2}\]
        Then $A + B = G$
    \end{lemma}
    \begin{proof}
        We randomly pick an element $g \in G$. Since $\abs{g-A} = \abs{A} > \frac{\abs{G}}{2}$
        , and $\abs{B} > \frac{\abs{G}}{2}$
        . We must have
        \[B \cap g \minus A \neq \emptyset\]
        in particular, $\exists a \in A, b \in B$, s.t. $g = a + b$, since $g$ is randomly picked, we
        obtain $A + B = G$        
    \end{proof}    

    \begin{lemma}\label{lem:plunneckeRuska_set_LowerBound}
        If $Y$ and $B$ are finite subsets of an abelian group and $h \in \NaturalNumber$, then
        \[\abs{Y + B} \geq \abs{Y} (\frac{\abs{hB}}{\abs{Y}})^{\frac{1}{h}}\]
    \end{lemma}
    \begin{corollary}\label{cor:YplusB_lowerBound}
        Let $Y$ be a subset of $\IntegerP{p}^d$ and let $B$ be an affine basis of $\IntegerP{p}^d$. If $\abs{Y} \leq x^d$ 
        for some integer $x \leq \frac{p}{4d}$ then $\abs{Y + B} \geq \abs{Y}2^{\frac{1}{2x}}$
    \end{corollary}

    \begin{proposition}\label{prop:A_sets_sum_Cardinality_lowerBound}
        \[\abs{A_1 + \ldots + A_s} \geq x^d.\]
    \end{proposition}
    \begin{proof}
        We start with $Y_1 := {0}$ and set $x_1 = 2$. We apply Corollary~\ref{cor:YplusB_lowerBound} to the set
        $Y = Y_1, B := A_1$, then it follows that $Y + B = Y_1 + A_1 = A_1$. Besides,
        \[\abs{Y_1 + A_1} \geq \abs{Y_1}2^{\frac{1}{2x_1}}\geq 2^{\frac{1}{2x_1}}\]
        We then update $B$ to $A_2$ , $Y$ to $Y + B$ and do the same
        procedure recursively, thus
        \begin{alignat*}{2}
            \abs{A_1 + A_2} &\geq \abs{A_1} \cdot 2^{\frac{1}{2x_1}} \geq 2^{\frac{2}{2x_1}},& &Y \leftarrow A_1 + A_2\\
            \abs{A_1 + A_2 + A_3} &\geq \abs{A_1 + A_2} \cdot 2^{\frac{1}{2x_1}} \geq 2^{\frac{3}{2x_1}}, &&Y \leftarrow A_1 + A_2 + A_3\\
            \phantom{\abs{A_1 + \ldots + A_m}} & \phantom{\geq \abs{A_1 +\ldots +  A_{m-1}} \cdot 2^{\frac{1}{2x_1}} \geq 2^{\frac{m}{2x_1}},}& \vdots &\phantom{Y \leftarrow A_1 + \ldots + A_m}\\
            \abs{A_1 + \ldots + A_m} &\geq \abs{A_1 +\ldots +  A_{m-1}} \cdot 2^{\frac{1}{2x_1}} \geq 2^{\frac{m}{2x_1}},& \quad &Y \leftarrow A_1 + \ldots + A_m
        \end{alignat*}
        Because 
        \[\frac{m}{2x_1} = d \Leftrightarrow m = 2 x_1 d\]
        It follows that after $j \leq 2 x_1 d$ steps, we could guarantee that 
        $\abs{A_1 + \ldots + A_{j}} \geq x_1^d$. Choose $i_1 := \myInf{\set{j; \abs{A_1 + \ldots + A_{j}} \geq x_1^d }}$ and set $Y = A_1 + \ldots + A_{i_1}$.
        Now we put $x_2 = 2^2$. Note that at present stage, $\abs{Y} \geq x_1^d$. \\
        Applying again Corollary~\ref{cor:YplusB_lowerBound}, we obtain
        \begin{alignat*}{2}
            \abs{Y + A_{i_1 + 1}} &\geq x_1^d \cdot 2^{\frac{1}{2x_2}}, && Y \leftarrow Y + A_{i_1 + 1}\\
            \vdots\\
            \abs{Y + A_{m}} &\geq x_1^d \cdot 2^{\frac{j - i_{1}}{2x_2}}, & \quad&Y \leftarrow Y + A_{m}
        \end{alignat*}
        Because 
        \[\frac{m}{2x_2} = d \Leftrightarrow m = 2 (i_1 + x_2) d \leq 2(x_1 + x_2) d,\]
        where we have used the fact that $i_1 \leq 2dx_1$ for the inequality. It follows that after $j \leq 2(x_1 + x_2)d$ steps in total,
        we could guarantee that 
        \[\abs{A_1 + \ldots + A_{j}} \geq x_1^d \cdot 2^{d} = x_2^d.\]
        Choose $i_2 := \myInf{\set{j; \abs{A_1 + \ldots + A_{j}} \geq x_2^d }}$ 
        and set $Y = A_1 + \ldots + A_{i_2}$.
        Now we put $x_3 = 2^3$ and do the similar procedure. Note that now $\abs{Y} \geq x_2^d$.
        Since $x$ is a power of $2$, we assume $x = 2^l$. From the discussion above, we could see that after $i_j$ steps,
        $\abs{Y} \geq x_j^d = 2^{jd}$. In particular after $i_l$ steps, $\abs{Y} \geq 2^{ld} = x^d$. In addition, we observe that $i_j \leq 2(x_1 +\ldots + x_j)d$,
        so at the $l$-th recursion, we obtain 
        \begin{align*}
            i_l &\leq 2 (x_1 + \ldots + x_l) d\\
            &= 2 (2 + 2^2 + \cdots + 2^l) d \\
            &\leq 2 \cdot 2^{l+1} \cdot d \\
            &= 4 \cdot 2^l \cdot d = 4xd = s
        \end{align*}
        In particular we see that $\abs{A_1 + \ldots + A_s} \geq x^d.$
    \end{proof}
    We state the following Lemma which gives the estimation of the cardinality of the set $(ia+Y)\setminus Y$.
    A proof can be found in \cite{DBLP_LatticePoint:journals/combinatorica/AlonD95}.
    \begin{lemma}\label{lem:minusA_minusY_and_IA_pYMinusY}
        Let $Y$ be a finite subset of an abelian group $G$.
        Choose an $g \in G$ and $i \in \NaturalNumber$. Then
        \begin{align}
            \abs{(\minus g + Y) \setminus Y} &= \abs{(g + Y) \setminus Y}\\
            \abs{(i g + Y) \setminus Y} &\leq i \cdot \abs{(g + Y)\setminus Y}
        \end{align}
    \end{lemma}

    \subsubsection{characters}
    Throughout this section, we assume $G$ is a finite abelian group, using additive notation. Let $S$ a multiset of $G$, with 
    $S$ being symmetric, which means:
    \[\forall x \in S \Rightarrow \minus x \in S\]
    \begin{definition}
        Let $G$ be an abelian group and $S$ a multiset of $G$, $S$ symmetric. The \emph{Cayley graph} 
        $\cayleyGraph{G}{S}$ is defined as the graph $(G,E)$, with 
        vertices be the group element of $G$ and the edge $E$ defined as:
        \[E := \set{(g,s); g \in G, s \in S}.\] We denote the adjacency matrix of the graph \cayleyGraph{G}{S} as \adjacencyMatrixCayley{G}{S}.
    \end{definition}
    From the property of $S$, we could see that the adjacency matrix $\adjacencyMatrixCayley{G}{S}$ has the following property.
    \begin{proposition}
        $\adjacencyMatrixCayley{G}{S}$ is symmetric.
    \end{proposition}
    We know that since $\adjacencyMatrixCayley{G}{S}$ is symmetric, all its eigenvalues are real. Moreover, it turns out that we can precisely find out each 
    eigenvalue of $\adjacencyMatrixCayley{G}{S}$. For that purpose, I would like to introduce the concept of characters.
    I first introduce the general concept of a character to a general group, then confine to the case of finite abelian group.
    \begin{definition}\label{def:characters}
        Let $\Gamma$ be a group, a function $f : \Gamma \rightarrow \ComplexUnit$ is a \emph{character} of $\Gamma$ if 
        $f$ is a group homomorphism, where $\ComplexUnit$ denotes the multiplicative group of the units of $\Complex$.
        We denote $\characterGroup[\Gamma]$ to be the set of all possible characters of $\Gamma$.
    \end{definition}
    We now turn to the case of finite abelian group $G$. Using the notation from Definition~\ref{def:characters}, we denote the set of all characters of $G$ 
    as $\characterGroup$.
    We could easily see that all $\forall \chi \in \characterGroup$:
    \[\chi (g)^{\abs[]{G}} = \chi(g^{\abs[]{G}}) = \chi (1) = 1,\]
    that is the range of each character must be the $\abs{G}$-th root of unit in $\ComplexUnit$.
    It is helpful to study the property of $\characterGroup$, which we state in the following lemma.
    \begin{lemma}\label{lem:characterGroupIsAGroup}
        $\characterGroup$ is a group with multiplication defined as the composition of characters, the neutral element
        be the characters which maps all elements to $1$, (we denote it as $1$) and the inverse as the conjugation
    \end{lemma}
    \begin{proof}
        We first prove that $\characterGroup$ is closed under multiplication. Because it is easily seen that 
        given $\chi_1, \chi_2$, $\chi := \composition{\chi_1}{\chi_2}$ satisfies the definition of a character.
        Let $\chi \in \characterGroup$, we could also easily see that $\chi^{*}$ is also a character.
        Furthermore, we have 
        \[\chi(g) \cdot \chi^{*}(g) = 1, \quad \forall g \in G\]
        clearly $\chi^{*} = \chi^{-1}$.
    \end{proof}
    An interesting property of the character is given in the following lemma.
    \begin{lemma}\label{lem:character_1st_Property}
        Let $\chi \in \characterGroup$ which is not identitcal to $1$-mapping, i.e., maps every element to $1$.
        Then
        \[\sum_{g \in G} \chi(g) = 0\]
    \end{lemma}
    \begin{proof}
        Since $\chi$ is not identical to $1$, $\exists h \in G$, with $\chi(h) \neq 1$. We obtain
        \[\chi(h) \cdot \sum_{g\in G}\chi(g) = \sum_{g \in G} \chi(g + h) = \sum_{g \in G} \chi (g)\]
        Reforming the equation we obtain:
        \[(\chi(h) - 1) \cdot \sum_{g \in G} \chi(g).\]
        Since $\chi(h) \neq 1$, it follows that $\sum_{g \in G} \chi(g) = 0$.
    \end{proof}
    We define an operator 
    \begin{align*}
    \bilinearForm{\cdot}{\cdot}: \characterGroup \times \characterGroup &\rightarrow \ComplexUnit\\
    \bilinearForm{f}{h} &\mapsto \sum_{g\in G} f(g) h^{*}(g),
    \end{align*}
    where $*$ denotes the conjugation.
    The characters have the following property:
    \begin{lemma}\label{lem:character_2nd_Sum_Property}
        Let $\chi_1,\chi_2$ be two characters of $G$, which are not equal. Then
        \[\bilinearForm{\chi_1}{\chi_2} = 0\]
    \end{lemma}
    \begin{proof}
        We know from Lemma~\ref{lem:characterGroupIsAGroup} that $\composition{\chi_1}{\chi_2^{*}}$ is also a character.
        Since $\chi_1 \neq \chi_2$, it follows that $\exists g \in G$, s.t. $\chi_1(g) \neq \chi_2(g)$
        \[\composition{\chi_1}{\chi_2^{*}}(g) \neq 1,\]
        which means that $\composition{\chi_1}{\chi_2^{*}} $ is not $1$-mapping.
        Using Lemma~\ref{lem:character_1st_Property}, we obtain
        \begin{align*}
            \bilinearForm{\chi_1}{\chi_2} &= \sum_{g \in G} \chi_1(g) \cdot \chi_2^{*}(g)\\
            &= \sum_{g \in G} \composition{\chi_1}{\chi_2^{*}(g)} \\
            &= 0.
        \end{align*}
    \end{proof}

    Since all the functions of $\Complex^{G}$ can be treated as a \Complex-vector space, and the characters 
    can be seen as a special case of $\Complex^{G}$ by extending the range to $\Complex$, we could find a natural bound of the cardinality of $\characterGroup$.
    \begin{proposition}\label{prop:characterGroupNaturalBound}
        $\abs{\characterGroup} \leq \abs{G}$
    \end{proposition}
    \begin{proof}
        From the above discussion, we see that $\Complex^{G}$ has dimension $\abs{G}$.
        We extend the bilinear form $\bilinearForm{\cdot}{\cdot}$ to $\Complex^{G} \times \Complex^{G}$, 
        we see that this is a scalar product for the vector space.
        From Lemma~\ref{lem:character_2nd_Sum_Property}, we see that different characters are orthogonal to each other.
        In particular linear independent, the bound follows from the fact that the number of linear independent vectors is bounded by the dimension of the vector space.
    \end{proof}
    Next we would like to prove that $\abs{\characterGroup} = \abs{G}$.
    We recall the structure theorem of finite abelian group that 
    \begin{equation*}
        G \cong \modularInteger{n_1} \bigoplus \cdots \bigoplus \modularInteger{n_s} ,\quad\hfill \hbox{for some }n_1, \ldots, n_s \in \NaturalNumber
    \end{equation*}
    It can be easily seen that if $\chi_1 \in \characterGroup{G_1}$ and $\chi_2 \in \characterGroup{G_2}$,
    then  is a character on $G_1 \times G_2$. 
    The injectiveness of the map
    \begin{align*}
        \characterGroup[G_1] \times \characterGroup[G_2] &\rightarrow \characterGroup[G_1 \times G_2]\\
        \chi  \chi(g_1,g_2) &:= \chi_1(g_1)\cdot \chi(g_2)
    \end{align*} 
    is clear.
    We state the result in a proposition.
    \begin{proposition}\label{prop:directSum_characterInjective}
        Let $G_1,G_2$ be two finite abelian groups and $\characterGroup{G_1},\characterGroup{G_2}$ the corresponding
        character groups. The map:
        \begin{align*}
            \characterGroup[G_1] \times \characterGroup[G_2] &\rightarrow \characterGroup[G_1 \times G_2]\\
            (\chi_1, \chi_2) &\mapsto  \chi: (g \mapsto \chi_1(g) \cdot \chi_2(g))
        \end{align*} 
        is injective.
    \end{proposition}
    Once we have proved that $\abs{\characterGroup[G_i]} = \abs{G_i}, i \in \set{1,2}$, we can obtain from Proposition~\ref{prop:directSum_characterInjective}
    that $\abs{\characterGroup} \geq \abs{G}$, combining the result with Proposition~\ref{prop:characterGroupNaturalBound}, we could obtain that 
    \[\abs{\characterGroup} = \abs{G}\]
    So we only need to focus on a single decomposition.
    \begin{lemma}\label{lem:ZnModulo_character}
        The character group of \modularInteger{n} has cardinality $n$.
    \end{lemma}
    \begin{proof}
        Each character $\chi$ can be uniquely defined by $\chi(1)$ since $\modularInteger{n}$ is cycle.
        The map
        \[\chi(1) \mapsto \exp ^{\frac{2\pi k}{n}}, \quad k = 0,\ldots, n\minus 1 \]
        uniquely determines $n$ characters.
    \end{proof}
    Combining Proposition~\ref{prop:characterGroupNaturalBound} and Lemma ~\ref{lem:ZnModulo_character},
    we can find the number character groups of a finite abelian group which we state in the following lemma.
    \begin{lemma}\label{lem:characterGroup_Finite_AbelianGroup_Cardinality}
         For every finite abelian group $G$, it holds that 
         \[\abs{\characterGroup} = \abs{G}\]
    \end{lemma}
    Now we can examine the eigenvalues of the adjacency matrix of the cayley graph.
    With the help of characters, the eigenvalues can be easily computed:
    \begin{proposition}\label{prop:adjacency_eigenvalue_character}
        Let $\adjacencyMatrixCayley{G}{S}$ be the adjacency matrix of a cayley graph \cayleyGraph{G}{S}.
        Then all the eigenvalues of $\adjacencyMatrixCayley{G}{S}$ is given by
        \begin{equation}\label{eqn:cayleygraph_Eigenvalue}
            \sum_{s\in S} \chi(s)
        \end{equation}
        where $\chi \in \characterGroup$.
        The corresponding eigenvector is given as
\[        \begin{pmatrix}
            \chi(g_1)\\
            \vdots\\
            \chi(g_n)
        \end{pmatrix},\]
        where we have enumerated the elements of $G$ as $g_1, \ldots, g_n$.
    \end{proposition}
    \begin{proof}
        Since $G$ is an finite abelian group. From Lemma~\ref{lem:characterGroup_Finite_AbelianGroup_Cardinality},
        we know that $\abs{characterGroup} =  \abs{G}$. In particular 
        the vector 
        \[v_{\chi} := \begin{pmatrix}
            \chi(g_1)\\
            \vdots\\
            \chi(g_n)
        \end{pmatrix},\]
        is dinstict for diffrent $\chi \in \characterGroup$. Since $\adjacencyMatrixCayley{G}{S}$ is symmetric,
        it has exactly $n$ eigenvectors. We are done if we prove that
        the eigenvalue of $v_{\chi}$ is given by \eqref{eqn:cayleygraph_Eigenvalue}. 
        We define $A := \adjacencyMatrixCayley{G}{S}$  and $\chi_g := \chi(g)$ for convenience of notation.
        \begin{align*}
            (A v_\chi)_{g} &=  \sum_h A_{g,h} v_{\chi,h}\\
            &= \sum_{s \in S} \chi(g + s)\\
            &= \chi_g \cdot \sum_{s \in S} \chi(s)
        \end{align*}
    \end{proof}
    Since $\cayleyGraph{G}{S}$ is $\abs{S}$-regular. It has an eigenvalue of $\abs{S}$.
    Moreover, based on Proposition~\ref{prop:adjacency_eigenvalue_character}, we deduce the following proposition.
    \begin{proposition}\label{prop:cayleyGraph_LargestEigenvalue}
        The largest eigenvalue of $\adjacencyMatrixCayley{G}{S}$ is $\abs{S}$.
    \end{proposition}
    \begin{proof}
        From Proposition~\ref{prop:adjacency_eigenvalue_character}, all the eigenvalue is given by
        \[\sum_{s\in S} \chi(s)\]
        Since the range of the character is constrained in the unit circle in the complex plane.
        It follows that for an arbitrary eigenvalue $\alpha$ of $\adjacencyMatrixCayley{G}{S}$
        \begin{align*}
            \abs{\alpha} &= \abs{\sum_{s\in S} \chi(s)} \\
            &\leq             \sum_{s\in S} \abs{\chi(s)}\\
            &\leq             \sum_{s\in S} 1\\
            &= \abs{S}
        \end{align*}
    \end{proof}
    % \begin{remark}
    %     From the proof we could also see that the subspace spanned by the eigenvectors corresponding to $\abs{S}$ 
    %     is one-dimensional. 
    % \end{remark}
    We state the following important lemma which gives a lowerbound between number of edges that spans two disjoint set of the nodes.
    A proof of the Lemma can be found in \cite{alonProbMe}.
    \begin{lemma}\label{lem:largest_second_Eigenvalue_expander}
        Let $\lambda$ be the largest eigenvalue of $\adjacencyMatrixCayley{G}{S}$, and $\mu$ the second largest eigenvalue.
        Let $A \subset G$ and $B = G \setminus A$. We define
        \[e(A,B) = \abs{\set{(u,v); u \in A, v \in B}},\]
        i.e., the number of edges that have one end in $A$ and another end in $B$.
        It holds that $e(A,B)$ has a lower bound given by the following equaiton
        \begin{equation*}
            e(A,B) \geq (\lambda - \mu) \cdot \frac{\abs{A} \abs{B}}{\abs{G}}
        \end{equation*}
    \end{lemma}
    We now state a lemma which is useful for the proof of the \alonDubinerTheorem.
    \begin{lemma}\label{lem:a_plus_Y_M_Y_cardinality}
        Suppose $W \geq 1$, let $A$ be a sequence of elements in $\IntegerP{p}^d$ (possible to have equal elements) and 
        suppose no hyperplanes in $\IntegerP{p}^d$ contains more than $\frac{\abs[]{A}}{4W}$ members of $A$. Then $\forall Y \subset \IntegerP{p}^d$,
        with $\abs{Y} \leq \frac{p^d}{2}$, $\exists a \in A$, s.t.
        \begin{equation}\label{eqn:a_plus_Y_minus_Y_lowerBound}
            \abs{(a + Y)\mySetMinus Y} \geq \frac{W}{16p}\abs{Y}
        \end{equation}            
    \end{lemma}
    \begin{remark}
        It follows from the condition that 
        \begin{align*}
            p \cdot \frac{\abs{A}}{4W} &\geq \abs{A}\\
            \frac{p}{4} &\geq W.
        \end{align*}
    \end{remark}
    \begin{proof}
        We put $G = \IntegerP{p}^d$. The idea is that we construct $Z := G \setminus Y$ and use 
        Lemma~\ref{lem:a_plus_Y_M_Y_cardinality} to find an element which satisfies \eqref{eqn:a_plus_Y_minus_Y_lowerBound}
        We define the multiset $S_a$ for every $a \in A$.
        \begin{equation*}
            S_a := \set{a,\ldots, 2\roundCeil{\frac{p}{W}}a} \cup \set{-a,\ldots, -2\roundCeil{\frac{p}{W}}a}.
        \end{equation*}
        We define the multiset $S$ as
        \begin{equation*}
            S := \bigcup_{a \in A} S_a.
        \end{equation*}
        We construct the cayley graph $\cayleyGraph{G}{S}$ and from Proposition~\ref{prop:cayleyGraph_LargestEigenvalue},
        we see that the largest eigenvalue $\lambda = \abs{S} = 4\roundCeil{\frac{p}{W}} \cdot \abs{A}$.
        We would like to find out the upper bound of the second largest eigenvalue.
        Recall from Proposition~\ref{prop:directSum_characterInjective} that 
        for each character $\chi$, it has a from of:
        \begin{equation*}
            \chi((g_1,\ldots,V_d)) = \prod_{l = 1}^{d} \chi_{l}(g_l),
        \end{equation*}
        where we have decomposed each element of $g \in G$ in to the component in the direct sum, and 
        $\chi_l$  is the character of the group $\IntegerP{p}$.
        From Lemma~\ref{lem:ZnModulo_character}, we see that each $\chi_l$ has the form of 
        \begin{equation*}
            \chi_l(g_l) = \exp (\frac{2\pi i \cdot g_l v_l}{p}),
        \end{equation*}
        for some $v_l \in \IntegerP{p}$.
        As a consequence we can write the $\chi$ as the following:
        \begin{align*}
            \chi(g) &= \prod_{l = 1}^{d} \exp (\frac{2\pi i \cdot v_l}{p}) \\
            &= \exp (\frac{2 \pi i \sum_lg_l v_l}{p})
        \end{align*}
        For some $v := (v_1,\ldots,v_d)$ in $\IntegerP{p}^d$. From Lemma~\ref{lem:characterGroup_Finite_AbelianGroup_Cardinality}
        we could see that each such $v$ determines uniquely a character, thus we can specify every character with a vector $v \in \IntegerP{p}^d$. 
        From Proposition~\ref{prop:adjacency_eigenvalue_character}, the eigenvalue is given by
        \begin{equation*}
            \sum_{s\in S} \omega^{v \cdot s} = \sum_{a \in S_a} \sum_{s \in S_a} \omega^{v \cdot s},
        \end{equation*}
        where $\omega = \exp (\frac{2\pi i}{p})$, and $\cdot$ denotes the standard scalar product. Since we want to find out the bound of second largest eigenvalue,
        we set $v \neq 0$.($v = 0$ leads to the largest eigenvalue).
        We can choose the range of the dot product so that for $a \in A$, 
        \[\frac{\minus p}{2} < v \cdot a \leq \frac{p}{2}\]
        By assumption for each $c \in \IntegerP{p}$, there are at most $\frac{\abs{A}}{4W}$ elements $a \in A$ so that
        $v\cdot a = c$. In particular:
        \begin{align*}
            \abs{\set{a \in A; v \cdot < W}} &\leq \frac{\abs{A}}{4W} \cdot (W-1 - (-W + 1) + 1)\\
            &= \frac{\abs{A}}{4W} \cdot (2W-1) < \frac{\abs{A}}{2},
        \end{align*}
        which means that there are at least $\frac{\abs[]{A}}{2}$ elements $a \in A$, such that
        $v \cdot a \geq \abs[]{W}$. For each such $a$, if $k = \abs[]{v \cdot a}$ and $l = 2 \roundCeil{\frac{p}{W}}$, 
        then
        \begin{align*}
            \sum_{s \in S_a} \omega^{v\cdot s} &= \sum_{j = 1}^{l} \omega^(k \cdot j) + \sum_{j = 1}^{l} \omega^(-k \cdot j) \\
            &= \frac{\omega^k (1 - \omega^{kl})}{1 - \omega^k} + \frac{\omega^{\minus k} (1 - \omega^{kl})}{1 - \omega^{\minus k}}.
        \end{align*}
        The amplitude is thus bounded by
        \begin{equation}\label{ieqn:character_boundedBy_l}
            \begin{aligned}[b]
                \abs{\sum_{s \in S_a} \omega^{v\cdot s}} &\leq \frac{\abs{\omega^k} 2 \cdot \abs{1 - \omega^{kl}}}{\abs{1 - \omega^k}}\\
                &\leq \frac{4}{\abs{1 - \omega^k}}\\
                &\stackrel{(*)}{\leq} \frac{4}{\frac{\pi W}{p}} = \frac{4}{\pi} \cdot \frac{p}{W}\\
                &< \frac{2 p}{W} \leq l.
            \end{aligned}
        \end{equation}
        Where we have used the fact at $(*)$ that 
        \[\abs{1 - \omega^k}^2 \geq 2 - 2\cos (2\theta), \quad \theta = \frac{\pi W}{p}\]
        By analysing the difference of the squared amplitude, we could see that see that
        \[2 - 2\cos (2\theta) -  \theta^2 \geq 0, \quad \theta \in (0, \frac{\pi}{2})\]
        which means that $\abs{1 - \omega^k} \geq \frac{\pi W}{p}$.\\
        Now since \eqref{ieqn:character_boundedBy_l} holds for at least $\frac{\abs[]{A}}{2}$ elements, we set
        the multiset 
        \[B := \set{a \in A; v \cdot a \geq W},\]
        and the multiset $C$ which is a submultiset of $B$ which has cardinality $\frac{\abs{A}}{2}$.
        we obtain that 
        % \begin{align*}
        %     \abs{\sum_{a \in S_a} \sum_{s \in S_a} \omega^{v \cdot s}} &\leq 
        %     \abs{\sum_{\substack{a \in S_a\\\abs[]{v\cdot a < W}}} \sum_{s \in S_a}\omega^{v \cdot s}} + 
        %     \abs{\sum_{\substack{a \in S_a\\\abs[]{v\cdot a \geq W}}} \sum_{s \in S_a}\omega^{v \cdot s}}\\
        %     &\leq \abs{\sum_{\substack{a \in S_a\\\abs[]{v\cdot a < W}}} \sum_{s \in S_a}1} + 
        %     \abs{\sum_{\substack{a \in S_a\\\abs[]{v\cdot a \geq W}}} \sum_{s \in S_a}\omega^{v \cdot s}}\\
        %     &\leq \abs{\sum_{\substack{a \in S_a\\\abs[]{v\cdot a < W}}} 2l} + 
        %     \abs{\sum_{\substack{a \in S_a\\\abs[]{v\cdot a \geq W}}} \sum_{s \in S_a}\omega^{v \cdot s}}\\
        %     &\leq \frac{\abs{A}}{2} \cdot l + \abs{A} \cdot l \\
        % \end{align*}
        \begin{align*}
            \abs{\sum_{a \in S_a} \sum_{s \in S_a} \omega^{v \cdot s}} &\leq 
            \abs{\sum_{a \in C } \sum_{s \in S_a}\omega^{v \cdot s}} + 
            \abs{\sum_{a \in A \setminus C} \sum_{s \in S_a}\omega^{v \cdot s}}\\
            &\leq \sum_{ a \in A \setminus C} \abs{\sum_{s \in S_a}\omega^{v \cdot s}} + 
            \sum_{a \in A \setminus C} \sum_{s \in S_a}\abs{\omega^{v \cdot s}}\\
            &\leq \sum_{a \in C} l + 
            \sum_{a \in A \setminus C} \sum_{s \in S_a}1\\
            &\leq \frac{\abs{A}}{2} \cdot l + \frac{\abs{A}}{2} \cdot 2l \\
            &\leq \frac{3\abs{A}}{2}l = \frac{3}{4} \cdot \lambda \\
        \end{align*}        
        Let $Y \subset G$ with $\abs{Y} \leq \frac{\abs[]{G}}{2}$ and define $Z := G \setminus Y$.
        From Lemma~\ref{lem:largest_second_Eigenvalue_expander}, we know that $e(Y,Z)$, i.e., the number of edges which has one end in $Y$ and the other in $Z$
        has a lower bound:
        \begin{align*}
            e(Y,Z) &\geq (\lambda - \mu) \frac{\abs{Y} \abs{Z}}{\abs{G}}\\
            &\geq \frac{\lambda}{4}\frac{\abs{Y} \frac{\abs[]{G}}{2}}{\abs{G}}\\
            &\geq \lambda \cdot \frac{\abs{Y}}{8}.
        \end{align*}
        This implies that the set $M:= \set{(y,s); y \in Y, s \in S, y + s \not\in Y}$ has at least $\lambda \cdot \frac{\abs{Y}}{8}$.
       elements. We could observe that by averaging, there is at least one element $s \in S$, s.t. $M_s := \set{y \in Y; y + s \not\in Y}$
       has more than $\frac{\abs[]{Y}}{8}$ elements, because otherwise $\forall s \in S, \abs{M_s} < \frac{\abs[]{Y}}{8}$. It follows that
       \begin{align*}
           \abs{M} &= \abs{\bigcup_{s \in S} M_s}\\
           &\leq \sum_{s\in S} \abs{M_s}\\
           &< \sum_{s \in S} \frac{\abs{Y}}{8}\\
           &= \lambda \cdot \frac{\abs{Y}}{8} \; \contradiction
       \end{align*}
        Choose such $s$, we know from the construction of the set $S$ that $s \in S_a$ for some $a \in A$, that is 
        $\exists \epsilon, i \in \NaturalNumber, i \leq 2 \roundCeil{\frac{p}{W}}$, s.t. $s = \epsilon i a$. 
        From Lemma~\ref{lem:minusA_minusY_and_IA_pYMinusY}, we know that 
        \begin{align*}
            \abs{(s + Y) \setminus Y} &= \abs{(\epsilon i a + Y)\setminus Y}\\
                &= \abs{(i a + Y) \setminus Y} \\
                &\leq i \cdot {(a + Y) \setminus Y},
        \end{align*}
        which follows that 
        \[\abs{(a + Y)\setminus Y} \geq \frac{\abs{Y}}{8 \cdot i} \geq \frac{\abs{Y}}{8 \cdot 2 \roundCeil{\frac{p}{W}}} \geq \frac{W}{16p}\abs{Y} \]
    \end{proof}


    \subsection{Proof of \alonDubinerTheorem}
    We can now prove the \alonDubinerTheorem{}. We first note that it suffices to prove the theorem
    for the prime number. Because if $\fnd{p}{d} \leq cp \leq 2cp - (2c - 1)$, then it follows from 
    \eqref{eqn:fndProp2} that for $n = pq$:
    \begin{align*}
        \fnd{pq}{d} \leq \fnd{p}{d} + p(\fnd{q}{d} - 1) &\leq 2cp - 2c + 1 + p (2cq - 2cp) \\
        &= 2cn - (2c - 1) \leq 2cn
    \end{align*}
    As a consequence, we only prove the case when $n = p$ for $p$ a prime number.
    It is worth noting that the constant $c$ is fixed given a dimension $d$, but may vary once the dimesion $d$ changes.\\
    That is :
    \begin{equation}\label{eqn:alonFndpBound}
        \fnd{p}{d}\leq c(d) p
    \end{equation}
    We prove \eqref{eqn:alonFndpBound} by induction on $d$, with the constant $c(d)$ are defined by the following scheme:
    \begin{equation}\label{eqn:alonCd_RecursiveForm}
        c(1) = 2, \; c(d) = 256(d\log_2 d + 5) c(d-1) + (d+1),\hfill \hbox{for } d\geq 2
    \end{equation}
    We first check that the $c(d)$ satisfy the condition that $c(d) \leq (cd\log_2 d)^d$ for some absolute constant c.
    Since by induction:
    \begin{align*}
        c(d) &= 256(d\log_2 d + 5) c(d-1) + (d+1) \\
            &\leq 256(d\log_2 d + 5) (c(d-1) \log_2 (d-1))^{d-1} + (d + 1)\\
            &\leq 256(d\log_2 d + 5) (cd \log_2 d)^{d-1} + (d + 1)\\
            & \leq (256^{1/d}c^{\frac{d-1}{d}} d \log_2 d)^{d} + \bigO{(d\log_2 d)^d}\\
            &\leq (\tilde{c}d\log_2 d)^d
    \end{align*}
    Besides, we could see that for the case $p \leq 32d$, \eqref{eqn:alonFndpBound} is already true, because:
    \begin{align*}
        \fnd{p}{d}\stackrel{(*)}{\leq} (p-1)\cdot p^d + 1 &= p^d \cdot p - p^d + 1\\
        &\leq (32d)^d \cdot p \\
        & \leq (32 d \log_2 d)^d p,
    \end{align*}
    where we have used \eqref{eqn:fndProp1} at (*).
    We have already seen from Theorem~\ref{theorem:Erdos2NM1} that $\fnd{p}{1} = 2p-1$, so \eqref{eqn:alonFndpBound} holds for $d = 1, c(1) = 2$.
    Assuming the correctedness for $d-1$, we now prove the case $d$. We denote $S$ be a sequence of $c(d)p$ elements from $\IntegerP{p}^d$.
    We consider two cases:
    \begin{enumerate}
        \item $\exists V \subset S$, with $\abs{V} \leq c(d-1) p$, s.t all the elements of $V$ lies on a hyperplane of $\IntegerP{p}^d$.
        \item There are no such $V$s.
    \end{enumerate}
    For the first case, we can use the induction hypothesis to find a \zeroSumSeq{p} in $V$, thus also in $S$. \\
    For the second case, we want to show that there exists pairwise disjoint subsets $A_1,\ldots,A_s, B_1,\ldots,B_t, \myAlphabetSubSupscript{A}{1}{'},\ldots,\myAlphabetSubSupscript{A}{s'}{'},B_1^{'},\ldots,B_{t'}^{'}$ of $S$,
    with the following properties:
    % \begin{property}
    \begin{enumerate}
        \item $\abs{A_i} = \abs{A_j^{'}} = s+1, \quad \forall i = 1,\ldots,s, j = 1,\ldots, s'$
        \item $\abs{B_i} = \abs{B_j^{'}} = 2, \quad \forall i = 1, \ldots,t, j = 1,\ldots,t'$
        \item $ s + t + s' + t' \leq p$
        \item $\abs{A_1 + \ldots + A_s + B_1 + \ldots + B_t} > \frac{p^d}{2}$
        \item $\abs{A_1^{'} + \ldots + A_{s'}^{'} + B_1^{'} + \ldots + B_{t'}^{'}} > \frac{p^d}{2}$
    \end{enumerate}
    % \end{property}
    From property 5 and 6, we could see from Lemma~\ref{lem:A_B_SumToG} that 
    \begin{equation}\label{eqn:A_B_setSum_WholeSet}
        A_1 + \ldots + A_s + B_1 + \ldots + B_t + A_1^{'} + \ldots + A_{s'}^{'} + B_1^{'} + \ldots + B_{t'}^{'} = \IntegerP{p}^d.
    \end{equation}
    which means, every element from $\IntegerP{p}^d$ can be written as a sum of $s + t + s' + t'$ elements in $S$. By choosing arbitrarily $m := p - s + t + s' + t'$ elements in
    $S$ $s_1, \ldots, s_m$, that do not lie in $A_i, \myAlphabetSubSupscript{A}{i'}{}, B_j, \myAlphabetSubSupscript{B}{j'}{}$, we can see that from \eqref{eqn:A_B_setSum_WholeSet} that $\exists a_i \ in A_i, a_{i'}^{'} \in A_{i'}, b_j \in B_j, b_{j'}^{'} \in B_{j'}$, s.t
    \begin{equation*}
        -(s_1 + \ldots + s_m) = a_1 + \ldots + a_s + b_1 + \ldots + b_t + a_1^{'} + \ldots + \myAlphabetSubSupscript{a}{s'}{'} + \myAlphabetSubSupscript{b}{1}{'} + \ldots + \myAlphabetSubSupscript{b}{t'}{'}
    \end{equation*}
    By reforming the equation above we obtain
    \begin{equation*}
        0 = a_1 + \ldots + a_s + b_1 + \ldots + b_t + a_1^{'} \\+ \ldots + \myAlphabetSubSupscript{a}{s'}{'} + \myAlphabetSubSupscript{b}{1}{'} + \ldots + \myAlphabetSubSupscript{b}{t'}{'} + s_1 + \ldots + s_m 
    \end{equation*}
    which means, we have found a \zeroSumSeq{p} in $S$.
    It remains to prove the existence of the sets $A_i, \myAlphabetSubSupscript{A}{i'}{}, B_j, \myAlphabetSubSupscript{B}{j'}{}$. We will construct them one by one.
    We first define $W = 64(d\log_2 d + 5 d)$ and we see from~\eqref{eqn:alonCd_RecursiveForm}:
    \begin{equation}\label{eqn:cd_cdM1_W_prop}
        \frac{c(d) p - p (d + 1)}{4W} = c(d - 1) p > \abs{V \cap S} \quad \forall V \hbox{ hyperplane in } \IntegerP{p}^d.
    \end{equation}
    This equation implies that even after we have deleted at most $p(d+1)$ elements from the set $S$, we can still make sure that no hyperplanes in $\IntegerP{p}^d$ contains
    $\frac{1}{4W}$ of the elements of the remaining sets. By property 1, 2, 3, we see that
    \[\abs{A_1 \cup \ldots \cup A_s \cup B_1 \cup \ldots \cup B_t \cup A_1^{'} \cup \ldots \cup A_{s'}^{'} \cup B_1^{'} \cup \ldots \cup B_{t'}^{'}} \leq p(d+1)\]
    which means that the condition for the 
    Lemma~\ref{lem:a_plus_Y_M_Y_cardinality} 
    will hold throughout the selection of the subsets $A_i, \myAlphabetSubSupscript{A}{i'}{}, B_j, \myAlphabetSubSupscript{B}{j'}{}$.

    We now begin to construct $A_i$. First, we find $x$ be a power of $2$ which satisfies:
    \[\frac{p}{32d} \leq x \leq \frac{p}{16d}.\]
    Put $s = s' = 4xd$, which leads to $s \leq \frac{p}{4}$. Let $A_1,\ldots,A_s$ and $\myAlphabetSubSupscript{A}{1}{'},\ldots,\myAlphabetSubSupscript{A}{s'}{'}$ be pairwise disjoint affine basis of $\IntegerP{p}^d$, each be a subset of $S$.
    The existence of those sets is justifed by ~\eqref{eqn:cd_cdM1_W_prop}, because it means that the remaing set of $S$ does not lie on a hyperplane.(since each hyperplane contains less than $\frac{1}{4W}$ elements of the remaing set).
    By Proposition~\ref{prop:A_sets_sum_Cardinality_lowerBound}
    \begin{equation}
        \abs{A_1 + \ldots + A_s} \geq x^d \geq (\frac{p}{32})^d.
    \end{equation}
    and similar inequality holds for $\myAlphabetSubSupscript{A}{1}{'},\ldots,\myAlphabetSubSupscript{A}{s'}{'}$.
    We next want to find the subset $B_1,\ldots,B_t$. We put $Y = A_1 + \ldots + A_s$. If $\abs{Y} \geq \frac{p^d}{2}$ we are already done,
    since we could just set $t = 0$ to satisfy property 4. 

    \begin{equation*}
        \abs{(a + Y)\cup Y} \geq (1 + \frac{W}{16p}) \abs{Y}.
    \end{equation*}
    Define $B_1 = {a + s', s'} \subset S'$, and observe that
    \begin{equation*}
        \abs{Y + B_1} = \abs{(a + Y)\cup Y} = \abs{(a+Y)\mySetMinus Y} + \abs{Y} \geq (1 + \frac{W}{16p}) \abs{Y}.
    \end{equation*}
    Next, we update $Y = A_1 + \ldots + A_s + B_1$ and update $S'$ to be $S' \mySetMinus B_1$. We stop this iterative process until
    the set $Y$ has cardinality $\abs{Y} > \frac{p^d}{2}$. We could check that the total step $t$ will be bounded by $\frac{p}{4}$.
    \begin{equation*}
        \frac{W}{16p}^k \cdot (\frac{p}{32d})^d \leq (1 + \frac{W}{16p})^k \abs{ A_1 + \ldots + A_s} \leq \frac{p^d}{2}\\
    \end{equation*}
    We deduce that
    \begin{align*}
        \frac{W}{16p}^k \cdot (\frac{p}{32d})^d &\leq  \frac{p^d}{2}\\
        k &\leq \frac{16p}{W}\log_2 (\frac{(32d)^d}{2})\\
          &\leq \frac{16p}{W}\log_2 (32d)^d\\
          &= \frac{16p}{64(d\log_2 d + 5 d)} \cdot (d\log_2 d + 5d) \\
          &= \frac{p}{4}
    \end{align*}
    The set $\myAlphabetSubSupscript{B}{j}{'}$ can be found similarly. We see that all the four variables $s,t,s',t'$ are bounded by $\frac{p}{4}$,
    which means that $ s + t + s' + t' \leq p$.
\newpage
    \section{Conclusion}
    In this thesis, I have studied the number $\fnd{n}{d}$ and stated the proof of the \kemnitzConjecture{}. 
    We see the generalization to the higher dimension which states that $\fnd{n}{d}$ is bounded linearly with respect to $n$. 
    Further problems maybe to determine the subsequence in the $2$-dimensional case which is guaranteed to be $4n-3$. It is also conjectured that
    $\fnd{n}{d} \leq c^d n$ for all $n$ and $d$. 
    \newpage
    \bibliography{reference}
    \bibliographystyle{ieeetr}    
\end{document}